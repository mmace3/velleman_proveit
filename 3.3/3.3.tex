\documentclass{article}

%\documentclass{scrartcl}
%\usepackage{changepage}
%\usepackage{scrextend}

\usepackage{amssymb,amsmath,amsthm}
% amssymb has empty set symbo
\usepackage{scrextend} % for \begin{addmargin}[0.55cm]{0cm} text \end{margin}

\usepackage{mathrsfs} % for \mathscr{P}
\usepackage{float}


\newcommand{\n}{ \noindent }
\newcommand{\F}{\mathcal{F}}
\newcommand{\G}{\mathcal{G}}
%\newcommand{\pwset}{\mathcal{P}}
\newcommand{\pwset}{\mathscr{P}}


\newtheorem*{theorem}{Theorem}  % This enables \begin{theorem}

%\DeclareFontFamily{U}{MnSymbolC}{}
%\DeclareSymbolFont{MnSyC}{U}{MnSymbolC}{m}{n}
%\DeclareFontShape{U}{MnSymbolC}{m}{n}{
%  <-6>    MnSymbolC5
%  <6-7>   MnSymbolC6
%  <7-8>   MnSymbolC7
%  <8-9>   MnSymbolC8
%  <9-10>  MnSymbolC9
%  <10-12> MnSymbolC10
%  <12->   MnSymbolC12%
%}{}
%\DeclareMathSymbol{\powerset}{\mathord}{MnSyC}{180}

%\usepackage[left=2cm,right=2cm,top=2cm,bottom=2cm]{geometry}

\begin{document}

\section*{Exercise 3.3.4}
\n Suppose $A \subseteq \pwset(A)$. Prove that $\pwset(A) \subseteq \pwset(\pwset(A))$. \\


\n So we want to prove that
$\forall x(x \in \pwset(A) \rightarrow x \in \pwset(\pwset(A)))$.\\

\n First we assume $x$ is arbitrary and make the antecedent a given and the consequent our goal to prove. \\

\begin{table}[h]
\begin{tabular}{ll}
\hline
Givens & Goals   \\ \hline
$A \subseteq \pwset(A)$ & $x \in \pwset(\pwset(A))$  \\ 
$x \in \pwset(A)$ & \\ \hline
\end{tabular}
\end{table}

\n Assume $x$ is an arbitrary element of $\pwset(A)$ \\
\indent Suppose $x \in \pwset(A)$ \\
\indent \indent [ proof of $x \in \pwset(\pwset(A))$ ] \\
\indent Therefore if $x \in \pwset(A) \rightarrow x \in \pwset(\pwset(A))$ \\
\n Since $x$ was arbitrary we can conclude $\forall x(x \in \pwset(A) \rightarrow x \in \pwset(\pwset(A)))$ \\

\n We can rewrite our goal as $x \subseteq \pwset(A)$ or $\forall y(y \in x \rightarrow y \in \pwset(A))$. So we assume $y$ is arbitrary and make the antecedent a given and the consequent our goal to prove.

\begin{table}[h]
\begin{tabular}{ll}
\hline
Givens & Goals   \\ \hline
$A \subseteq \pwset(A)$ & $y \in \pwset(A)$  \\ 
$x \in \pwset(A)$ & \\ 
$y \in x$ & \\ \hline
\end{tabular}
\end{table}

\n Assume $x$ is an arbitrary element of $\pwset(A)$ \\
\indent Suppose $x \in \pwset(A)$ \\
\indent \indent Suppose $y$ is an arbitrary element of $x$. \\
\indent \indent \indent Suppose $y \in x$. \\
\indent \indent \indent \indent[ proof of $y \in \pwset(A)$ ] \\
\indent \indent \indent Therefore if $y \in x \rightarrow y \in \pwset(A)$.\\
\indent \indent Since $y$ was arbitrary we can conclude that $x \subseteq \pwset(A)$. \\
\indent Therefore if $x \in \pwset(A) \rightarrow x \in \pwset(\pwset(A))$ \\
\n Since $x$ was arbitrary we can conclude $\forall x(x \in \pwset(A) \rightarrow x \in \pwset(\pwset(A)))$ \\

\n Now looking at our givens $x \in \pwset(A)$ means that $x \subseteq A$ or $\forall z(z \in x \rightarrow z \in A)$. Using universal instantiation we will plug in $y$ for $z$ and using modus ponens we can conclude that $y \in A$. \\

\n Now looking at our other given $A \subseteq \pwset(A) \rightarrow \forall m(m \in A \rightarrow m \in \pwset(A))$. Using universal instantiation we will plug in $y$ for $m$ and using modus ponens we can conclude that $y \in \pwset(A)$, which was our goal to prove.

\begin{theorem} Suppose $A \subseteq \pwset(A)$. Then $\pwset(A) \subseteq \pwset(\pwset(A))$.
\end{theorem}
\begin{proof}
Suppose $x$ is an arbitrary element of $\pwset(A)$ and $y$ is an arbitrary element of $x$. It follows that $y \in A$. But since $A \subseteq \pwset(A)$ then it also follows that $y \in \pwset(A)$. So $y \in x \rightarrow y \in \pwset(A)$ and since $y$ was arbitrary we can conclude that $x \subseteq \pwset(A)$. Therefore, if $x \in \pwset(A) \rightarrow x \in \pwset(\pwset(A))$. Since $x$ was arbitrary we can also conclude that $\pwset(A) \subseteq \pwset(\pwset(A))$.
\end{proof}

\n Alternate proof (not sure if this is correct)

\begin{proof}
Suppose $x$ is an arbitrary element of $\pwset(A)$. Then $x \in A$. Since $A \subseteq \pwset(A)$ and $x \in A$ then $x \subseteq \pwset(A)$. Therefore, $x \in \pwset(\pwset(A))$.
\end{proof}

\section*{Exercise 3.3.5}
\n The hypothesis of the theorem proven in exercise 3.3.4 is $A \subseteq \pwset(A)$.
\subsection*{A}
\n Can you think of a set $A$ for which this hypothesis is true? \\

\n The empty set $\varnothing$ is a set for which the hypothesis is true. \\

\n $A \subseteq \pwset(A)$ means $x \in A \rightarrow x \in \pwset(A)$. For $\varnothing$ this would mean that $x \in \varnothing \rightarrow x \in \pwset(\varnothing)$, but by definition there are no elements in $\varnothing$. Therefore $x \in \varnothing$ will always be false and the conditional statement $x \in \varnothing \rightarrow x \in \pwset(\varnothing)$ is always true. Therefore if $\varnothing = A$ then $A \subseteq \pwset(A)$. 

\subsection*{B}
\n Can you think of another? \\

\n In exercise 3.3.4 we proved that if $A \subseteq \pwset(A)$ then $\pwset(A) \subseteq \pwset(\pwset(A))$. Therefore, the set $\{\varnothing, \{\varnothing\}\}$, which is the $\pwset(A)$ if $A = \varnothing$, is another set for which the hypothesis is true. If we let $B = \pwset(A) = \{\varnothing, \{\varnothing\}\}$ and replace $A$ in the hypothesis $A \subseteq \pwset(A)$ with $B$, then we can conclude that $B \subseteq \pwset(B)$.

\section*{Exercise 3.3.6}
Suppose $x$ is a real number.
\subsection*{A}
Prove that if $x \neq 1$ then there is a real number $y$ such that $\tfrac{y+1}{y-2} = x$. \\

So we want to prove that $(x \neq 1) \rightarrow \exists y \left( \tfrac{y+1}{y+2} = x \right)$ \\

We assume the antecedent and make the consequent our goal to prove. \\

\begin{table}[h]
\begin{tabular}{ll}
\hline
Givens & Goals   \\ \hline
$x \neq 1$ & $\exists y \left( \tfrac{y+1}{y+2} = x \right)$   \\ \hline
\end{tabular}
\end{table}

\n To prove our goal we need to find  a $y$ that makes the equation $ \tfrac{y+1}{y+2} = x $ true. So let's try solving the equation for $y$.

\begin{align*}
\frac{y+1}{y+2} &= x \\
y+1 &= x(y-2) \\
y+1 &= xy - 2x \\
2x+1 &= xy-y \\
2x+1 &= y(x-1) \\
y &= \frac{2x+1}{x-1}
\end{align*}

We see that this $y$ works because we have $x \neq 1$ as a given.

\begin{theorem} Suppose $x \neq 1$. Then there is a real number $y$ such that $\tfrac{y+1}{y-2} = x$.
\end{theorem}
\begin{proof}
Suppose $x \neq 1$ and $y = \tfrac{2x+1}{x-1}$. Then

\begin{equation*}
\frac{ \frac{2x+1}{x-1} +1 }{ \frac{2x+1}{x-1} -2} = \frac{ \frac{3x}{x-1} } { \frac{3}{x-1} } = \frac{3x}{x-1} \cdot \frac{x-1}{3} = x
\end{equation*}
\end{proof}

\subsection*{B}

\n Prove that if there is a real number $y$ such that $\tfrac{y+1}{y-2} = x$ then $x \neq 1$. \\

\n So we want to prove that
$\exists y \left(\frac{y+1}{y-1} = x \right) \rightarrow (x \neq 1)$ \\

\n We assume the antecedent and make the consequent our goal to prove. \\

\begin{table}[h]
\begin{tabular}{ll}
\hline
Givens & Goals   \\ \hline
$\exists y \left(\frac{y+1}{y-1} = x \right)$ & $x \neq 1$   \\ \hline
\end{tabular}
\end{table}


\n Using existential instantiation we assume there is a value $y_0$ such that $ \tfrac{y+1}{y-1} = x$ is true. From part A above, we know that $\left(\frac{y+1}{y-1} = x \right) \rightarrow \left( y = \tfrac{2x+1}{x-1} \right)$ and so $y_0 = \tfrac{2x+1}{x-1}$. Since $y$ is a real number, then clearly $x \neq 1$.

\begin{theorem} If $y$ is a real number and $\tfrac{y+1}{y-2} = x$ then $x \neq 1$.
\end{theorem}
\begin{proof}
Suppose $y$ is a real number and $\tfrac{y+1}{y-2} = x$. It follows that $y = \tfrac{2x+1}{x-1}$ and since $y$ is real number then $x \neq 1$.
\end{proof}

\section*{Exercise 3.3.7}
Prove for every real number $x$, if $x>2$ then there is a real number $y$ such that $y + \tfrac{1}{y} = x$. \\

\n So we want to prove
$\forall x \in \mathbb{R}(x>2 \rightarrow \exists y \in \mathbb{R}(y + \tfrac{1}{y} = x))$ \\

\n So we let $x$ be an arbitrary real number, then we assume the antecedent and make the consequent our goal to prove. \\

\begin{table}[h]
\begin{tabular}{ll}
\hline
Givens & Goals   \\ \hline
$x$ is arbitrary real number & $\exists y (y + \tfrac{1}{y} = x)$   \\ 
$x > 2$  & \\ \hline
\end{tabular}
\end{table}

\n Our goal is of the form $\exists y P(y)$ where $P(y)$ is $y + \tfrac{1}{y} = x$ and our strategy suggests we try to find a $y$ for which $P(y)$ is true. We can do this by solving the equation $y + \tfrac{1}{y} = x$ for $y$. We can rewrite this equation as $y^2 - \tfrac{x}{y} + 1 = 0$ and we see this is a quadratic equation and therefore we can use the quadratic formula to solve for $y$,



\begin{align*}
y = \frac{-(-x) \pm \sqrt{(-x)^2 - 4\cdot1\cdot1}}{2 \cdot 1} = \frac{x \pm \sqrt{x^2 - 4}}{2} \text{.}
\end{align*}

\n We note that $\sqrt{x^2 - 4}$ is defined because $x>2$. We have found two solutions that satisfy our original equation, but we only need one to complete the proof. We will use $\tfrac{x + \sqrt{x^2 - 4}}{2}$.

\begin{theorem} For every real number $x$, if $x>2$ then there is a real number $y$ such that $y + \tfrac{1}{y} = x$.
\end{theorem}
\begin{proof}
Suppose $x$ and $y$ are real numbers, $x>2$, and $y = \tfrac{x + \sqrt{x^2 - 4}}{2}$. Then

\begin{align*}
\frac{x + \sqrt{x^2 -4}}{2} + \frac{1}{ \frac{x + \sqrt(x^2 - 4}{2}} &= \frac{x + \sqrt{x^2 - 4}}{2} + \frac{2}{x + \sqrt{x^2 - 4}} \\
&= \frac{2x^2 + 2(x\sqrt{x^2 - 4})}{2x + 2\sqrt{x^2-4}} \\
&= x 
\end{align*}

\end{proof}

\section*{Exercise 3.3.8}
\n Prove that if $\F$ is a family of sets and $A \in \F$, then $A \subseteq \cup \F$. \\

\n So we want to prove that
$ A \in \F \rightarrow A \subseteq \cup \F$. \\

\n We assume the antecedent and make the consequent our goal to prove. \\

\begin{table}[h]
\begin{tabular}{ll}
\hline
Givens & Goals   \\ \hline
$ A \in \F$  & $A \subseteq \cup \F$   \\ \hline
\end{tabular}
\end{table}

\n Assume $A \in \F$ \\
\indent [ proof of $A \subseteq \cup \F$ ] \\
\n Therefore if $A \in \F$ then  $A \subseteq \cup \F$\\



\n Our goal $A \subseteq \cup \F$ can be rewritten as $\forall x (x \in A \rightarrow x \in \cup \F)$. We assume $x$ is arbitrary and then assume the antecedent and make the consequent our goal to prove. \\


\begin{table}[h]
\begin{tabular}{ll}
\hline
Givens & Goals   \\ \hline
$ A \in \F$  & $x \in \cup \F$   \\ 
$ x \in A$ & \\ \hline
\end{tabular}
\end{table}

\n Assume $A \in \F$ \\
\indent Assume $x$ is arbitrary \\
\indent \indent Assume $x \in A$ \\
\indent \indent \indent [ proof of $x \in \cup \F$ ] \\
\indent \indent Therefore if $x \in A$ then $x \cup \F$ \\
\indent Since $x$ was arbitrary we can conclude that $A \subseteq \cup \F$. \\
\n Therefore if $A \in \F$ then  $A \subseteq \cup \F$\\

\n Our new goal can be rewritten as $\exists B \in \F (x \in B)$. From our givens we see that $A \in \F$ and $x \in A$, so we have found a set such that $A \in \F(x \in A)$.

\begin{theorem} If $\F$ is a family of sets and $A \in \F$, then $A \subseteq \cup \F$.
\end{theorem}
\begin{proof}
Assume $A \in \F$ and $x$ is an arbitrary member of $A$. Then since $x \in A$ and $A \in \F$, it follows that $x \in \cup \F$. Since $x$ was arbitrary we can conclude that $A \subseteq \cup \F$ and therefore if $A \in \F$ then $A \subseteq \cup \F$.
\end{proof}

\section*{Exercise 3.3.9}
Prove that if $\F$ is a family of sets and $A \in \F$, then $\cap \F \subseteq A$. \\

\n We want to prove that
$A \in \F \rightarrow \cap \F \subseteq A$. \\

\n We assume the antecedent and make the consequent our goal to prove. \\

\begin{table}[h]
\begin{tabular}{ll}
\hline
Givens & Goals   \\ \hline
$ A \in \F$  & $\cap \F \subseteq A$   \\ \hline
\end{tabular}
\end{table}

\n Assume $A \in \F$ \\
\indent [proof of $\cap \F \subseteq A$] \\
\n Therefore, if $A \in \F$ then $\cap \F \subseteq A$. \\

\n We can rewrite our goal as $\forall x(x \in \cap \F \rightarrow x \in A$). We assume $x$ is arbitrary and then assume the antecedent and make the consequent our goal to prove. \\

\begin{table}[h]
\begin{tabular}{ll}
\hline
Givens & Goals   \\ \hline
$ A \in \F$  & $x \in A$   \\ 
$x \in \cap \F$ & \\ \hline
\end{tabular}
\end{table}

\n Assume $A \in \F$ \\
\indent Assume $x$ is arbitrary \\
\indent \indent Assume $x \in \cap \F$ \\
\indent \indent \indent [proof of $x \in A$] \\
\indent \indent Therefore, if $x \in \cap \F$ then $x \in A$. \\
\indent Since $x$ was arbitrary we can conclude that $\cap \F \subseteq A$. \\
\n Therefore, if $A \in \F$ then $\cap \F \subseteq A$. \\

\n Our given $x \in \cap \F$ can be rewritten as $\forall B \in \F (x \in B)$, therefore if $A \in \F$ then $x \in A$, which was our goal to prove.

\begin{theorem} If $\F$ is a family of sets and $A \in \F$, then $\cup \F \in A$.
\end{theorem}
\begin{proof}
Assume $A \in \F$ and $x$ is an arbitrary member of $\cap \F$. Since $A \in \F$ and $x \in \cap \F$ it follows that $x \in A$ and therefore, if $x \in \cap \F$ then $x \in A$. Since $x$ was arbitrary we can conclude that $\cap \F \subseteq A$. Therefore, if $A \in \F$ then $\cap \F \subseteq A$.
\end{proof}

\section*{Exercise 3.3.10}
Suppose that $\F$ is a nonempty family of sets $B$ is a set, and $\forall A \in \F(B \subseteq A)$. Prove that $B \subseteq \cap \F$. \\

\n We want to prove
$\forall A \in \F(B \subseteq A) \rightarrow B \subseteq \cap \F$. \\

\n We assume the antecedent and make the consequent our goal to prove. \\

\begin{table}[h]
\begin{tabular}{ll}
\hline
Givens & Goals   \\ \hline
$\forall A \in \F(B \subseteq A)$  & $B \subseteq \cap \F$   \\ \hline
\end{tabular}
\end{table}

\n Suppose $\forall A \in \F(B \subseteq A)$ \\
\indent [proof of $B \subseteq \cap \F$] \\
\n Therefore if $\forall A \in \F(B \subseteq A)$ then $B \subseteq \cap \F$. \\

\n Our goal can be rewritten as $\forall x(x \in B \rightarrow x \in \cap \F)$. So we assume $x$ is arbitrary and then assume the antecedent and make the consequent our goal to prove. \\

\begin{table}[h]
\begin{tabular}{ll}
\hline
Givens & Goals   \\ \hline
$\forall A \in \F(B \subseteq A)$  & $B \in \cap \F$   \\ 
$x \in B$ & \\ \hline
\end{tabular}
\end{table}

\n Suppose $\forall A \in \F(B \subseteq A)$ \\
\indent Suppose $x$ is arbitrary. \\
\indent \indent Suppose $x \in B$. \\
\indent \indent \indent [proof of $x \in \cap \F$] \\
\indent \indent Therefore $x \in B \rightarrow x \in \cap \F$ \\
\indent Since $x$ was arbitrary we can conclude $B \subseteq \cap \F$ \\
\n Therefore if $\forall A \in \F(B \subseteq A)$ then $B \subseteq \cap \F$. \\

\n Our goal can be rewritten as $\forall M \in F(x \in M)$ and so we can assume $M$ is an arbitrary set in $\F$ and make our goal $x \in M$.\\


\begin{table}[h]
\begin{tabular}{ll}
\hline
Givens & Goals   \\ \hline
$\forall A \in \F(B \subseteq A)$  & $x \in M$   \\ 
$x \in B$ & \\
$M \in \F$ & \\ \hline
\end{tabular}
\end{table}

\n Suppose $\forall A \in \F(B \subseteq A)$ \\
\indent Suppose $x$ is arbitrary. \\
\indent \indent Suppose $x \in B$. \\
\indent \indent \indent Suppose $M$ is an arbitrary set in $\F$. \\
\indent \indent \indent \indent [proof of $x \in M$] \\
\indent \indent \indent Therefore $x \in \cap \F$ \\
\indent \indent Therefore $x \in B \rightarrow x \in \cap \F$ \\
\indent Since $x$ was arbitrary we can conclude $B \subseteq \cap \F$ \\
\n Therefore if $\forall A \in \F(B \subseteq A)$ then $B \subseteq \cap \F$. \\

\n Using universal instantiation we will plug in $M$ for $A$ in our given $\forall A \in \F(B \subseteq A)$ and conclude that $B \subseteq M$. We can rewrite $B \subseteq M$ as $\forall y(y \in B \rightarrow y \in M)$ and using universal instantiation plug in $x$ for $y$ and then use moden ponens to conclude $x \in M$, which was our goal to prove.

\begin{theorem} If $\F$ is a nonempty family of sets, $B$ is a set, and $\forall A \in \F(B \subseteq A)$, then $B \subseteq \cap \F$.
\end{theorem}
\begin{proof}
Suppose $\forall A \in \F(B \subseteq A)$. Suppose $x$ is an arbitrary member of $B$ and $M$ is an arbitrary set in $\F$. Then it follows that $x \in M$ and since $M$ was arbitrary we can conclude that $x$ is in all sets that are in $\F$ or $x \in \cap \F$. Therefore, if $x \in B$ then $x \in \cap \F$, and since $x$ was arbitrary, we can conclude that $B \subseteq \cap \F$. 
\end{proof}


\section*{Exercise 3.3.11}

Suppose that $\F$ is a family of sets. Prove that if $\varnothing \in \F$ then $\cap \F = \varnothing$.

\n We want to prove that
$\varnothing \in \F \rightarrow \cap \F = \varnothing$ \\

\n We assume the antecedent and make the consequent our goal to prove. \\

\begin{table}[h]
\begin{tabular}{ll}
\hline
Givens & Goals   \\ \hline
$\varnothing \in \F$  & $\cap \F = \varnothing$   \\ \hline
\end{tabular}
\end{table}

\n Suppose $\varnothing \in \F$ \\
\indent [proof of $\cap \F = \varnothing$] \\
\n Therefore if $\varnothing \in \F$ then $\cap \F = \varnothing$. \\

\n We will try a proof by contradiction. So we assume that $\cap \F \neq \varnothing$ and try to find a contradiction.

\begin{table}[h]
\begin{tabular}{ll}
\hline
Givens & Goals   \\ \hline
$\varnothing \in \F$  &  contradiction \\
$\cap \F \neq \varnothing$  & \\ \hline
\end{tabular}
\end{table}

\n Our given $\cap \F \neq \varnothing$ means that there is an element that is in all sets in $\F$. However, this contradicts $\varnothing \in \F$ because $\varnothing$ is the set that contains nothing.

\begin{theorem} If $\F$ is a family of sets and $\varnothing \in \F$, then $\cap \F = \varnothing$.
\end{theorem}
\begin{proof}
We will prove by contradiction. Suppose $\varnothing \in \F$ and $\cap \F \neq \varnothing$. Since $\cap \F \neq \varnothing$ it follows that there is an element that is within all of the sets that are in $\F$. However, this contradicts $\varnothing \in \F$ because $\varnothing$ is the set that contains nothing. Therefore, if $\varnothing \in \F$ then $\cap \F = \varnothing$.
\end{proof}


\section*{Exercise 3.3.12}
\n Suppose $\F$ and $\G$ are families of sets. Prove that if $\F \subseteq \G$ then 
$\cup \F \subseteq \cup \G$. \\

\n So we want to prove that 
$\F \subseteq \G \rightarrow \cup \F \subseteq \cup \G$ \\

\n First we assume the antecedent and make the consequent our goal to prove.

\begin{table}[h]
\begin{tabular}{ll}
\hline
Givens & Goals   \\ \hline
$\F \subseteq \G$ & $\cup \F \subseteq \cup \G$   \\ \hline

\end{tabular}
\end{table}

\n Suppose $\F \subseteq \G$ \\
\indent [proof of $\cup \F \subseteq \cup \G$ ] \\
\n So if $\F \subseteq \G \rightarrow \cup \F \subseteq \cup \G$ \\


\n $\cup \F \subseteq \cup \G \rightarrow \forall b (b \in \cup \F \rightarrow b \in \cup \G)$ so we assume $b$ is an arbitrary element of $\cup \F$ and assume the antecedent and make the consequent our goal to prove. \\

\begin{table}[h]
\begin{tabular}{ll}
\hline
Givens & Goals   \\ \hline
$\F \subseteq \G$ & $b \in \cup \G$   \\
$b \in \cup \F$ & \\ \hline
\end{tabular}
\end{table}

\n Suppose $\F \subseteq \G$ \\
\indent Let $b$ be an arbitrary element of $\cup \F$  \\
\indent \indent [proof of $b \in \cup \G$ ] \\
\indent Therefore if $b \in \cup \F \rightarrow b \in \cup \G$ \\
\n Since $b$ was arbitrary we can conclude $\forall b (b \in \cup \F \rightarrow b \in \cup \G)$. So if $\F \subseteq \G \rightarrow \cup \F \subseteq \cup \G$ \\

\n $b \in \cup \F \rightarrow \exists M (M \in \F \wedge b \in M)$, so let $M = A_{0}$ (Existential Instantiation)


\begin{table}[h]
\begin{tabular}{ll}
\hline
Givens & Goals   \\ \hline
$\F \subseteq \G$ & $b \in \cup \G$   \\
$A_{0} \in \F \wedge b \in A_{0}$ & \\ \hline
\end{tabular}
\end{table}

\n Suppose $\F \subseteq \G$
\begin{addmargin}[0.55cm]{0cm}
\n Let $b$ be an arbitrary element and suppose $b \in \cup \F$, which implies there is a set in $\F$ and $b$ is in that set. Let that set = $A_{0}$
\end{addmargin}

\indent \indent [proof of $b \in \cup \G$ ] \\
\indent Therefore if $b \in \cup \F \rightarrow b \in \cup \G$ \\
\n Since $b$ was arbitrary we can conclude $\forall b (b \in \cup \F \rightarrow b \in \cup \G)$. So if $\F \subseteq \G \rightarrow \cup \F \subseteq \cup \G$ \\

\n $\F \subseteq \G \rightarrow \forall A(A \in \F \rightarrow A \in \G)$. Using universal instantiation we will plug in $A_{0}$ for $A$ since then we can use modens ponens to conclude that $A_{0} \in \G$.

\begin{table}[h]
\begin{tabular}{ll}
\hline
Givens & Goals   \\ \hline
$A_{0} \in \F \rightarrow A_{0} \in \G$ & $b \in \cup \G$  \\
$A_{0} \in \F \wedge b \in A_{0}$ & \\ \hline
\end{tabular}
\end{table}

\n Our goal $b \in \cup \G \rightarrow \exists N (N \in \G \wedge b \in N)$, which we can now prove. Since $A_{0} \in \F$ and $\F$ is a subset of $\G$, it follows that $A_{0} \in G$. By the definition of $\cup \G$ it follows that $b \in \cup \G$ because $A_{0} \in G \wedge b \in A_{0}$, the latter statement being one of our givens. \\


\n \textbf{Theorem.} \textit{Suppose $\F$ and $\G$ are families of sets. If $\F \subseteq \G$ then 
$\cup \F \subseteq \cup \G$.}

\n \textit{Proof.} Suppose $\F \subseteq \G$. Let $b$ be an arbitrary element of $\cup \F$, which implies there is a set in $\mathcal{F}$ that contains $b$. Call this set $A_{0}$. Since $A_{0} \in \F$ and $\F$ is a subset of $\G$ it follows that $A_{0} \in G$, which implies that $b \in \cup \G$. Therefore if $b \in \cup \F$ then $b \in \cup \G$. Since $b$ was arbitrary we can conclude that if $\F \subseteq \G$ then $\cup \F \subseteq \cup \G$. This completes the proof.


\section*{Exercise 3.3.13}
\n Suppose $\F$ and $\G$ are families of sets. Prove that if $\F \subseteq \G$ then 
$\cap \G \subseteq \cap \F$. \\

\n So we want to prove that 
$\F \subseteq \G \rightarrow \cap \G \subseteq \cap \F$ \\

\n First we assume the antecedent and make the consequent our goal to prove.

\begin{table}[h]
\begin{tabular}{ll}
\hline
Givens & Goals   \\ \hline
$\F \subseteq \G$ & $\cap \G \subseteq \cap \F$   \\ \hline
\end{tabular}
\end{table}

\n Suppose $\F \subseteq \G$ \\
\indent [proof of $\cap \G \subseteq \cap \F$ ] \\
\n So if $\F \subseteq \G \rightarrow \cap \G \subseteq \cap \F$ \\

\n $\cap \G \subseteq \cap \F \rightarrow \forall b ( b \in \cap \G \rightarrow b \in \cap \F)$, so we assume $b$ is an arbitrary element of $\cap \G$ and assume the antecedent and make the consequent our goal to prove. \\

\begin{table}[h]
\begin{tabular}{ll}
\hline
Givens & Goals   \\ \hline
$\F \subseteq \G$ & $b \in \cap \F$   \\
$b \in \cap \G$ & \\ \hline
\end{tabular}
\end{table}

\n Suppose $\F \subseteq \G$ \\
\indent Let $b$ be an arbitrary element of $\cap \G$  \\
\indent \indent [proof of $b \in \cap \F$ ] \\
\indent Therefore if $b \in \cap \G \rightarrow b \in \cap \F$ \\
\n Since $b$ was arbitrary we can conclude $\forall b (b \in \cap \G \rightarrow b \in \cap \F)$. So $\F \subseteq \G \rightarrow \cap \G \subseteq \cap \F$ \\


\n $b \in \cap \F \rightarrow \forall A (A \in \F \rightarrow b \in A)$, so we assume $A$ is an arbitrary element of $\F$ and assume the antecedent and make the consequent our goal to prove. \\

\begin{table}[h]
\begin{tabular}{ll}
\hline
Givens & Goals   \\ \hline
$\F \subseteq \G$ & $b \in A$   \\
$b \in \cap \G$ & \\
$A \in \F$ & \\ \hline
\end{tabular}
\end{table}

\n Suppose $\F \subseteq \G$

\begin{addmargin}[0.55cm]{0cm}
\indent Let $b$ be an arbitrary element of $\cap \G$
\end{addmargin}

\indent \indent Suppose $A$ is an arbitrary set in $\F$ \\
\indent \indent \indent [proof of $b \in A$ ] \\
\indent \indent Therefore if $A \in \F \rightarrow b \in A $ \\
\indent \indent Since A was arbitrary we can conclude $b \in \cap \F$ \\
\indent Therefore if $b \in \cap \G \rightarrow b \in \cap \F$ \\
\n Since $b$ was arbitrary we can conclude $\forall b (b \in \cap \G \rightarrow b \in \cap \F)$. So $\F \subseteq \G \rightarrow \cap \G \subseteq \cap \F$ \\


\n Now looking at our givens, $\F \subseteq \G \rightarrow \forall Z ( Z \in \F \rightarrow Z \in \G)$. Using universal instantiation we will plug in $A$ for $Z$ and using modus ponens we can conclude that $A \in \G$. \\

\n Our other given, $b \in \cap \G \rightarrow \forall Y (Y \in \G \rightarrow b \in Y)$. Using universal instantiation we will plug in $A$ for $Y$ and using modus ponens we can conclude that $b \in A$, which was our goal, and we can now write our proof. \\


\n \textbf{Theorem.} \textit{Suppose $\F$ and $\G$ are families of sets. If $\F \subseteq \G$ then 
$\cap \G \subseteq \cap \F$.}
\n \textit{Proof.} Suppose $\F \subseteq \G$. Let $b$ be an arbitrary element of $\cap \G$. Suppose $A$ is an arbitrary element of $\F$, then because $\F \subseteq \G$ then it follows that $A \in \G$. By the definition of $\cap \G$ it follows that $b \in A$ and since $A$ was arbitrary then $b \in \cap \F$. Since $b$ was arbitrary we can conclude $\cap \G \subseteq \cap \F$ and therefore that if $\F \subseteq \G$ then $\cap \G \subseteq \cap \F$. This completes the proof.



\section*{Exercise 3.3.14}
% Potential answer I came across, haven't looked at it yet.
% https://math.stackexchange.com/questions/220572/the-union-of-powersets-is-contained-in-the-powerset-of-union

Suppose $\{ A_{i} | i \in I \} $ is an indexed family of sets. Prove that $\bigcup_{i \in I} \pwset(A_{i}) \subseteq \pwset (\bigcup_{i \in I} A_{i})$. \\

\n So we want to prove that $\forall a (a \in \bigcup_{i \in I} \pwset(A_{i}) \rightarrow a \in \pwset(\bigcup_{i \in I} A_i))$ \\

\n First we assume $a$ is arbitrary and make the antecedent a given and the consequent our goal to prove. \\

\begin{table}[h]
\begin{tabular}{ll}
\hline
Givens & Goals   \\ \hline
$a \in \bigcup_{i \in I} \pwset(A_{i})$ & $a \in \pwset(\bigcup_{i \in I} A_i)$   \\ \hline
\end{tabular}
\end{table}


\n Assume $a$ is an arbitrary element of $\bigcup_{i \in I} \pwset(A_{i})$ \\
\indent Suppose $a \in \bigcup_{i \in I} \pwset(A_{i})$ \\
\indent \indent [ proof of $a \in \pwset(\bigcup_{i \in I} A_i)$ ] \\
\indent Therefore if $a \in \bigcup_{i \in I} \pwset(A_{i}) \rightarrow a \in \pwset(\bigcup_{i \in I} A_i)$ \\
\n Since $a$ was arbitrary we can conclude $\bigcup_{i \in I} \pwset(A_{i}) \subseteq \pwset (\bigcup_{i \in I} A_{i})$ \\

\n Looking at our goal we see that $a \in \pwset(\bigcup_{i \in I} A_i) \rightarrow a \subseteq \bigcup_{i \in I} A_i \rightarrow \forall z (z \in a \rightarrow z \in \bigcup_{i \in I} A_i ) $. Therefore we assume $z$ is arbitrary, assume the antecedent, and make the consequent our goal to prove. \\

\begin{table}[h]
\begin{tabular}{ll}
\hline
Givens & Goals   \\ \hline
$a \in \bigcup_{i \in I} \pwset(A_{i})$ & $z \in \bigcup_{i \in I} A_i $   \\
$z \in a$ & \\ \hline

\end{tabular}
\end{table}

\n Assume $a$ is an arbitrary element of $\bigcup_{i \in I} \pwset(A_{i})$ \\
\indent Suppose $a \in \bigcup_{i \in I} \pwset(A_{i})$ \\
\indent \indent Assume z is arbitrary \\
\indent \indent \indent Assume $z \in a$ \\
\indent \indent \indent \indent [ proof of $z \in \bigcup_{i \in I} A_i $ ] \\
\indent \indent \indent Therefore $z \in a \rightarrow z \in \bigcup_{i \in I} A_i $ \\
\indent \indent Since z was arbitrary we can conclude $a \in \pwset(\bigcup_{i \in I} A_i)$ \\
\indent Therefore if $a \in \bigcup_{i \in I} \pwset(A_{i}) \rightarrow a \in \pwset(\bigcup_{i \in I} A_i)$ \\
\n Since $a$ was arbitrary we can conclude $\bigcup_{i \in I} \pwset(A_{i}) \subseteq \pwset (\bigcup_{i \in I} A_{i})$ \\

\n Looking at our given we see that $a \in \bigcup_{i \in I} \pwset(A_{i}) \rightarrow a \in \{ a | \exists i \in I (a \in \pwset(A_i)) \}$. Using existential instantiation we will select an $i$ such that $a \in \pwset(A_i)$ which implies $a \subseteq A_i$. Since $a \subseteq A_i \rightarrow \forall m(m \in a \rightarrow m \in A_i)$ and using universal instantiation we will plug in $z$ for $m$ and we get $\forall z(z \in a \rightarrow z \in A_i)$ and using modus ponens we can conclude that $z \in A_i$, which implies that $z \in \bigcup_{i \in I} A_i $, which was our goal. We can now right our proof. \\


\n \textbf{Theorem.} \textit{Suppose $\{ A_{i} | i \in I \} $ is an indexed family of sets, then $\bigcup_{i \in I} \pwset(A_{i}) \subseteq \pwset (\bigcup_{i \in I} A_{i})$.} \\
\n \textit{Proof.} Suppose that $a$ is an arbitrary element of $\bigcup_{i \in I} \pwset(A_{i})$. We choose an $i \in I$ such that $a \in \pwset(A_i)$, which implies that $a \subseteq A_i$. Suppose $z$ is an arbitrary element of $a$, then it follows that $z \in A_i$ and therefore $z \in \bigcup_{i \in I} A_i$. Since $z$ was an arbitrary element of $a$ then $a \subseteq \bigcup_{i \in I} A_i$, and it follows that $a \in \pwset(\bigcup_{i \in I} A_i)$. Thus we can conclude $\bigcup_{i \in I} \pwset(A_{i}) \subseteq \pwset (\bigcup_{i \in I} A_{i})$. This completes the proof.


\section*{Exercise 3.3.15}
Suppose $ \{ A_i | i \in I \}$ is an indexed family of sets and $ I \neq \varnothing$. Prove that $\bigcap_{i \in I} A_i \in \bigcap_{i \in I} \pwset(A_i)$ \\

\n So we want to prove that $\forall y (y \in \bigcap_{i \in I} A_i \rightarrow y \in \bigcap_{i \in I} \pwset(A_i))$. \\

\n First we assume $y$ is arbitrary and make the antecedent a given and the consequent our goal to prove. \\

\begin{table}[h]
\begin{tabular}{ll}
\hline
Givens & Goals   \\ \hline
$y \in \bigcap_{i \in I} A_i$ & $y \in \bigcap_{i \in I} \pwset(A_i)$   \\ \hline
\end{tabular}
\end{table}

\n Suppose $y$ is arbitrary element of $\bigcap_{i \in I} A_i$. \\
\indent [proof of $y \in \bigcap_{i \in I} \pwset(A_i)$] \\
\n Since y was arbitrary we can conclude $\forall y (y \in \bigcap_{i \in I} A_i \rightarrow y \in \bigcap_{i \in I} \pwset(A_i))$. \\

\n Our goal $y \in \bigcap_{i \in I} \pwset(A_i)$ so we make $m$ an arbitrary element of $I$ and therefore $y \in \pwset(A_m) \rightarrow y \subseteq A_m \rightarrow \forall z (z \in y \rightarrow z \in A_m)$. So we make $z$ arbitrary and make the antecedent a given and the consequent our goal to prove. \\


\begin{table}[h]
\begin{tabular}{ll}
\hline
Givens & Goals   \\ \hline
$y \in \bigcap_{i \in I} A_i$ & $z \in A_m$   \\
$z \in y$ & \\ \hline
\end{tabular}
\end{table}

\n Suppose $y$ is arbitrary element of $\bigcap_{i \in I} A_i$.

\begin{addmargin}[0.55cm]{0cm}
Suppose $m$ is an arbitrary element of $I$ and therefore $y \in \pwset(A_m) \rightarrow y \subseteq A_m \rightarrow \forall z (z \in y \rightarrow z \in A_m)$.
\end{addmargin}

\indent \indent Suppose $z$ is an arbitrary element of $y$ \\
\indent \indent \indent [proof of $z \in A_m$] 
\begin{addmargin}[0.55cm]{0cm}
Therefore $z \in y \rightarrow z \in A_m$ and since $z$ was arbitrary $y \subseteq A_m \rightarrow y \in \pwset(A_m)$ and since $m$ was arbitrary $y \in \bigcap_{i \in I} \pwset(A_i)$
\end{addmargin}
\n Since y was arbitrary we can conclude $\forall y (y \in \bigcap_{i \in I} A_i \rightarrow y \in \bigcap_{i \in I} \pwset(A_i))$. \\

\n Now looking at our given $y \in \bigcap_{i \in I} A_i \rightarrow \forall i \in I(y \in A_i)$. Using universal instantiation we plug in $m$ for $i$ and therefore $y \in A_m$ and since $z \in y$ we can conclude $z \in A_m$, which was our goal. Now we can write our proof. \\

\n \textbf{Theorem.} \textit{Suppose $ \{ A_i | i \in I \}$ is an indexed family of sets and $ I \neq \varnothing$, then $\bigcap_{i \in I} A_i \in \bigcap_{i \in I} \pwset(A_i)$.} \\
\n \textit{Proof.} Suppose $y$ is an arbitrary element of $\bigcap_{i \in I} A_i$. Suppose $m$ is an arbitrary member of $I$ and therefore $y \subseteq A_m$ which implies $y \subseteq A_m$. Now suppose $z$ is an arbitrary element of $y$. Since $y \in \bigcap_{i \in I} A_i$ if we choose an $i$ such that $y \in \bigcap_{m \in I} A_m$ then $y \in A_m$ which implies $z \in A_m$. Therefore if $z \in y$ then $z \in A_m$ and since $z$ was arbitrary then $y \subseteq A_m$ or $y \in \pwset(A_m)$ and since $m$ was arbitrary then $y \in \bigcap_{i \in I} \pwset(A_i)$. Since $y$ was arbitrary then $\bigcap_{i \in I} A_i \in \bigcap_{i \in I} \pwset(A_i)$. This completes the proof.

\section*{Exercise 3.3.16}
Prove the converse of the statement proven in Example 3.3.5. In other words, prove that if $\F \subseteq \pwset(B)$ then $\cup \F \subseteq B$. \\

\n We want to prove
$\F \subseteq \pwset(B) \rightarrow \cup \F \subseteq B$. \\

\n We assume the antecedent and make the consequent our goal to prove. \\

\begin{table}[h]
\begin{tabular}{ll}
\hline
Givens & Goals   \\ \hline
$\F \subseteq \pwset(B)$ & $\cup \F \subseteq B$   \\ \hline
\end{tabular}
\end{table}

\n Suppose $\F \subseteq \pwset(B)$ \\
\indent [proof of $\cup \F \subseteq B$ \\
\n Therefore if $\F \subseteq \pwset(B)$ then $\cup F \subseteq B$. \\

\n $\cup F \subseteq B \rightarrow \forall x (x \in \cup \F \rightarrow x \in B)$. So we assume $x$ is arbitrary and then assume the antecedent and make the consequent our goal to prove. \\

\begin{table}[h]
\begin{tabular}{ll}
\hline
Givens & Goals   \\ \hline
$\F \subseteq \pwset(B)$ & $x \in B$   \\ 
$x \in \cup \F$ & \\ \hline
\end{tabular}
\end{table}

\n Suppose $\F \subseteq \pwset(B)$ \\
\indent Suppose $x$ is arbitrary \\
\indent \indent Suppose $x \in \cup \F$ \\
\indent \indent \indent [proof of $x \in B$] \\
\indent \indent Therefore if $x \in \cup \F$ then $x \in B$ \\
\indent Since $x$ was arbitrary we can conclude that $\cup \F \subseteq B$. \\
\n Therefore if $\F \subseteq \pwset(B)$ then $\cup F \subseteq B$. \\

\n $x \in \cup \F \rightarrow \exists M \in \F(x \in M)$. We use existential instantiation and assume there is a set $M$ in $\F$ and $x$ is in that set. \\

\n Suppose $\F \subseteq \pwset(B)$ \\
\indent Suppose $x$ is arbitrary \\
\indent \indent Suppose $M$ is arbitrary set in $\F$ \\
\indent \indent \indent $x \in M$ \\
\indent \indent \indent \indent [proof of $x \in B$] \\
\indent \indent \indent Since $x \in M$ and $M$ is a set in $\F$ then $x \in \cup \F$ \\
\indent \indent Therefore if $x \in \cup \F$ then $x \in B$ \\
\indent Since $x$ was arbitrary we can conclude that $\cup \F \subseteq B$. \\
\n Therefore if $\F \subseteq \pwset(B)$ then $\cup F \subseteq B$. \\


\n Our given $\F \subseteq \pwset(B)$ means that $\forall N(N \in \F \rightarrow \forall z(z \in N \rightarrow z \in B)$. We will use universal instantiation and plug in $M$ for $N$ and $x$ for $z$ and we can conclude that $x \in B$, which was our goal to prove.

\begin{theorem} Suppose $B$ is a set and $\F$ is a family of sets. If $\F \subseteq \pwset(B)$ then $\cup \F \subseteq B$.
\end{theorem}
\begin{proof}
Suppose $x$ is an arbitrary member of $\cup \F$, which means that $x$ is a member of a set that is in $\F$. Suppose $\F \subseteq \pwset(B)$, which means that any element that is in a set that is a member of $\F$ is also in the set $B$. It follows that since $x$ is a member of a set in $\F$ then $x \in B$. Therefore, if $x \in \cup \F$ then $x \in B$ and since $x$ was arbitrary we can conclude $\cup \F \subseteq B$. Therefore, if $\F \subseteq \pwset(B)$ then $\cup \F \subseteq B$.
\end{proof}

\section*{Exercise 3.3.17}
\n Suppose $\F$ and $\G$ are nonempty families of sets, and every element of $\F$ is a subset of every element of $\G$. Prove that $\cup \F \subseteq \cap \G$.\\

\n We want to prove that
$\cup \F \subseteq \cap \G$. \\

\n We can rewrite this goal as $\forall x(x \in \cup \F \rightarrow x \in \cap \G)$. We assume $x$ is arbitrary and then assume the antecedent and make the consequent our goal. \\

\begin{table}[h]
\begin{tabular}{ll}
\hline
Givens & Goals   \\ \hline
$x \in \cup \F$ & $x \in \cap \G$   \\ 
$\forall A \in \F \forall B \in \G(y \in A \rightarrow y \in B)$ & \\ \hline
\end{tabular}
\end{table}

\n Suppose $x$ is arbitrary. \\
\indent Suppose $x \in \cup \F$.\\
\indent \indent [proof of $x \in \cap \G$] \\
\indent Therefore if $x \in \cup \F$ then $x \in \cap \G$ \\
\n Since $x$ was arbitrary we can conclude that $\cup \F \subseteq \cap \G$. \\

\n We can rewrite our goal as $\forall M \in \G(x \in M)$. We assume $M$ is an arbitrary set in $\G$ and then our goal becomes $x \in M$.


\begin{table}[h]
\begin{tabular}{ll}
\hline
Givens & Goals   \\ \hline
$x \in \cup \F$ & $x \in M$   \\ 
$M \in \G$ & \\
$\forall A \in \F \forall B \in \G(y \in A \rightarrow y \in B)$ & \\ \hline
\end{tabular}
\end{table}

\n Suppose $x$ is arbitrary. \\
\indent Suppose $x \in \cup \F$.\\
\indent \indent Suppose $M$ is an arbitrary set in $\G$ \\
\indent \indent \indent [proof of $x \in M$] \\
\indent \indent Since $M$ was arbitrary we can conclude that $x \in \cap \G$ \\
\indent Therefore if $x \in \cup \F$ then $x \in \cap \G$ \\
\n Since $x$ was arbitrary we can conclude that $\cup \F \subseteq \cap \G$. \\

\n We can rewrite our given $x \in \cup \F$ as $\exists N \in \F(x \in N)$. We use existential instantiation and assume there is a set $N \in \F$ and $x \in N$.

\begin{table}[h]
\begin{tabular}{ll}
\hline
Givens & Goals   \\ \hline
$N \in \F$ & $x \in M$   \\
$x \in N$ & \\ 
$M \in \G$ & \\
$\forall A \in \F \forall B \in \G(y \in A \rightarrow y \in B)$ & \\ \hline
\end{tabular}
\end{table}

\n Now we can use universal instantiation to plug in $N$ for $A$ and $M$ for $B$. Then since $x \in N$ we can use modus ponens to conclude that $x \in M$, which was our goal.

\begin{theorem} If $\F$ and $\G$ are nonempty families of sets, and every element of $\F$ is a subset of every element of $\G$, then $\cup \F \subseteq \cap \G$.
\end{theorem}
\begin{proof}
Suppose $x$ is an arbitrary member of $\cup \F$, which means there is a set in $\F$ that contains $x$. Suppose $M$ is an arbitrary set in $\G$. Then since every set in $\F$ is a subset of every set in $\G$ it follows that $x \in M$. Since $M$ was arbitrary we can conclude that $x \in \cap \G$ and therefore if $x \in \cup F$ then $x \in \cap \G$. Since $x$ was arbitrary we can conclude that $\cup \F \subseteq \cap \G$.
\end{proof}

\section*{Exercise 3.3.18}
In this problem all variables range over $\mathbb{Z}$, the set of all integers.

\subsection*{A}
Prove that if $a | b$ and $a | c$, then $a | (b+c)$. \\

\n We want to prove
$(a|b) \land (a|c) \rightarrow a|(b+c)$ \\

\n We assume the antecedent and make the consequent our goal. \\

\begin{table}[h]
\begin{tabular}{ll}
\hline
Givens & Goals   \\ \hline
$a|b$ & $a|(b+c)$   \\ 
$a|c$ & \\ \hline
\end{tabular}
\end{table}

\n Suppose $a|b$ and $a|c$ \\
\indent [proof of $a| (b+c)$ \\
\n Therefore if $a|b$ and $a|c$ then $a| (b+c)$. \\

\n Our goal means that $\exists x \in \mathbb{Z}(ax = (b+c))$. So we need to find an $x$ that makes this statement true. Our goals can be rewritten as $\exists y \in \mathbb{Z}(ay = b)$ and $\exists w \in \mathbb{Z}(aw = c)$. Using existential instantiation we will assume there is a $y$ and and $w$ that makes both of the previous statement true.

\begin{table}[h]
\begin{tabular}{ll}
\hline
Givens & Goals   \\ \hline
$ay = b$ & $a|(b+c)$   \\ 
$aw = c$ & \\ \hline
\end{tabular}
\end{table}

\n Suppose $ay = b$ and $aw = c$ \\
\indent [proof of $a| (b+c)$ \\
\n Therefore if $a|b$ and $a|c$ then $a| (b+c)$. \\

\n Adding the two inequalities $ay = b$ and $aw = c$ we have $ay + aw = b + c$ or $a(y+w) = b+c$. Since $y$ and $w$ are integers we can conclude that $a|(b+c)$, which was our goal to prove.

\begin{theorem} If $a$, $b$, and $c$ are integers, $a|b$, and $a|c$, then $a|(b+c)$.
\end{theorem}
\begin{proof}
Suppose $a$, $b$, and $c$ are integers, $a|b$, and $a|c$. Since $a|b$ there must be an integer $y$ such that $ay = b$. Also, since $a|c$ there must be an integer $w$ such that $aw = c$. Adding together the previous two equalities we have $ay + aw = b + c$ or $a(y+w) = b+c$. Since $y$ and $w$ are integers we can conclude that $a|(b+c)$.
\end{proof}

\subsection*{B}

\n Prove that if $ac|bc$ and $c \neq 0$, then $a|b$. \\

\n We want to prove
$(ac|bc) \land (c \neq 0) \rightarrow a|b$. \\ 

\n We assume the antecedent and make the consequent our goal. \\

\begin{table}[h]
\begin{tabular}{ll}
\hline
Givens & Goals   \\ \hline
$ac|bc$ & $a|b$   \\ 
$c \neq 0$ & \\ \hline
\end{tabular}
\end{table}

\n Suppose $ac|bc$ and $c \neq 0$\\
\indent [proof of $a|b$] \\
\n Therefore $ac|bc$ and $c \neq 0$, then $a|b$. \\

\n Our goal means that $\exists x(ax = b)$ and we want to find an $x$ that makes this statement true. Looking at our goals we can rewrite $ac|bc$ as $\exists y (acy = bc)$. Using existential instantiation we will assume there is a $y$ that makes $acy = bc$ true and we can add $acy = bc$ to our givens. Since $c \neq 0$ we can divide both sides of $acy = bc$ by $c$ and we have $ay = b$. Since $y$ is an integer we can conclude that $a|b$, which was our goal to prove.

\begin{theorem} If $a$, $b$, and $c$ are integers, $ac|bc$, and $c \neq 0$, then $a|b$.
\end{theorem}
\begin{proof}
Suppose $a$, $b$, and $c$ are integers, $ac|bc$, and $c \neq 0$. Since $ac|bc$ there must be an integer $x$ such that $acx = bc$. Since $c \neq 0$ we can simplify the previous equation by dividing both sides by $c$ so that $ax = b$. Since $x$ is an integer we can conclude that $a|b$.
\end{proof}

\section*{Exercise 3.3.19}
\subsection*{A}
Prove that for all real numbers $x$ and $y$ there is a real number $z$ such that $x + z = y - z$. \\

\n We want to prove that $\forall x \in \mathbb{R} \forall y \in \mathbb{R} \exists z \in \mathbb{R}(x + z = y - z)$. \\

\n We let $x$ and $y$ stand for arbitrary real numbers and make $\exists z \in \mathbb{R}(x + z = y - z)$ our goal to prove.

\begin{table}[h]
\begin{tabular}{ll}
\hline
Givens & Goals   \\ \hline
$x$ arbitrary & $\exists z \in \mathbb{R}(x + z = y - z)$   \\ 
$y$ arbitrary & \\ \hline
\end{tabular}
\end{table}

\n Suppose $x$ and $y$ are arbitrary real numbers \\
\indent [proof of $\exists z \in \mathbb{R}(x + z = y - z)$ ] \\
\n Since $x$ and $y$ are arbitrary we can conclude $\forall x \in \mathbb{R} \forall y \in \mathbb{R} \exists z \in \mathbb{R}(x + z = y - z)$ \\

\n We want to find a $z$ such that $x + z = y - z$, which suggests we try solving this equation for $z$

\begin{align*}
x + z &= y - z \\
x + 2z &= y \\
z &= \frac{y - x}{2}.
\end{align*}

\n Now we are ready to complete our proof.

\begin{theorem} For all real numbers $x$ and $y$ there is a real number $z$ such that $x + z = y - z$.
\end{theorem}
\begin{proof}
Suppose $x$ and $y$ are arbitrary real numbers and $z = \tfrac{y - x}{2}$. Then

\begin{align*}
x + \frac{y - x}{2} &= y - \frac{y - x}{2} \\
\frac{2x + y - x}{2} &= \frac{2y - (y - x}{2} \\
2x + y - x &= 2y - y + x \\
x(2-1) + y &= y(2-1) + x\\
x + y &= x + y
\end{align*}
\end{proof}

\subsection*{B}
\n Would the statement in part (A) be correct if "real number" were changed to "integer"? Justify your answer. \\

\n No, because there are instances where $z = \tfrac{x - y}{2}$ would not result in an integer. For example, if $x = 5$ and $y = 2$ then $z = \tfrac{3}{2}$, which is not an integer. Therefore the statement in part (A) would not be correct.

\section*{Exercise 3.3.20}
\n Consider the following theorem: \\

\begin{theorem} For every real number $x$, $x^2 \geq 0$.
\end{theorem}

\n What's wrong with the following proof?

\begin{proof}
Suppose not. Then for every real number $x$, $x^2 < 0$. In particular,
plugging in $x = 3$ we would get $9 < 0$, which is clearly false. This
contradiction shows that for every number $x, x^2 \geq 0$.
\end{proof}

\n The sentence "Then for every real number $x$, $x^2 < 0$" is not correct because if we let $x = 0$ then $0 < 0$ is not true.

\section*{3.3.21}
\n Consider the following incorrect theorem: \\

\n \textbf{Incorrect Theorem.} \textit{If $\forall x \in A (x \neq 0)$ and $A \subseteq B$ then $\forall x \in B (x \neq 0)$.}

\subsection*{A}
\n What's wrong with the following proof? \\

\begin{proof}
Let $x$ be an arbitrary element of $A$. Since $\forall x \in A(x \neq 0)$, we can conclude that $x \neq 0$. Also, since $A \subseteq B$, $x \in B$. Since $x \in B$, $x \neq 0$, and $x$ was arbitrary, we can conclude that $\forall x \in B(x \neq 0)$.
\end{proof}

\n The last sentence is not correct. $A \subseteq B$ means that all elements in $A$ are in $B$ and since $x \neq 0$ then $0 \notin A$, but this doesn't mean that $0 \notin B$, because there can be elements in $B$ that are not in $A$.

\subsection*{B}
Find a counterexample to the theorem. In other words, find an example
of sets $A$ and $B$ for which the hypotheses of the theorem are
true but the conclusion is false. \\

\n Let $A = \{1,2,3\}$ and $B = \{0,1,2,3\}$. Then the hypotheses of the theorem are true, specifically $\forall x \in A (x \neq 0)$ and $A \subseteq B$, but the conclusion $\forall x \in B (x \neq 0)$ is false.

\section*{3.3.22}
Consider the following incorrect theorem: \\

\n \textbf{Incorrect Theorem.} $\exists x \in \mathbb{R} \forall y \in \mathbb{R} (xy^2 = y - x)$. \\

\n What's wrong with the following proof of the theorem? \\

\begin{proof}
Let $x = \tfrac{y}{y^2+1}$. Then \\
\begin{equation*}
y - x = y - \frac{y}{y^2} = \frac{y^3}{y^2 + 1} = \frac{y}{y^2 + 1} \cdot y^2 = xy^2.
\end{equation*}
\end{proof}

\n In the proof, $x$ is defined in terms of $y$ but $y$ has not been introduced into the proof yet. The theorem should start with "Let $x = ...$. and let $y$ be an arbitrary real number...".

\section*{3.3.23}
\n Consider the following incorrect theorem: \\

\n \textbf{Incorrect Theorem} \textit{Suppose $\F$ and $\G$ are families of sets. If $\cup \F$ and $\cup G$ are disjoint, then so are $\F$ and $\G$.} \\

\subsection*{A}
What's wrong with the following proof of the theorem?

\begin{proof}
Suppose $\cup \F$ and $\cup G$ are disjoint. Suppose $\F$ and $\G$ are not disjoint. Then we can choose some set $A$ such that $A \in \F$ and $a \in \G$. Since $A \in \F$, by exercise 8, $A \subseteq \cup \F$, so every element of $A$ is in $\cup \F$. Similarly, since $A \in \G$, every element of $A$ is in $\cup \G$. But then every element of $A$ is in both $\cup \F$ and $\cup \G$, and this is impossible since $\cup \F$ and $\cup \G$ are disjoint. Thus, we have reached a contradiction, so $\F$ and $\G$ must be disjoint.
\end{proof}

\n The statement ``But then every element of $A$ is in both $\cup \F$ and $\cup \G$, and this is impossible since $\cup \F$ and $\cup \G$ are disjoint." is not correct. If $A = \{\varnothing\}$ then every element of $A$, which is $\varnothing$, is in both $\cup \F$ and $\cup \G$ and by definition $\cup \F$ and $\cup \G$ are disjoint, or $(\cup \F) \cap (\cup \G) = \varnothing$. So there is no contradiction in this case.

\subsection*{B}
Find a counterexample to the theorem. \\

\n Let $\F = \{\{\varnothing\}, \{1\}\}$ and $G = \{\{\varnothing\}, \{2\}\}$. Then $\cup \F = \{1, \varnothing\}$ and $\cup \G = \{2, \varnothing\}$ and therefore $\cup \F$ and $\cup \G$ are disjoint, or $(\cup \F) \cap (\cup \G) = \varnothing$. However, $\F$ and $\G$ are not disjoint because $\F \cap \G = \{\varnothing\}$. (Remember, the empty set $\varnothing$ is the set that has no elements, but the set $\{\varnothing\}$ is a set that contains one element, $\varnothing$.)

\section*{3.3.24}
Consider the following putative theorem: \\

\n \textbf{Theorem?} \textit{For all real numbers $x$ and $y$, $x^2 + xy - 2y^2 = 0$.} \\

\subsection*{A}
What's wrong with the following proof of the theorem? \\

\begin{proof}
Let $x$ and $y$ be equal to some arbitrary real number $r$. Then

\begin{equation*}
x^2 + xy - 2y^2 = r^2 + r \cdot r - 2r^2 = 0.
\end{equation*}

\n Since $x$ and $y$ were both arbitrary, this shows that for all real numbers $x$ and $y$, $x^2 + xy - 2y^2 = 0$. \\
\end{proof}

\n The first sentence of the proof assigns $x$ and $y$ to be the same arbitrary real number, however, it should start with ``Let $x$ be an arbitrary real number and let $y$ be an arbitrary real number" or ``let $x$ and $y$ be arbitrary real numbers". 

\subsection*{B}
No, the theorem is not correct. We will provide a counterexample. Let $x = 2$ and let $y = 3$, then

\begin{equation*}
2^2 + 2 \cdot 3 - 2 \cdot 3^2 = 4 + 6 - 2 \cdot 9 = 10 - 18 = -8
\end{equation*}

\n and $ -8 \neq 0$ so the theorem is not correct.

\section*{3.3.25}
Prove that for every real number $x$ there is a real number $y$ such that for every real number $z$, $yz = (x + z)^2 - (x^2 + z^2)$. \\

\n We want to prove
$\forall x \exists y \forall z (yz = (x + z)^2 - (x^2 + z^2))$.\\

\n We let $x$ be arbitrary and make our goal $\exists y \forall z (yz = (x + z)^2 - (x^2 + z^2))$. \\

\begin{table}[h]
\begin{tabular}{ll}
\hline
Givens & Goals   \\ \hline
$x$ is arbitrary & $\exists y \forall z (yz = (x + z)^2 - (x^2 + z^2))$ \\  \hline
\end{tabular}
\end{table}

\n Let $x$ be an arbitrary real number \\
\indent [proof of $\exists y \forall z (yz = (x + z)^2 - (x^2 + z^2))$] \\
\n Therefore $\forall x \exists y \forall z (yz = (x + z)^2 - (x^2 + z^2))$ \\

\n Now we need to find a $y$ that makes the statement $\forall z (yz = (x + z)^2 - (x^2 + z^2))$ true. This suggests we solve the equation $yz = (x + z)^2 - (x^2 + z^2)$ for $y$.

\begin{align*}
yz &= (x + z)^2 - (x^2 + z^2) \\
yz &= x^2 + 2xz + z^2 - x^2 - z^2 \\
yz &= 2xz \\
y &= 2x \\
\end{align*}

\n The last line above $y = 2x$ works even if $z = 0$ because in that case $yz = 0$ and $2xz = 0$ and there is no need to divide by $z$ because we have $0 = 0$. \\


\n Let $x$ be an arbitrary real number \\
\indent Let $y = 2x$ \\
\indent \indent Let $z$ be an arbitrary real number \\
\indent \indent \indent [proof of $yz = (x + z)^2 - (x^2 + z^2)$] \\
\indent \indent Therefore $\forall z (yz = (x + z)^2 - (x^2 + z^2))$ \\
\indent Therefore $\exists y \forall z (yz = (x + z)^2 - (x^2 + z^2))$ \\
\n Therefore $\forall x \exists y \forall z (yz = (x + z)^2 - (x^2 + z^2))$ \\

\n When writing the proof we have to make sure the order we introduce the variables is the same as above (i.e., we introduce $x$, $y$, and then $z$). When we state $y = 2x$ in the proof, we only have defined $x$ up to that point, so both values we choose for $x$ and $y$ must then work for every value of $z$, or every real number. (See https://github.com/kstratto/How-to-Prove-It/blob/master/How\%20to\%20Prove\%20It\%20-\%20Chapter\%203.pdf.)

\begin{theorem} For every real number $x$ there is a real number $y$ such that for every real number $z$, $yz = (x + z)^2 - (x^2 + z^2)$.
\end{theorem}
\begin{proof}
Let $x$ be an arbitrary real number. Let $y = 2x$. Let $z$ be an arbitrary real number. Then

\begin{equation*}
2xz = (x + z)^2 - (x^2 + z^2) = x^2 + 2xz + z^2 - x^2 - z^2 = 2xz
\end{equation*}
\end{proof}

\section*{Exercise 3.3.26}

\subsection*{A}
Comparing the various rules for dealing with quantifiers in proofs, you should see a similarity between the rules for goals of the form $\forall x P(x)$ and givens of the form $\exists x P(x)$. What is this similarity? What about the rules for goals of the form $\exists x P(x)$ and givens of the from $\forall x P(x)$? \\

\n Rules for goals of the form $\forall P(x)$ and givens of the form $\exists x P(x)$ are similar because the strategy for both of these involve introducing a new variable into the proof. In the case of a goal of the form $\forall P(x)$, say $\forall x \in A P(x)$, a new variable $y$ can be introduced that stands for an arbitrary element of the set $A$ and this new variable can be used like any other given. In the case of a given of the form $\exists x P(x)$, a new variable is also introduced, say $x_0$, that we assume makes the statement $P(x)$ of true. This new variable $x_0$ can also now be used as a given. With both of the new variables it is important not to make any other assumptions about them. \\

\n Rules for goals of the form $\exists x P(x)$ and givens of the form $\forall x P(x)$ are similar because both of these strategies involve introducing a specific value for $x$ that makes $P(x)$ true instead of just introducing a variable that is assumed to make $P(x)$ true.

\subsection*{B}
Can you think of a reason why these similarities might be expected? \\
\n (Hint: Think about how proof by contradiction works when the goal starts with a quantifier.)\\

\n When proving a goal of the form $\forall x P(x)$ by contradiction, we assume $\exists x \neg P(x)$ as a goal. When proving a goal of the form $\exists x P(x)$ by contradiction, we assume $\forall x \neg P(x)$ as a goal.\\

\n Maybe the similarities are to be expected because the strategies for each set of givens and goals (e.g., goals of the form $\forall x P(x)$ and givens of the form $\exists x P(x)$) are like inverses. For example, for the statement $\forall x P(x)$ to not be true then the statement $\exists x \neg P(x)$ must be true.


\end{document}

