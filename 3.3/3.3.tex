\documentclass{article}

%\documentclass{scrartcl}
%\usepackage{changepage}
%\usepackage{scrextend}

\usepackage{amssymb,amsmath,amsthm}
% amssymb has empty set symbo
\usepackage{scrextend} % for \begin{addmargin}[0.55cm]{0cm} text \end{margin}

\usepackage{mathrsfs} % for \mathscr{P}
\usepackage{float}


\newcommand{\n}{ \noindent }
\newcommand{\F}{\mathcal{F}}
\newcommand{\G}{\mathcal{G}}
%\newcommand{\pwset}{\mathcal{P}}
\newcommand{\pwset}{\mathscr{P}}


\newtheorem*{theorem}{Theorem}  % This enables \begin{theorem}

%\DeclareFontFamily{U}{MnSymbolC}{}
%\DeclareSymbolFont{MnSyC}{U}{MnSymbolC}{m}{n}
%\DeclareFontShape{U}{MnSymbolC}{m}{n}{
%  <-6>    MnSymbolC5
%  <6-7>   MnSymbolC6
%  <7-8>   MnSymbolC7
%  <8-9>   MnSymbolC8
%  <9-10>  MnSymbolC9
%  <10-12> MnSymbolC10
%  <12->   MnSymbolC12%
%}{}
%\DeclareMathSymbol{\powerset}{\mathord}{MnSyC}{180}

\begin{document}

\section*{Exercise 3.3.4}
\n Suppose $A \subseteq \pwset(A)$. Prove that $\pwset(A) \subseteq \pwset(\pwset(A))$. \\


\n So we want to prove that
$\forall x(x \in \pwset(A) \rightarrow x \in \pwset(\pwset(A)))$.\\

\n First we assume $x$ is arbitrary and make the antecedent a given and the consequent our goal to prove. \\

\begin{table}[h]
\begin{tabular}{ll}
\hline
Givens & Goals   \\ \hline
$A \subseteq \pwset(A)$ & $x \in \pwset(\pwset(A))$  \\ 
$x \in \pwset(A)$ & \\ \hline
\end{tabular}
\end{table}

\n Assume $x$ is an arbitrary element of $\pwset(A)$ \\
\indent Suppose $x \in \pwset(A)$ \\
\indent \indent [ proof of $x \in \pwset(\pwset(A))$ ] \\
\indent Therefore if $x \in \pwset(A) \rightarrow x \in \pwset(\pwset(A))$ \\
\n Since $x$ was arbitrary we can conclude $\forall x(x \in \pwset(A) \rightarrow x \in \pwset(\pwset(A)))$ \\

\n We can rewrite our goal as $x \subseteq \pwset(A)$ or $\forall y(y \in x \rightarrow y \in \pwset(A))$. So we assume $y$ is arbitrary and make the antecedent a given and the consequent our goal to prove.

\begin{table}[h]
\begin{tabular}{ll}
\hline
Givens & Goals   \\ \hline
$A \subseteq \pwset(A)$ & $y \in \pwset(A)$  \\ 
$x \in \pwset(A)$ & \\ 
$y \in x$ & \\ \hline
\end{tabular}
\end{table}

\n Assume $x$ is an arbitrary element of $\pwset(A)$ \\
\indent Suppose $x \in \pwset(A)$ \\
\indent \indent Suppose $y$ is an arbitrary element of $x$. \\
\indent \indent \indent Suppose $y \in x$. \\
\indent \indent \indent \indent[ proof of $y \in \pwset(A)$ ] \\
\indent \indent \indent Therefore if $y \in x \rightarrow y \in \pwset(A)$.\\
\indent \indent Since $y$ was arbitrary we can conclude that $x \subseteq \pwset(A)$. \\
\indent Therefore if $x \in \pwset(A) \rightarrow x \in \pwset(\pwset(A))$ \\
\n Since $x$ was arbitrary we can conclude $\forall x(x \in \pwset(A) \rightarrow x \in \pwset(\pwset(A)))$ \\

\n Now looking at our givens $x \in \pwset(A)$ means that $x \subseteq A$ or $\forall z(z \in x \rightarrow z \in A)$. Using universal instantiation we will plug in $y$ for $z$ and using modus ponens we can conclude that $y \in A$. \\

\n Now looking at our other given $A \subseteq \pwset(A) \rightarrow \forall m(m \in A \rightarrow m \in \pwset(A))$. Using universal instantiation we will plug in $y$ for $m$ and using modus ponens we can conclude that $y \in \pwset(A)$, which was our goal to prove.

\begin{theorem} Suppose $A \subseteq \pwset(A)$. Then $\pwset(A) \subseteq \pwset(\pwset(A))$.
\end{theorem}
\begin{proof}
Suppose $x$ is an arbitrary element of $\pwset(A)$ and $y$ is an arbitrary element of $x$. It follows that $y \in A$. But since $A \subseteq \pwset(A)$ then it also follows that $y \in \pwset(A)$. So $y \in x \rightarrow y \in \pwset(A)$ and since $y$ was arbitrary we can conclude that $x \subseteq \pwset(A)$. Therefore, if $x \in \pwset(A) \rightarrow x \in \pwset(\pwset(A))$. Since $x$ was arbitrary we can also conclude that $\pwset(A) \subseteq \pwset(\pwset(A))$.
\end{proof}

\n Alternate proof (not sure if this is correct)

\begin{proof}
Suppose $x$ is an arbitrary element of $\pwset(A)$. Then $x \in A$. Since $A \subseteq \pwset(A)$ and $x \in A$ then $x \subseteq \pwset(A)$. Therefore, $x \in \pwset(\pwset(A))$.
\end{proof}

\section*{Exercise 3.3.5}
\n The hypothesis of the theorem proven in exercise 3.3.4 is $A \subseteq \pwset(A)$.
\subsection*{A}
\n Can you think of a set $A$ for which this hypothesis is true? \\

\n The empty set $\varnothing$ is a set for which the hypothesis is true. \\

\n $A \subseteq \pwset(A)$ means $x \in A \rightarrow x \in \pwset(A)$. For $\varnothing$ this would mean that $x \in \varnothing \rightarrow x \in \pwset(\varnothing)$, but by definition there are no elements in $\varnothing$. Therefore $x \in \varnothing$ will always be false and the conditional statement $x \in \varnothing \rightarrow x \in \pwset(\varnothing)$ is always true. Therefore if $\varnothing = A$ then $A \subseteq \pwset(A)$. 

\subsection*{B}
\n Can you think of another? \\

\n In exercise 3.3.4 we proved that if $A \subseteq \pwset(A)$ then $\pwset(A) \subseteq \pwset(\pwset(A))$. Therefore, the set $\{\varnothing, \{\varnothing\}\}$, which is the $\pwset(A)$ if $A = \varnothing$, is another set for which the hypothesis is true. If we let $B = \pwset(A) = \{\varnothing, \{\varnothing\}\}$ and replace $A$ in the hypothesis $A \subseteq \pwset(A)$ with $B$, then we can conclude that $B \subseteq \pwset(B)$.

\section*{3.3.6}
Suppose $x$ is a real number.
\subsection*{A}
Prove that if $x \neq 1$ then there is a real number $y$ such that $\tfrac{y+1}{y-2} = x$. \\

So we want to prove that $(x \neq 1) \rightarrow \exists y \left( \tfrac{y+1}{y+2} = x \right)$ \\

We assume the antecedent and make the consequent our goal to prove. \\

\begin{table}[h]
\begin{tabular}{ll}
\hline
Givens & Goals   \\ \hline
$x \neq 1$ & $\exists y \left( \tfrac{y+1}{y+2} = x \right)$   \\ \hline
\end{tabular}
\end{table}

\n To prove our goal we need to find  a $y$ that makes the equation $ \tfrac{y+1}{y+2} = x $ true. So let's try solving the equation for $y$.

\begin{align*}
\frac{y+1}{y+2} &= x \\
y+1 &= x(y-2) \\
y+1 &= xy - 2x \\
2x+1 &= xy-y \\
2x+1 &= y(x-1) \\
y &= \frac{2x+1}{x-1}
\end{align*}

We see that this $y$ works because we have $x \neq 1$ as a given.

\begin{theorem} Suppose $x \neq 1$. Then there is a real number $y$ such that $\tfrac{y+1}{y-2} = x$.
\end{theorem}
\begin{proof}
Suppose $x \neq 1$ and $y = \tfrac{2x+1}{x-1}$. Then

\begin{equation*}
\frac{ \frac{2x+1}{x-1} +1 }{ \frac{2x+1}{x-1} -2} = \frac{ \frac{3x}{x-1} } { \frac{3}{x-1} } = \frac{3x}{x-1} \cdot \frac{x-1}{3} = x
\end{equation*}
\end{proof}

\subsection*{B}

\n Prove that if there is a real number $y$ such that $\tfrac{y+1}{y-2} = x$ then $x \neq 1$. \\

\n So we want to prove that
$\exists y \left(\frac{y+1}{y-1} = x \right) \rightarrow (x \neq 1)$ \\

\n We assume the antecedent and make the consequent our goal to prove. \\

\begin{table}[h]
\begin{tabular}{ll}
\hline
Givens & Goals   \\ \hline
$\exists y \left(\frac{y+1}{y-1} = x \right)$ & $x \neq 1$   \\ \hline
\end{tabular}
\end{table}


\n Using existential instantiation we assume there is a value $y_0$ such that $ \tfrac{y+1}{y-1} = x$ is true. From part A above, we know that $\left(\frac{y+1}{y-1} = x \right) \rightarrow \left( y = \tfrac{2x+1}{x-1} \right)$ and so $y_0 = \tfrac{2x+1}{x-1}$. Since $y$ is a real number, then clearly $x \neq 1$.

\begin{theorem} If $y$ is a real number and $\tfrac{y+1}{y-2} = x$ then $x \neq 1$.
\end{theorem}
\begin{proof}
Suppose $y$ is a real number and $\tfrac{y+1}{y-2} = x$. It follows that $y = \tfrac{2x+1}{x-1}$ and since $y$ is real number then $x \neq 1$.
\end{proof}

\section*{Exercise 3.3.7}
Prove for every real number $x$, if $x>2$ then there is a real number $y$ such that $y + \tfrac{1}{y} = x$. \\

\n So we want to prove
$\forall x \in \mathbb{R}(x>2 \rightarrow \exists y \in \mathbb{R}(y + \tfrac{1}{y} = x))$ \\

\n So we let $x$ be an arbitrary real number, then we assume the antecedent and make the consequent our goal to prove. \\

\begin{table}[h]
\begin{tabular}{ll}
\hline
Givens & Goals   \\ \hline
$x$ is arbitrary real number & $\exists y (y + \tfrac{1}{y} = x)$   \\ 
$x > 2$  & \\ \hline
\end{tabular}
\end{table}

\n Our goal is of the form $\exists y P(y)$ where $P(y)$ is $y + \tfrac{1}{y} = x$ and our strategy suggests we try to find a $y$ for which $P(y)$ is true. We can do this by solving the equation $y + \tfrac{1}{y} = x$ for $y$. We can rewrite this equation as $y^2 - \tfrac{x}{y} + 1 = 0$ and we see this is a quadratic equation and therefore we can use the quadratic formula to solve for $y$,



\begin{align*}
y = \frac{-(-x) \pm \sqrt{(-x)^2 - 4\cdot1\cdot1}}{2 \cdot 1} = \frac{x \pm \sqrt{x^2 - 4}}{2} \text{.}
\end{align*}

\n We note that $\sqrt{x^2 - 4}$ is defined because $x>2$. We have found two solutions that satisfy our original equation, but we only need one to complete the proof. We will use $\tfrac{x + \sqrt{x^2 - 4}}{2}$.

\begin{theorem} For every real number $x$, if $x>2$ then there is a real number $y$ such that $y + \tfrac{1}{y} = x$.
\end{theorem}
\begin{proof}
Suppose $x$ and $y$ are real numbers, $x>2$, and $y = \tfrac{x + \sqrt{x^2 - 4}}{2}$. Then

\begin{align*}
\frac{x + \sqrt{x^2 -4}}{2} + \frac{1}{ \frac{x + \sqrt(x^2 - 4}{2}} &= \frac{x + \sqrt{x^2 - 4}}{2} + \frac{2}{x + \sqrt{x^2 - 4}} \\
&= \frac{2x^2 + 2(x\sqrt{x^2 - 4})}{2x + 2\sqrt{x^2-4}} \\
&= x 
\end{align*}

\end{proof}

\section*{Exercise 3.3.8}
\n Prove that if $\F$ is a family of sets and $A \in \F$, then $A \subseteq \cup \F$. \\

\n So we want to prove that
$ A \in \F \rightarrow A \subseteq \cup \F$. \\

\n We assume the antecedent and make the consequent our goal to prove. \\

\begin{table}[h]
\begin{tabular}{ll}
\hline
Givens & Goals   \\ \hline
$ A \in \F$  & $A \subseteq \cup \F$   \\ \hline
\end{tabular}
\end{table}

\n Assume $A \in \F$ \\
\indent [ proof of $A \subseteq \cup \F$ ] \\
\n Therefore if $A \in \F$ then  $A \subseteq \cup \F$\\



\n Our goal $A \subseteq \cup \F$ can be rewritten as $\forall x (x \in A \rightarrow x \in \cup \F)$. We assume $x$ is arbitrary and then assume the antecedent and make the consequent our goal to prove. \\


\begin{table}[h]
\begin{tabular}{ll}
\hline
Givens & Goals   \\ \hline
$ A \in \F$  & $x \in \cup \F$   \\ 
$ x \in A$ & \\ \hline
\end{tabular}
\end{table}

\n Assume $A \in \F$ \\
\indent Assume $x$ is arbitrary \\
\indent \indent Assume $x \in A$ \\
\indent \indent \indent [ proof of $x \in \cup \F$ ] \\
\indent \indent Therefore if $x \in A$ then $x \cup \F$ \\
\indent Since $x$ was arbitrary we can conclude that $A \subseteq \cup \F$. \\
\n Therefore if $A \in \F$ then  $A \subseteq \cup \F$\\

\n Our new goal can be rewritten as $\exists B \in \F (x \in B)$. From our givens we see that $A \in \F$ and $x \in A$, so we have found a set such that $A \in \F(x \in A)$.

\begin{theorem} If $\F$ is a family of sets and $A \in \F$, then $A \subseteq \cup \F$.
\end{theorem}
\begin{proof}
Assume $A \in \F$ and $x$ is an arbitrary member of $A$. Then since $x \in A$ and $A \in \F$, it follows that $x \in \cup \F$. Since $x$ was arbitrary we can conclude that $A \subseteq \cup \F$ and therefore if $A \in \F$ then $A \subseteq \cup \F$.
\end{proof}

\section*{3.3.9}
Prove that if $\F$ is a family of sets and $A \in \F$, then $\cap \F \subseteq A$. \\

\n We want to prove that
$A \in \F \rightarrow \cap \F \subseteq A$. \\

\n We assume the antecedent and make the consequent our goal to prove. \\

\begin{table}[h]
\begin{tabular}{ll}
\hline
Givens & Goals   \\ \hline
$ A \in \F$  & $\cap \F \subseteq A$   \\ \hline
\end{tabular}
\end{table}

\n Assume $A \in \F$ \\
\indent [proof of $\cap \F \subseteq A$] \\
\n Therefore, if $A \in \F$ then $\cap \F \subseteq A$. \\

\n We can rewrite our goal as $\forall x(x \in \cap \F \rightarrow x \in A$). We assume $x$ is arbitrary and then assume the antecedent and make the consequent our goal to prove. \\

\begin{table}[h]
\begin{tabular}{ll}
\hline
Givens & Goals   \\ \hline
$ A \in \F$  & $x \in A$   \\ 
$x \in \cap \F$ & \\ \hline
\end{tabular}
\end{table}

\n Assume $A \in \F$ \\
\indent Assume $x$ is arbitrary \\
\indent \indent Assume $x \in \cap \F$ \\
\indent \indent \indent [proof of $x \in A$] \\
\indent \indent Therefore, if $x \in \cap \F$ then $x \in A$. \\
\indent Since $x$ was arbitrary we can conclude that $\cap \F \subseteq A$. \\
\n Therefore, if $A \in \F$ then $\cap \F \subseteq A$. \\

\n Our given $x \in \cap \F$ can be rewritten as $\forall B \in \F (x \in B)$, therefore if $A \in \F$ then $x \in A$, which was our goal to prove.

\begin{theorem} If $\F$ is a family of sets and $A \in \F$, then $\cup \F \in A$.
\end{theorem}
\begin{proof}
Assume $A \in \F$ and $x$ is an arbitrary member of $\cap \F$. Since $A \in \F$ and $x \in \cap \F$ it follows that $x \in A$ and therefore, if $x \in \cap \F$ then $x \in A$. Since $x$ was arbitrary we can conclude that $\cap \F \subseteq A$. Therefore, if $A \in \F$ then $\cap \F \subseteq A$.
\end{proof}

\section*{Exercise 3.3.10}
Suppose that $\F$ is a family of sets. Prove that if $\varnothing \in \F$ then $\cap \F = \varnothing$.


\section*{Exercise 3.3.12}
\n Suppose $\F$ and $\G$ are families of sets. Prove that if $\F \subseteq \G$ then 
$\cup \F \subseteq \cup \G$. \\

\n So we want to prove that 
$\F \subseteq \G \rightarrow \cup \F \subseteq \cup \G$ \\

\n First we assume the antecedent and make the consequent our goal to prove.

\begin{table}[h]
\begin{tabular}{ll}
\hline
Givens & Goals   \\ \hline
$\F \subseteq \G$ & $\cup \F \subseteq \cup \G$   \\ \hline

\end{tabular}
\end{table}

\n Suppose $\F \subseteq \G$ \\
\indent [proof of $\cup \F \subseteq \cup \G$ ] \\
\n So if $\F \subseteq \G \rightarrow \cup \F \subseteq \cup \G$ \\


\n $\cup \F \subseteq \cup \G \rightarrow \forall b (b \in \cup \F \rightarrow b \in \cup \G)$ so we assume $b$ is an arbitrary element of $\cup \F$ and assume the antecedent and make the consequent our goal to prove. \\

\begin{table}[h]
\begin{tabular}{ll}
\hline
Givens & Goals   \\ \hline
$\F \subseteq \G$ & $b \in \cup \G$   \\
$b \in \cup \F$ & \\ \hline
\end{tabular}
\end{table}

\n Suppose $\F \subseteq \G$ \\
\indent Let $b$ be an arbitrary element of $\cup \F$  \\
\indent \indent [proof of $b \in \cup \G$ ] \\
\indent Therefore if $b \in \cup \F \rightarrow b \in \cup \G$ \\
\n Since $b$ was arbitrary we can conclude $\forall b (b \in \cup \F \rightarrow b \in \cup \G)$. So if $\F \subseteq \G \rightarrow \cup \F \subseteq \cup \G$ \\

\n $b \in \cup \F \rightarrow \exists M (M \in \F \wedge b \in M)$, so let $M = A_{0}$ (Existential Instantiation)


\begin{table}[h]
\begin{tabular}{ll}
\hline
Givens & Goals   \\ \hline
$\F \subseteq \G$ & $b \in \cup \G$   \\
$A_{0} \in \F \wedge b \in A_{0}$ & \\ \hline
\end{tabular}
\end{table}

\n Suppose $\F \subseteq \G$
\begin{addmargin}[0.55cm]{0cm}
\n Let $b$ be an arbitrary element and suppose $b \in \cup \F$, which implies there is a set in $\F$ and $b$ is in that set. Let that set = $A_{0}$
\end{addmargin}

\indent \indent [proof of $b \in \cup \G$ ] \\
\indent Therefore if $b \in \cup \F \rightarrow b \in \cup \G$ \\
\n Since $b$ was arbitrary we can conclude $\forall b (b \in \cup \F \rightarrow b \in \cup \G)$. So if $\F \subseteq \G \rightarrow \cup \F \subseteq \cup \G$ \\

\n $\F \subseteq \G \rightarrow \forall A(A \in \F \rightarrow A \in \G)$. Using universal instantiation we will plug in $A_{0}$ for $A$ since then we can use modens ponens to conclude that $A_{0} \in \G$.

\begin{table}[h]
\begin{tabular}{ll}
\hline
Givens & Goals   \\ \hline
$A_{0} \in \F \rightarrow A_{0} \in \G$ & $b \in \cup \G$  \\
$A_{0} \in \F \wedge b \in A_{0}$ & \\ \hline
\end{tabular}
\end{table}

\n Our goal $b \in \cup \G \rightarrow \exists N (N \in \G \wedge b \in N)$, which we can now prove. Since $A_{0} \in \F$ and $\F$ is a subset of $\G$, it follows that $A_{0} \in G$. By the definition of $\cup \G$ it follows that $b \in \cup \G$ because $A_{0} \in G \wedge b \in A_{0}$, the latter statement being one of our givens. \\


\n \textbf{Theorem.} \textit{Suppose $\F$ and $\G$ are families of sets. If $\F \subseteq \G$ then 
$\cup \F \subseteq \cup \G$.}

\n \textit{Proof.} Suppose $\F \subseteq \G$. Let $b$ be an arbitrary element of $\cup \F$, which implies there is a set in $\mathcal{F}$ that contains $b$. Call this set $A_{0}$. Since $A_{0} \in \F$ and $\F$ is a subset of $\G$ it follows that $A_{0} \in G$, which implies that $b \in \cup \G$. Therefore if $b \in \cup \F$ then $b \in \cup \G$. Since $b$ was arbitrary we can conclude that if $\F \subseteq \G$ then $\cup \F \subseteq \cup \G$. This completes the proof.


\section*{Exercise 3.3.13}
\n Suppose $\F$ and $\G$ are families of sets. Prove that if $\F \subseteq \G$ then 
$\cap \G \subseteq \cap \F$. \\

\n So we want to prove that 
$\F \subseteq \G \rightarrow \cap \G \subseteq \cap \F$ \\

\n First we assume the antecedent and make the consequent our goal to prove.

\begin{table}[h]
\begin{tabular}{ll}
\hline
Givens & Goals   \\ \hline
$\F \subseteq \G$ & $\cap \G \subseteq \cap \F$   \\ \hline
\end{tabular}
\end{table}

\n Suppose $\F \subseteq \G$ \\
\indent [proof of $\cap \G \subseteq \cap \F$ ] \\
\n So if $\F \subseteq \G \rightarrow \cap \G \subseteq \cap \F$ \\

\n $\cap \G \subseteq \cap \F \rightarrow \forall b ( b \in \cap \G \rightarrow b \in \cap \F)$, so we assume $b$ is an arbitrary element of $\cap \G$ and assume the antecedent and make the consequent our goal to prove. \\

\begin{table}[h]
\begin{tabular}{ll}
\hline
Givens & Goals   \\ \hline
$\F \subseteq \G$ & $b \in \cap \F$   \\
$b \in \cap \G$ & \\ \hline
\end{tabular}
\end{table}

\n Suppose $\F \subseteq \G$ \\
\indent Let $b$ be an arbitrary element of $\cap \G$  \\
\indent \indent [proof of $b \in \cap \F$ ] \\
\indent Therefore if $b \in \cap \G \rightarrow b \in \cap \F$ \\
\n Since $b$ was arbitrary we can conclude $\forall b (b \in \cap \G \rightarrow b \in \cap \F)$. So $\F \subseteq \G \rightarrow \cap \G \subseteq \cap \F$ \\


\n $b \in \cap \F \rightarrow \forall A (A \in \F \rightarrow b \in A)$, so we assume $A$ is an arbitrary element of $\F$ and assume the antecedent and make the consequent our goal to prove. \\

\begin{table}[h]
\begin{tabular}{ll}
\hline
Givens & Goals   \\ \hline
$\F \subseteq \G$ & $b \in A$   \\
$b \in \cap \G$ & \\
$A \in \F$ & \\ \hline
\end{tabular}
\end{table}

\n Suppose $\F \subseteq \G$

\begin{addmargin}[0.55cm]{0cm}
\indent Let $b$ be an arbitrary element of $\cap \G$
\end{addmargin}

\indent \indent Suppose $A$ is an arbitrary set in $\F$ \\
\indent \indent \indent [proof of $b \in A$ ] \\
\indent \indent Therefore if $A \in \F \rightarrow b \in A $ \\
\indent \indent Since A was arbitrary we can conclude $b \in \cap \F$ \\
\indent Therefore if $b \in \cap \G \rightarrow b \in \cap \F$ \\
\n Since $b$ was arbitrary we can conclude $\forall b (b \in \cap \G \rightarrow b \in \cap \F)$. So $\F \subseteq \G \rightarrow \cap \G \subseteq \cap \F$ \\


\n Now looking at our givens, $\F \subseteq \G \rightarrow \forall Z ( Z \in \F \rightarrow Z \in \G)$. Using universal instantiation we will plug in $A$ for $Z$ and using modus ponens we can conclude that $A \in \G$. \\

\n Our other given, $b \in \cap \G \rightarrow \forall Y (Y \in \G \rightarrow b \in Y)$. Using universal instantiation we will plug in $A$ for $Y$ and using modus ponens we can conclude that $b \in A$, which was our goal, and we can now write our proof. \\


\n \textbf{Theorem.} \textit{Suppose $\F$ and $\G$ are families of sets. If $\F \subseteq \G$ then 
$\cap \G \subseteq \cap \F$.}
\n \textit{Proof.} Suppose $\F \subseteq \G$. Let $b$ be an arbitrary element of $\cap \G$. Suppose $A$ is an arbitrary element of $\F$, then because $\F \subseteq \G$ then it follows that $A \in \G$. By the definition of $\cap \G$ it follows that $b \in A$ and since $A$ was arbitrary then $b \in \cap \F$. Since $b$ was arbitrary we can conclude $\cap \G \subseteq \cap \F$ and therefore that if $\F \subseteq \G$ then $\cap \G \subseteq \cap \F$. This completes the proof.



\section*{Exercise 3.3.14}
% Potential answer I came across, haven't looked at it yet.
% https://math.stackexchange.com/questions/220572/the-union-of-powersets-is-contained-in-the-powerset-of-union

Suppose $\{ A_{i} | i \in I \} $ is an indexed family of sets. Prove that $\bigcup_{i \in I} \pwset(A_{i}) \subseteq \pwset (\bigcup_{i \in I} A_{i})$. \\

\n So we want to prove that $\forall a (a \in \bigcup_{i \in I} \pwset(A_{i}) \rightarrow a \in \pwset(\bigcup_{i \in I} A_i))$ \\

\n First we assume $a$ is arbitrary and make the antecedent a given and the consequent our goal to prove. \\

\begin{table}[h]
\begin{tabular}{ll}
\hline
Givens & Goals   \\ \hline
$a \in \bigcup_{i \in I} \pwset(A_{i})$ & $a \in \pwset(\bigcup_{i \in I} A_i)$   \\ \hline
\end{tabular}
\end{table}


\n Assume $a$ is an arbitrary element of $\bigcup_{i \in I} \pwset(A_{i})$ \\
\indent Suppose $a \in \bigcup_{i \in I} \pwset(A_{i})$ \\
\indent \indent [ proof of $a \in \pwset(\bigcup_{i \in I} A_i)$ ] \\
\indent Therefore if $a \in \bigcup_{i \in I} \pwset(A_{i}) \rightarrow a \in \pwset(\bigcup_{i \in I} A_i)$ \\
\n Since $a$ was arbitrary we can conclude $\bigcup_{i \in I} \pwset(A_{i}) \subseteq \pwset (\bigcup_{i \in I} A_{i})$ \\

\n Looking at our goal we see that $a \in \pwset(\bigcup_{i \in I} A_i) \rightarrow a \subseteq \bigcup_{i \in I} A_i \rightarrow \forall z (z \in a \rightarrow z \in \bigcup_{i \in I} A_i ) $. Therefore we assume $z$ is arbitrary, assume the antecedent, and make the consequent our goal to prove. \\

\begin{table}[h]
\begin{tabular}{ll}
\hline
Givens & Goals   \\ \hline
$a \in \bigcup_{i \in I} \pwset(A_{i})$ & $z \in \bigcup_{i \in I} A_i $   \\
$z \in a$ & \\ \hline

\end{tabular}
\end{table}

\n Assume $a$ is an arbitrary element of $\bigcup_{i \in I} \pwset(A_{i})$ \\
\indent Suppose $a \in \bigcup_{i \in I} \pwset(A_{i})$ \\
\indent \indent Assume z is arbitrary \\
\indent \indent \indent Assume $z \in a$ \\
\indent \indent \indent \indent [ proof of $z \in \bigcup_{i \in I} A_i $ ] \\
\indent \indent \indent Therefore $z \in a \rightarrow z \in \bigcup_{i \in I} A_i $ \\
\indent \indent Since z was arbitrary we can conclude $a \in \pwset(\bigcup_{i \in I} A_i)$ \\
\indent Therefore if $a \in \bigcup_{i \in I} \pwset(A_{i}) \rightarrow a \in \pwset(\bigcup_{i \in I} A_i)$ \\
\n Since $a$ was arbitrary we can conclude $\bigcup_{i \in I} \pwset(A_{i}) \subseteq \pwset (\bigcup_{i \in I} A_{i})$ \\

\n Looking at our given we see that $a \in \bigcup_{i \in I} \pwset(A_{i}) \rightarrow a \in \{ a | \exists i \in I (a \in \pwset(A_i)) \}$. Using existential instantiation we will select an $i$ such that $a \in \pwset(A_i)$ which implies $a \subseteq A_i$. Since $a \subseteq A_i \rightarrow \forall m(m \in a \rightarrow m \in A_i)$ and using universal instantiation we will plug in $z$ for $m$ and we get $\forall z(z \in a \rightarrow z \in A_i)$ and using modus ponens we can conclude that $z \in A_i$, which implies that $z \in \bigcup_{i \in I} A_i $, which was our goal. We can now right our proof. \\


\n \textbf{Theorem.} \textit{Suppose $\{ A_{i} | i \in I \} $ is an indexed family of sets, then $\bigcup_{i \in I} \pwset(A_{i}) \subseteq \pwset (\bigcup_{i \in I} A_{i})$.} \\
\n \textit{Proof.} Suppose that $a$ is an arbitrary element of $\bigcup_{i \in I} \pwset(A_{i})$. We choose an $i \in I$ such that $a \in \pwset(A_i)$, which implies that $a \subseteq A_i$. Suppose $z$ is an arbitrary element of $a$, then it follows that $z \in A_i$ and therefore $z \in \bigcup_{i \in I} A_i$. Since $z$ was an arbitrary element of $a$ then $a \subseteq \bigcup_{i \in I} A_i$, and it follows that $a \in \pwset(\bigcup_{i \in I} A_i)$. Thus we can conclude $\bigcup_{i \in I} \pwset(A_{i}) \subseteq \pwset (\bigcup_{i \in I} A_{i})$. This completes the proof.


\section*{3.3.15}
Suppose $ \{ A_i | i \in I \}$ is an indexed family of sets and $ I \neq \varnothing$. Prove that $\bigcap_{i \in I} A_i \in \bigcap_{i \in I} \pwset(A_i)$ \\

\n So we want to prove that $\forall y (y \in \bigcap_{i \in I} A_i \rightarrow y \in \bigcap_{i \in I} \pwset(A_i))$. \\

\n First we assume $y$ is arbitrary and make the antecedent a given and the consequent our goal to prove. \\

\begin{table}[h]
\begin{tabular}{ll}
\hline
Givens & Goals   \\ \hline
$y \in \bigcap_{i \in I} A_i$ & $y \in \bigcap_{i \in I} \pwset(A_i)$   \\ \hline
\end{tabular}
\end{table}

\n Suppose $y$ is arbitrary element of $\bigcap_{i \in I} A_i$. \\
\indent [proof of $y \in \bigcap_{i \in I} \pwset(A_i)$] \\
\n Since y was arbitrary we can conclude $\forall y (y \in \bigcap_{i \in I} A_i \rightarrow y \in \bigcap_{i \in I} \pwset(A_i))$. \\

\n Our goal $y \in \bigcap_{i \in I} \pwset(A_i)$ so we make $m$ an arbitrary element of $I$ and therefore $y \in \pwset(A_m) \rightarrow y \subseteq A_m \rightarrow \forall z (z \in y \rightarrow z \in A_m)$. So we make $z$ arbitrary and make the antecedent a given and the consequent our goal to prove. \\


\begin{table}[h]
\begin{tabular}{ll}
\hline
Givens & Goals   \\ \hline
$y \in \bigcap_{i \in I} A_i$ & $z \in A_m$   \\
$z \in y$ & \\ \hline
\end{tabular}
\end{table}

\n Suppose $y$ is arbitrary element of $\bigcap_{i \in I} A_i$.

\begin{addmargin}[0.55cm]{0cm}
Suppose $m$ is an arbitrary element of $I$ and therefore $y \in \pwset(A_m) \rightarrow y \subseteq A_m \rightarrow \forall z (z \in y \rightarrow z \in A_m)$.
\end{addmargin}

\indent \indent Suppose $z$ is an arbitrary element of $y$ \\
\indent \indent \indent [proof of $z \in A_m$] 
\begin{addmargin}[0.55cm]{0cm}
Therefore $z \in y \rightarrow z \in A_m$ and since $z$ was arbitrary $y \subseteq A_m \rightarrow y \in \pwset(A_m)$ and since $m$ was arbitrary $y \in \bigcap_{i \in I} \pwset(A_i)$
\end{addmargin}
\n Since y was arbitrary we can conclude $\forall y (y \in \bigcap_{i \in I} A_i \rightarrow y \in \bigcap_{i \in I} \pwset(A_i))$. \\

\n Now looking at our given $y \in \bigcap_{i \in I} A_i \rightarrow \forall i \in I(y \in A_i)$. Using universal instantiation we plug in $m$ for $i$ and therefore $y \in A_m$ and since $z \in y$ we can conclude $z \in A_m$, which was our goal. Now we can write our proof. \\

\n \textbf{Theorem.} \textit{Suppose $ \{ A_i | i \in I \}$ is an indexed family of sets and $ I \neq \varnothing$, then $\bigcap_{i \in I} A_i \in \bigcap_{i \in I} \pwset(A_i)$.} \\
\n \textit{Proof.} Suppose $y$ is an arbitrary element of $\bigcap_{i \in I} A_i$. Suppose $m$ is an arbitrary member of $I$ and therefore $y \subseteq A_m$ which implies $y \subseteq A_m$. Now suppose $z$ is an arbitrary element of $y$. Since $y \in \bigcap_{i \in I} A_i$ if we choose an $i$ such that $y \in \bigcap_{m \in I} A_m$ then $y \in A_m$ which implies $z \in A_m$. Therefore if $z \in y$ then $z \in A_m$ and since $z$ was arbitrary then $y \subseteq A_m$ or $y \in \pwset(A_m)$ and since $m$ was arbitrary then $y \in \bigcap_{i \in I} \pwset(A_i)$. Since $y$ was arbitrary then $\bigcap_{i \in I} A_i \in \bigcap_{i \in I} \pwset(A_i)$. This completes the proof.

\end{document}