\documentclass{article}

%\documentclass{amsart}
%\documentclass{scrartcl}
%\usepackage{changepage}
%\usepackage{scrextend}

\usepackage{amssymb,amsmath,amsthm}
% amssymb has empty set symbo
\usepackage{scrextend} % for \begin{addmargin}[0.55cm]{0cm} text \end{margin}

\usepackage{mathrsfs} % for \mathscr{P}
\usepackage{float}


\newcommand{\n}{ \noindent }
\newcommand{\F}{\mathcal{F}}
\newcommand{\G}{\mathcal{G}}
%\newcommand{\pwset}{\mathcal{P}}
\newcommand{\pwset}{\mathscr{P}}


\newtheorem*{theorem}{Theorem}  % This enables \begin{theorem}

%\DeclareFontFamily{U}{MnSymbolC}{}
%\DeclareSymbolFont{MnSyC}{U}{MnSymbolC}{m}{n}
%\DeclareFontShape{U}{MnSymbolC}{m}{n}{
%  <-6>    MnSymbolC5
%  <6-7>   MnSymbolC6
%  <7-8>   MnSymbolC7
%  <8-9>   MnSymbolC8
%  <9-10>  MnSymbolC9
%  <10-12> MnSymbolC10
%  <12->   MnSymbolC12%
%}{}
%\DeclareMathSymbol{\powerset}{\mathord}{MnSyC}{180}

%\usepackage[left=2cm,right=2cm,top=2cm,bottom=2cm]{geometry}

\begin{document}
\section*{3.4.1}
Use the methods of this chapter to prove that $\forall x (P(x) \land Q(x))$ is equivalent to $\forall x P(x) \land \forall x Q(x)$. \\

\n We want to prove
$\forall x(P(x) \land Q(x) \iff \forall x P(x) \land \forall x Q(x))$. \\

\begin{theorem} The statement $\forall x (P(x) \land Q(x))$ is equivalent to $\forall x P(x) \land \forall x Q(x)$.
\end{theorem}
\begin{proof}
$(\rightarrow)$ Suppose $\forall x (P(x) \land Q(x))$. Let $y$ be arbitrary. Since $\forall x (P(x) \land Q(x))$ it follows $P(y)$ and $Q(y)$. Since $y$ was arbitrary, we can conclude $\forall x P(x)$ and $\forall x Q(x)$ or $\forall x P(x) \land \forall x Q(x)$.

$(\leftarrow)$ Let $y$ be arbitrary. Since $\forall x P(x)$ and $\forall x Q(x)$ then it follows $P(y)$ and $Q(y)$. Since $y$ was arbitrary we can conclude $\forall x (P(x) \land Q(x))$. 
\end{proof}

\section*{3.4.2}
Prove that if $A \subseteq B$ and $A \subseteq C$ then $A \subseteq B \cap C$.

\begin{theorem} If $A \subseteq B$ and $A \subseteq C$ then $A \subseteq B \cap C$.
\end{theorem}
\begin{proof}
Let $x$ be arbitrary and suppose $x \in A$. Since $A \subseteq B$ then $x \in B$ and since $A \subseteq C$ then $x \in C$ or $x \in B \cap C$. Therefore, if $x \in A$ then $x \in B \cap C$ and since $x$ was arbitrary we can conclude $A \subseteq B \cap C$.
\end{proof}

\section*{3.4.3}
Suppose $A \subseteq B$. Prove that for every set $C$, $C \setminus B \subseteq C \setminus A$.

\begin{theorem} Suppose $A \subseteq B$, then for every set $C$, $C \setminus B \subseteq C \setminus A$.
\end{theorem}
\begin{proof}
Suppose $A \subseteq B$ and $C$ is an arbitrary set. Let $x$ be arbitrary and suppose $x \in C \setminus B$, which means $x \in C$ and $x \notin B$. Since $x \notin B$ and $A \subseteq B$, then $x \notin A$, which means that $x \in C \setminus A$. Therefore, if $x \in C \setminus B$ then $x \in C \setminus A$ and since $x$ and $C$ were arbitrary, we can conclude $\forall C (C \setminus B \subseteq C \setminus A).$
\end{proof}

\section*{3.4.5}
Prove that if $A \subseteq B \setminus C$ and $A \neq \varnothing$ then $B \not\subseteq C$.

\begin{theorem} If $A \subseteq B \setminus C$ and $A \neq \varnothing$ then $B \not\subseteq C$.
\end{theorem}
\begin{proof}
Let $x$ be arbitrary and suppose $x \in A$. Since $A \subseteq B \setminus C$ then $x \in B$ and $x \notin C$. Since $x$ was arbitrary we can conclude $B \not\subseteq C$.
\end{proof}

\section*{3.4.6}
Prove that for any sets $A$, $B$, and $C$, $A \setminus(B \cap C) = (A \setminus B) \cup (A \setminus C)$ finding a string of equivalences starting with $x \in A \setminus (B \cap C)$ and ending with $x \in (A \setminus B) \cup (A \setminus C)$.

\begin{theorem} for any sets $A$, $B$, and $C$, $A \setminus(B \cap C) = (A \setminus B) \cup (A \setminus C)$.
\end{theorem}
\begin{proof}
Suppose $A$, $B$, and $C$ are arbitrary sets. Then
\begin{align*}
x \in A \setminus (B \cap C) ~ &\text{iff} ~ x \in A \rightarrow (x \notin B \land x \notin C) \\
&\text{iff} ~ x \notin A \lor (x \notin B \land x \notin C) \\
&\text{iff} ~ (x \notin A \lor x \notin B) \land (x \notin A \lor x \notin C) \\
&\text{iff} ~ (x \in A \rightarrow x \notin B) \lor (x \in A \rightarrow x \notin C) \\
&\text{iff} ~ x \in A \setminus B \lor x \in A \setminus C \\
&\text{iff} ~ x \in (A \setminus B) \cup (A \setminus C) \\
\end{align*}
\end{proof}

\section*{3.4.7}

\begin{theorem} For any sets $A$ and $B$, $\pwset(A \cap B) = \pwset(A) \cap \pwset(B)$.
\end{theorem}
\begin{proof}
$(\rightarrow)$ Let $M$ be an arbitrary set and suppose $M \in \pwset(A \cap B)$. Then $M \subseteq A \cap B$. Let $x$ be arbitrary and suppose $x \in M$. Since $M \subseteq A \cap B$, $x \in A \cap B$ and therefore $x \in A$. Since $x$ was arbitrary, $M \subseteq A$ and therefore $M \in \pwset(A)$. Similarly, since $M \subseteq A \cap B$, $x \in B$. Since $x$ was arbitrary, $M \subseteq B$ and therefore $M \in \pwset(B)$. Therefore, $M \in \pwset(A)$ and $M \in \pwset(B)$.

$(\leftarrow)$ Now suppose $M \in \pwset(A) \cap \pwset(B)$. Then $M \subseteq A$ and $M \subseteq B$. Suppose $x \in M$. Since $M \subseteq A$ and $M \subseteq B$ then $x \in A \cap B$. Since $x$ was arbitrary, $M \subseteq A \cap B$ and therefore $M \in \pwset(A \cap B)$.
\end{proof}

\section*{3.4.8}
\begin{theorem} $A \subseteq B \iff \pwset(A) \subseteq \pwset(B)$
\end{theorem}
\begin{proof}
$(\rightarrow)$ Suppose $A \subseteq B$. Let $M$ be an arbitrary set and suppose $M \in \pwset(A)$. Then $M \subseteq A$. Now let $y$ be arbitrary and suppose $y \in M$. Since $M \subseteq A$ then $y \in A$, and since $A \subseteq B$ then $y \in B$. Since $y$ was arbitrary, $M \subseteq B$ and therefore $M \in \pwset(B)$. Since $M$ was arbitrary, $\pwset(A) \subseteq \pwset(B)$.

$(\leftarrow)$ Now suppose $\pwset(A) \subseteq \pwset(B)$ and $y \in A$. Then the set $\{y\}$ is in $\pwset(A)$. Since $\pwset(A) \subseteq \pwset(B)$ then $\{y\} \in \pwset(B)$ and $y \in B$. Since $y$ was arbitrary, $A \subseteq B$. 
\end{proof}

\section*{3.4.9}
\begin{theorem} If $x$ and $y$ are odd integers, then $xy$ is odd.
\end{theorem}
\begin{proof}
Suppose $x$ and $y$ are odd integers. This means there is an integer $k$ such that $x = 2k + 1$ and there is an integer $j$ such that $y = 2j + 1$. Therefore, $xy = 2(2kj + k + j) = 4kj + 2k + 2j + 1 = (2k + 1)(2j + 1)$, and since $2kj + k + j$ is an integer, then $xy$ is odd.
\end{proof}

\section*{3.4.10}
\begin{theorem} For every integer $n$, $n^3$ is even iff $n$ is even.
\end{theorem}
\begin{proof}
$(\rightarrow)$ Let $n$ be arbitrary. We will prove the contrapositive. Suppose $x$ is odd, which means there exists an integer $k$ such that $x = 2k + 1$. Therefore, $n^3 = (2k + 1)^3 = 8k^3 + 12k^2 + 6k + 1 = 2(4k^3 + 6k^2 + 3k) + 1$. Since $4k^3 + 6k^2 + 3k$ is an integer, $n^3$ is odd. Therefore, if $n^3$ is even, $n$ is even.

$(\leftarrow)$ Now suppose $n$ is even, which means there exists an integer $m$ such that $n = 2m$. Now $n^3 = (2m)^3 = 8m^3 = 2(4m^3)$ and since $4m^3$ is an integer, $n^3$ is even.
\end{proof}

\section*{3.4.11}
\subsection*{A}
The problem is with using the same variable $k$ for defining $m$ as an even integer and $n$ as an odd integer when $k$ may take on different values for $n$ and $m$.

\subsection*{B}
Let $m = 2$ and $n = -3$. Then $n^2 - m^2 = (-3)^2 - 2^2 = 9 - 4 = 5$ and $n + m = -3 + 2 = -1$. Therefore $n^2 - m^2 \neq n + m$.


\section*{3.4.12}
\begin{theorem} $\forall x \in \mathbb{R}[\exists y \in \mathbb{R} (x + y = xy) \iff x \neq 1] $
\end{theorem}
\begin{proof}
$(\rightarrow)$ We will prove by contradiction. Suppose $x$ is an arbitrary real number and there exists a real number $y$ such that $x + y = xy$. Now suppose $x = 1$. Since $x + y = xy$, then $y = \tfrac{x}{x - 1}$. But this contradicts $x = 1$ because there is no real number $y$ such that $y = x/0$.

$(\leftarrow)$ Now suppose $x \neq 1$ and $y = \tfrac{x}{x-1}$. Then

\begin{align*}
x + y &= x + \frac{x}{x+1} = \frac{x(x-1) + x}{x-1} \\
&= \frac{x^2 - x + x}{x-1} \\
&= \frac{x^2}{x-1} = xy
\end{align*}
\end{proof}


\section*{3.4.13}
\begin{theorem} $\exists z \in \mathbb{R} \forall x \in \mathbb{R}^+ [\exists y \in \mathbb{R} (y - x = \tfrac{y}{x}) \iff x \neq z]$
\end{theorem}
\begin{proof}
$(\rightarrow)$ Let $z = 1$. Let $x$ be an arbitrary real number and suppose $x > 0$. Suppose $y \in \mathbb{R}$ and $y - x = \tfrac{y}{x}$. Then $y = \tfrac{x^2}{x-1}$. Now suppose $x = 1$. This contradicts $y \in \mathbb{R}$ and $y = \tfrac{x^2}{x-1}$. Therefore, $x \neq z$ and since $x$ was arbitrary we can conclude $\exists z \in \mathbb{R} \forall x \in \mathbb{R}^+ [\exists y \in \mathbb{R} (y - x = \tfrac{y}{x}) \rightarrow x \neq z]$.

$(\leftarrow)$ Now suppose $x \neq 1$ and $y = \tfrac{x^2}{x-1}$. Then

\begin{align*}
y - x = \frac{x^2}{x-1} - x &= \frac{x^2 - x(x-1)}{x-1} \\
&= \frac{x^2-x+2+x}{x-1} = \frac{x}{x-1} = \frac{y}{x}
\end{align*}
\end{proof}

\section*{3.4.14}
\begin{theorem} If $B$ is a set and $\F$ is a family of sets, then $\cup \{A \setminus B | A \in \F\} \subseteq \cup (\F \setminus \pwset(B))$.
\end{theorem}
\begin{proof}
Let $x$ be arbitrary and suppose $x \in \cup\{A \setminus B | A \in \F\}$. This means that there is a set $A \in \F$ such that $x \in A$ and also $x \notin B$. Since $x \in A$ and $x \notin B$, then $A \not\subseteq B$ and $A \notin \pwset(B)$. Thus there is a set $A \in \F$ such that $x \in A$, and $A \notin \pwset(B)$, which means that $x \in \cup(\F \setminus \pwset(B))$. Therefore, if $x \in \cup \{A \setminus B | A \in \F\}$ then $x \in \cup(\F \setminus \pwset(B))$ and since $x$ was arbitrary, we can conclude $\cup \{ A \setminus B | A \in \F\} \subseteq \cup (\F \setminus \pwset (B))$.
\end{proof}

\section*{3.4.15}
\begin{theorem}If $\F$ and $\G$ are nonempty families of sets and every element of $\F$ is disjoint from some element of $\G$, then $\cup \F$ and $\cap \G$ are disjoint.
\end{theorem}
\begin{proof}
Suppose $\F$ and $\G$ are nonempty families of sets and every element of $\F$ is disjoint from some element of $\G$. We will use proof by contradiction. Now suppose $\cup \F$ and $\cap \G$ are not disjoint. Then there exists a $y$ such that $y \in \cup \F$ and $y \in \cap \G$. Since $y \in \cup \F$ there is a set in $\F$ that contains $y$ and since $y \in \cap \G$, $y$ is in every set in $\G$. But because every element of $\F$ is disjoint from some element of $\G$, then there is at least one set in $\G$ that does not contain $y$. But this contradicts $y \in \cap \G$. Therefore, $(\cup \F) \cap (\cap \G) = \varnothing$. 
\end{proof}

\section*{3.4.16}
\begin{theorem} For any set $A$, $A = \cup \pwset(A)$.
\end{theorem}
\begin{proof}
$(\rightarrow)$ Suppose $A$ is an arbitrary set, $x$ is arbitrary, and $x \in A$. Then there is subset of $A$ that contains $x$ and, by definition, this subset is in $\pwset(A)$. Therefore, $x \in \cup \pwset(A)$. Since $x$ was arbitrary $A \subseteq \pwset(A)$.

$(\leftarrow)$ Now suppose $x \in \cup \pwset(A)$. This means there is a subset of $A$ that contains $x$ and therefore $x \in A$. Since $x$ was arbitrary we conclude $\cup \pwset(A) \subseteq A$. Since $A$ was arbitrary, we can conclude for all sets $A$, $A = \cup \pwset(A)$. 
\end{proof}

\section*{3.4.17}
\subsection*{A}
\begin{theorem} $\cup ( \F \cap \G) \subseteq (\cup \F) \cap (\cup \G)$
\end{theorem}
\begin{proof}
Let $x$ be arbitrary and suppose $x \in \cup (\F \cap \G)$. Since $x \in \cup (\F \cap \G)$ there is a set in $\F$ and in $\G$ that both contain $x$. Since there is a set in $\F$ than contains $x$, then $x \in \cup \F$ and since there is a set in $\G$ that contains $x$, $x \in \cup \G$. Therefore, $x \in (\cup \F) \cap (\cup \G)$. Since $x$ was arbitrary, we can conclude $\cup (\F \cap \G) \subseteq (\cup \F) \cap (\cup \G)$.
\end{proof}

\subsection*{B}
The mistake is that we can't choose a set $A$ such that $A \in \F$ and $A \in \G$ and $x \in A$. The given $x \in (\cup \F) \cap (\cup \G)$ means that $x$ is within a set in $\F$ and within a set in $\G$, but these two sets are not necessarily the same set.

\subsection*{C}
Let $\F = \{\{1,2\}, \{3\}\}$ and $\G = \{\{4,5\}, \{1\}\}$. Then $\cup (\F \cap \G) = \varnothing$, but $(\cup \F) \cap (\cup \G) = \{1\}$.


\section*{3.4.18}
\begin{theorem} Suppose $\F$ and $\G$ are families of sets, then $(\cup \F) \cap (\cup \G) \subseteq \cup (\F \cap \G) \iff \forall A \in \F \forall B \in \G (A \cap B \subseteq \cup (\F \cap \G))$.
\end{theorem}
\begin{proof}
$(\rightarrow)$ Suppose $(\cup \F) \cap (\cup \G) \subseteq \cup (\F \cap \G)$. Suppose $A$ is an arbitrary set in $\F$, $B$ is an arbitrary set in $\G$, $x$ is arbitrary, and $x \in A \cap B$. Since $x \in A \cap B$ and $A$ is an arbitrary set in $\F$, then $x \in \cup F$. Also, since $x \in A \cap B$ and $B$ is an arbitrary set in $\G$, then $x \in  \cup \G$. Therefore $x \in (\cup \F) \cap (\cup \G)$ and since $(\cup \F) \cap (\cup \G) \subseteq \cup (\F \cap \G)$, it follows that $x \in \cup (\F \cap \G)$. Therefore, if $x \in (\cup \F) \cap (\cup \G) \rightarrow x \in \cup (\F \cap \G)$ and since $x$, $A$, and $B$ were arbitrary we can conclude that $\forall A \in \F \forall B \in \G (A \cap B \subseteq \cup (\F \cap \G))$.

$(\leftarrow)$ Now suppose $\forall A \in \F \forall B \in \G (A \cap B \subseteq \cup (\F \cap \G))$ and $x \in (\cup \F) \cap (\cup \G)$. Since $x \in (\cup \F) \cap (\cup \G)$, then there is a set $M \in F$ such that $x \in M$ and there is a set $N \in \G$ such that $x \in N$ and it follows that $x \in M \cap \G$. Then since $M \in \F$, $N \in \G$, $x \in M \cap \G$, and $\forall A \in \F \forall B \in \G (A \cap B \subseteq \cup (\F \cap \G))$ we can conclude that $x \in \cup (\F \cap \G))$. Therefore if $\forall A \in \F \forall B \in \G (A \cap B \subseteq \cup (\F \cap \G))$ then $(\cup \F) \cap (\cup \G) \subseteq \cup (\F \cap \G)$.

\end{proof}

\section*{3.4.19}

\begin{theorem} Suppose $\F$ and $\G$ are families of sets. Prove that $\cup \F$ and $\cup \G$ are disjoint iff for all $A \in \F$ and $B \in \G$, $A$ and $B$ are disjoint.
\end{theorem}

\begin{proof}
$(\rightarrow)$ Suppose $\cup \F \cap \cup \G = \varnothing$. We will prove by contradiction. Let $A$ be an arbitrary set in $\F$ and $B$ be an arbitrary set in $\G$. Suppose $x \in A \cap B$, which means $x \in A$, $x \in B$, and $A \cap B \neq \varnothing$. Since $x \in A$ and $A \in \F$ then $x \in \cup \F$ and since $x \in B$ and $B \in \G$ then $x \in \cup \G$. Therefore $x \in \cup \F \cap \cup \G$, but this contradicts $\cup \F \cap \cup \G = varnothing$. Therefore $A \cap B = \varnothing$ and since $A$ and $B$ were arbitrary we can conclude $\forall A \in \F \forall B \in \G (A \cap B = \varnothing)$.

$(\leftarrow)$ Now suppose $\forall A \in \F \forall B \in \G (A \cap B = \varnothing)$. We will again prove by contradiction. Suppose $\cup \F$ and $\cup \G$ are not disjoint, which means there is an element $x$ that is in both $\cup \F$ and $\cup \G$. This means that there is a set in $\F$ that contains $x$ and there is a set in $\G$ that contains $x$. However, this contradicts our given that every set in $\F$ is disjoint from every set in $\G$. Therefore $\cup \F \cap \cup \G = \varnothing$.
\end{proof}


\end{document}