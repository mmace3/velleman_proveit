\documentclass{article}

%\documentclass{scrartcl}
%\usepackage{changepage}
%\usepackage{scrextend}

\usepackage{amssymb,amsmath,amsthm}
% amssymb has empty set symbo
\usepackage{scrextend} % for \begin{addmargin}[0.55cm]{0cm} text \end{margin}

\usepackage{mathrsfs} % for \mathscr{P}
\usepackage{float}


\newcommand{\n}{ \noindent }
\newcommand{\F}{\mathcal{F}}
\newcommand{\G}{\mathcal{G}}
%\newcommand{\pwset}{\mathcal{P}}
\newcommand{\pwset}{\mathscr{P}}


\newtheorem*{theorem}{Theorem}  % This enables \begin{theorem}

%\DeclareFontFamily{U}{MnSymbolC}{}
%\DeclareSymbolFont{MnSyC}{U}{MnSymbolC}{m}{n}
%\DeclareFontShape{U}{MnSymbolC}{m}{n}{
%  <-6>    MnSymbolC5
%  <6-7>   MnSymbolC6
%  <7-8>   MnSymbolC7
%  <8-9>   MnSymbolC8
%  <9-10>  MnSymbolC9
%  <10-12> MnSymbolC10
%  <12->   MnSymbolC12%
%}{}
%\DeclareMathSymbol{\powerset}{\mathord}{MnSyC}{180}

%\usepackage[left=2cm,right=2cm,top=2cm,bottom=2cm]{geometry}

\begin{document}
\section*{3.4.1}
Use the methods of this chapter to prove that $\forall x (P(x) \land Q(x))$ is equivalent to $\forall x P(x) \land \forall x Q(x)$. \\

\n We want to prove
$\forall x(P(x) \land Q(x) \iff \forall x P(x) \land \forall x Q(x))$. \\

\begin{theorem} The statement $\forall x (P(x) \land Q(x))$ is equivalent to $\forall x P(x) \land \forall x Q(x)$.
\end{theorem}
\begin{proof}
$(\rightarrow)$ Suppose $\forall x (P(x) \land Q(x))$. Let $y$ be arbitrary. Since $\forall x (P(x) \land Q(x))$ it follows $P(y)$ and $Q(y)$. Since $y$ was arbitrary, we can conclude $\forall x P(x)$ and $\forall x Q(x)$ or $\forall x P(x) \land \forall x Q(x)$.

$(\leftarrow)$ Let $y$ be arbitrary. Since $\forall x P(x)$ and $\forall x Q(x)$ then it follows $P(y)$ and $Q(y)$. Since $y$ was arbitrary we can conclude $\forall x (P(x) \land Q(x))$. 
\end{proof}




\end{document}