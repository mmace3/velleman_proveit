\documentclass{article}

%\documentclass{scrartcl}
%\usepackage{changepage}
%\usepackage{scrextend}

\usepackage{amssymb,amsmath,amsthm}
% amssymb has empty set symbo
\usepackage{scrextend} % for \begin{addmargin}[0.55cm]{0cm} text \end{margin}

\usepackage{mathrsfs} % for \mathscr{P}
\usepackage{float}


\newcommand{\n}{ \noindent }
\newcommand{\F}{\mathcal{F}}
\newcommand{\G}{\mathcal{G}}
%\newcommand{\pwset}{\mathcal{P}}
\newcommand{\pwset}{\mathscr{P}}


\newtheorem*{theorem}{Theorem}  % This enables \begin{theorem}

%\DeclareFontFamily{U}{MnSymbolC}{}
%\DeclareSymbolFont{MnSyC}{U}{MnSymbolC}{m}{n}
%\DeclareFontShape{U}{MnSymbolC}{m}{n}{
%  <-6>    MnSymbolC5
%  <6-7>   MnSymbolC6
%  <7-8>   MnSymbolC7
%  <8-9>   MnSymbolC8
%  <9-10>  MnSymbolC9
%  <10-12> MnSymbolC10
%  <12->   MnSymbolC12%
%}{}
%\DeclareMathSymbol{\powerset}{\mathord}{MnSyC}{180}

%\usepackage[left=2cm,right=2cm,top=2cm,bottom=2cm]{geometry}

\begin{document}
\section*{3.4.1}
Use the methods of this chapter to prove that $\forall x (P(x) \land Q(x))$ is equivalent to $\forall x P(x) \land \forall x Q(x)$. \\

\n We want to prove
$\forall x(P(x) \land Q(x) \iff \forall x P(x) \land \forall x Q(x))$. \\

\begin{theorem} The statement $\forall x (P(x) \land Q(x))$ is equivalent to $\forall x P(x) \land \forall x Q(x)$.
\end{theorem}
\begin{proof}
$(\rightarrow)$ Suppose $\forall x (P(x) \land Q(x))$. Let $y$ be arbitrary. Since $\forall x (P(x) \land Q(x))$ it follows $P(y)$ and $Q(y)$. Since $y$ was arbitrary, we can conclude $\forall x P(x)$ and $\forall x Q(x)$ or $\forall x P(x) \land \forall x Q(x)$.

$(\leftarrow)$ Let $y$ be arbitrary. Since $\forall x P(x)$ and $\forall x Q(x)$ then it follows $P(y)$ and $Q(y)$. Since $y$ was arbitrary we can conclude $\forall x (P(x) \land Q(x))$. 
\end{proof}

\section*{3.4.2}
Prove that if $A \subseteq B$ and $A \subseteq C$ then $A \subseteq B \cap C$.

\begin{theorem} If $A \subseteq B$ and $A \subseteq C$ then $A \subseteq B \cap C$.
\end{theorem}
\begin{proof}
Let $x$ be arbitrary and suppose $x \in A$. Since $A \subseteq B$ then $x \in B$ and since $A \subseteq C$ then $x \in C$ or $x \in B \cap C$. Therefore, if $x \in A$ then $x \in B \cap C$ and since $x$ was arbitrary we can conclude $A \subseteq B \cap C$.
\end{proof}

\section*{3.4.3}
Suppose $A \subseteq B$. Prove that for every set $C$, $C \setminus B \subseteq C \setminus A$.

\begin{theorem} Suppose $A \subseteq B$, then for every set $C$, $C \setminus B \subseteq C \setminus A$.
\end{theorem}
\begin{proof}
Suppose $A \subseteq B$ and $C$ is an arbitrary set. Let $x$ be arbitrary and suppose $x \in C \setminus B$, which means $x \in C$ and $x \notin B$. Since $x \notin B$ and $A \subseteq B$, then $x \notin A$, which means that $x \in C \setminus A$. Therefore, if $x \in C \setminus B$ then $x \in C \setminus A$ and since $x$ and $C$ were arbitrary, we can conclude $\forall C (C \setminus B \subseteq C \setminus A).$
\end{proof}

\section*{3.4.5}
Prove that if $A \subseteq B \setminus C$ and $A \neq \varnothing$ then $B \not\subseteq C$.

\begin{theorem} If $A \subseteq B \setminus C$ and $A \neq \varnothing$ then $B \not\subseteq C$.
\end{theorem}
\begin{proof}
Let $x$ be arbitrary and suppose $x \in A$. Since $A \subseteq B \setminus C$ then $x \in B$ and $x \notin C$. Since $x$ was arbitrary we can conclude $B \not\subseteq C$.
\end{proof}

\section*{3.4.6}
Prove that for any sets $A$, $B$, and $C$, $A \setminus(B \cap C) = (A \setminus B) \cup (A \setminus C)$ finding a string of equivalences starting with $x \in A \setminus (B \cap C)$ and ending with $x \in (A \setminus B) \cup (A \setminus C)$.

\begin{theorem} for any sets $A$, $B$, and $C$, $A \setminus(B \cap C) = (A \setminus B) \cup (A \setminus C)$.
\end{theorem}
\begin{proof}
Suppose $A$, $B$, and $C$ are arbitrary sets. Then
\begin{align*}
x \in A \setminus (B \cap C) ~ &\text{iff} ~ x \in A \rightarrow (x \notin B \land x \notin C) \\
&\text{iff} ~ x \notin A \lor (x \notin B \land x \notin C) \\
&\text{iff} ~ (x \notin A \lor x \notin B) \land (x \notin A \lor x \notin C) \\
&\text{iff} ~ (x \in A \rightarrow x \notin B) \lor (x \in A \rightarrow x \notin C) \\
&\text{iff} ~ x \in A \setminus B \lor x \in A \setminus C \\
&\text{iff} ~ x \in (A \setminus B) \cup (A \setminus C) \\
\end{align*}

\end{proof}

\end{document}