\documentclass{article}

%\documentclass{amsart}
%\documentclass{scrartcl}
%\usepackage{changepage}
%\usepackage{scrextend}

\usepackage{amssymb,amsmath,amsthm}
% amssymb has empty set symbo
\usepackage{scrextend} % for \begin{addmargin}[0.55cm]{0cm} text \end{margin}

\usepackage{mathrsfs} % for \mathscr{P}
\usepackage{float}


\newcommand{\n}{ \noindent }
\newcommand{\F}{\mathcal{F}}
\newcommand{\G}{\mathcal{G}}


%\newcommand{\pwset}{\mathcal{P}}
\newcommand{\pwset}{\mathscr{P}}


\newtheorem*{theorem}{Theorem}  % This enables \begin{theorem}

%\DeclareFontFamily{U}{MnSymbolC}{}
%\DeclareSymbolFont{MnSyC}{U}{MnSymbolC}{m}{n}
%\DeclareFontShape{U}{MnSymbolC}{m}{n}{
%  <-6>    MnSymbolC5
%  <6-7>   MnSymbolC6
%  <7-8>   MnSymbolC7
%  <8-9>   MnSymbolC8
%  <9-10>  MnSymbolC9
%  <10-12> MnSymbolC10
%  <12->   MnSymbolC12%
%}{}
%\DeclareMathSymbol{\powerset}{\mathord}{MnSyC}{180}

%\usepackage[left=2cm,right=2cm,top=2cm,bottom=2cm]{geometry}

\begin{document}
\section*{3.5.1}
Suppose $A$, $B$, and $C$ are sets.

\begin{theorem} $A \cap ( B \cup C ) \subseteq ( A \cap B ) \cup C $
\end{theorem}
\begin{proof}
Let $x$ be arbitrary and suppose $x \in A \cap ( B \cup C )$. Thus $x \in A$ and $x \in B$ or $x \in C$. If $x \in C$ then $x \in ( A \cap B ) \cup C$. In the case where $x \in B$ it follows that $x \in A \cap B$ and therefore $x \in ( A \cap B ) \cup C$. Since $x$ was arbitrary we can conclude that $A \cap ( B \cup C ) \subseteq ( A \cap B ) \cup C $.
\end{proof}

\section*{3.5.2}
Suppose $A$, $B$, and $C$ are sets.

\begin{theorem} $(A \cup B) \setminus C \subseteq A \cup (B \setminus C)$
\end{theorem}
\begin{proof}
Let $x$ be arbitrary and suppose $x \in ( A \cup B ) \setminus C$. Thus $x \notin C$ and $x \in A$ or $x \in B$. If $x \in A$ then $x \in A \cup ( B \setminus C )$. If $x \in B$ then if follows that $x \in B \setminus C$ and therefore $x \in A \cup ( B \setminus C )$. Since $x$ was arbitrary we can conclude $A \cap ( B \cup C ) \subseteq ( A \cap B ) \cup C$.
\end{proof}

\section*{3.5.3}
Suppose $A$ and $B$ are sets.

\begin{theorem} $A \setminus ( A \setminus B ) = A \cap B$
\end{theorem}

\begin{proof}
Let $x$ be arbitrary and suppose $x \in A \setminus ( A \setminus B )$. Then
\begin{align*}
x \in A \setminus ( A \setminus B ) ~ &\text{iff} ~ x \in A \land x \notin A \setminus B \\
&\text{iff} ~  x \in A \land \lnot ( x \in A \land x \notin B ) \\
&\text{iff} ~ x \in A \land ( x \notin A \lor x \in B ) \\
&\text{iff} ~ ( x \in A \land x \notin A ) \lor ( x \in A \land x \in B ) \\
&\text{iff} ~ x \in A \land x \in B \\
&\text{iff} ~ x \in ( A \cap B )
\end{align*}
\end{proof}

\section*{3.5.4}
\begin{theorem} If $A \cap C \subseteq B \cap C$ and $A \cup C \subseteq B \cup C$ then $A \subseteq B$.
\end{theorem}

\begin{proof}
Suppose $A \cap C \subseteq B \cap C$ and $A \cup C \subseteq B \cup C$. Let $x$ be arbitrary and suppose $x \in A$. Thus $x \in A \cup C$ and it follows that $x \in B \cup C$. Now if $x \in B \cup C$ then either  $x \in B$ or $x \in C$. If $x \in B$ then since $x$ was arbitrary we can conclude $A \subseteq B$. In the case that $x \in C$, then $x \in A \cap C$ and it follows that $x \in B \cap C$. Therefore $x \in C$ and $x \in B$. Thus, if $x \in A$ then $x \in B$ and since $x$ was arbitrary we can conclude $A \subseteq B$. 
\end{proof}

\section*{3.5.5}
Suppose $A$ and $B$ are sets.

\begin{theorem} If $A \triangle B \subseteq A$ then $B \subseteq A$.
\end{theorem}

\begin{proof}
Suppose $A \triangle B \subseteq A$. We will prove by contradiction. Let $x$ be arbitrary and suppose $x \in B$ and $x \notin A$. Since $x \in B$ and $x \notin A$ then $x \in A \triangle B$. Since $A \triangle B \subseteq A$, then $x \in A$. But this contradicts $x \notin A$. Therefore, if $x \in B$ then $x \in A$ and since $x$ was arbitrary we can conclude that $B \subseteq A$.
\end{proof}


\section*{3.5.6}
Suppose $A$, $B$, and $C$ are sets.
\begin{theorem} $A \cup C \subseteq B \cup C \iff A \setminus C \subseteq B \setminus C$.
\end{theorem}

\begin{proof}
$(\rightarrow)$ Suppose $A$, $B$, and $C$ are sets. Suppose $(A \cup C) \subseteq (B \cup C)$. Let $x$ be arbitrary and suppose $c \in A \setminus C$, which means $x \in A$ and $x \notin C$. Since $x \in A$, then $x \in A \cup C$ and therefore $x \in B \cup C$. This means $x \in B$ or $x \in C$ and since $x \notin C$, it must be that $x \in B$. Now since $x \in B$ and $x \notin C$ then $x \in B \setminus C$. Therefore, if $x \in A \setminus C$ then $x \in B \setminus C$ and since $x$ was arbitrary we can conclude if $A \cup C \subseteq B \cup C$ then $A \setminus C \subseteq B \setminus C$.

$(\leftarrow)$ Now suppose $A \setminus C \subseteq B \setminus C$. Let $x$ be arbitrary and suppose $x \in A \cup C$, which means $x \in A$ or $x \in C$. If $x \in C$ then $x \in B \cup C$ and since $x$ was arbitrary then $A \cup C \subseteq B \cup C$. In the case that $x \in A$, since $A \setminus C \subseteq B \setminus C$ then $x \in B$. Therefore, $x \in B \cup C$ and since $x$ was arbitrary then $A \cup C \subseteq B \cup C$.  
\end{proof}

\section*{3.5.7}
\begin{theorem} For any sets $A$ and $B$, $\pwset (A) \cup \pwset (B) \subseteq \pwset (A \cup B)$
\end{theorem}

\begin{proof}
Let $A$ and $B$ be arbitrary sets. Let $M$ be arbitrary and suppose $M \in \pwset (A) \cup \pwset (B)$. Thus $M \in \pwset (A)$ or $M \in \pwset (B)$, which means $M \subseteq A$ or $M \subseteq B$. In the case where $M \subseteq A$, let $x$ be an arbitrary member of $M$ and it follows that $x \in A$. Since $x \in A$ then $x \in A \cup B$ and because $x$ was arbitrary we can conclude $M \subseteq A \cup B$ and therefore $M \in \pwset (A \cup B)$. In the case where $M \subseteq B$, let $x$ be an arbitrary member of $M$ and it follows that $x \in B$. Since $x \in B$ then $x \in A \cup B$ and because $x$ was arbitrary we can conclude $M \subseteq A \cup B$ and therefore $M \in \pwset (A \cup B)$.
\end{proof}

\section*{3.5.8}
\begin{theorem} For any sets $A$ and $B$, if $\pwset (A) \cup \pwset (B) = \pwset (A \cup B)$ then either $A \subseteq B$ or $B \subseteq A$.
\end{theorem}

\begin{proof}

We will prove the contrapositive. Since we proved that $\pwset (A) \cup \pwset (B) \subseteq \pwset (A \cup B)$ in exercise 3.5.7, we must show that $\pwset (A \cup B) \not\subseteq \pwset (A) \cup \pwset (B)$ to prove our goal that $\pwset (A) \cup \pwset (B) \neq \pwset (A \cup B)$. Let $A$ and $B$ be arbitrary sets and suppose $A \not\subseteq B$ and $B \not\subseteq A$. This means there is an element $x \in A \setminus B$ and an element $y \in B \setminus A$. Since $x \in A$ and $y \in B$ then both $x$ and $y$ are in $A \cup B$ and therefore the set $\{ x, y \}$ is in $\pwset (A \cup B)$ but not in $\pwset (A)$ or $\pwset (B)$. Thus $\pwset (A \cup B) \not\subseteq \pwset (A) \cup \pwset (B)$.
\end{proof}

\section*{3.5.9}
\begin{theorem} Suppose $x$ and $y$ are real numbers and $x \neq 0$. Then $y + 1/x = 1 + y/x$ iff either $x = 1$ or $y = 1$.
\end{theorem}

\begin{proof}
$(\rightarrow)$ Suppose that $y + 1/x = 1 + y/x$. Now if $y = 1$ then we have proven our goal. So now assume $y \neq 1$ and $y + 1/x = 1 + y/x$, then it follows that $x = 1$.

$(\leftarrow)$ Now suppose $x = 1$ or $y = 1$. In the case that $x = 1$ we have

\begin{equation*}
y + \frac{1}{x} = y + \frac{1}{1} = y + 1 = 1 + \frac{y}{1} = 1 + \frac{y}{x}
\end{equation*}
In the case that $y = 1$ we have
\begin{equation*}
y + \frac{1}{x} = 1 + \frac{1}{x} = 1 + \frac{y}{x}
\end{equation*}
\end{proof}

\section*{3.5.10}
\begin{theorem} For every real number $x$, if $ |x - 3| > 3$ then $x^2 > 6x$.
\end{theorem}

\begin{proof}
Suppose that $x$ is an arbitrary real number and that $ | x - 3 | > 3 $. Then either $x - 3 \geq 0$ or $x - 3 < 0$. In the case that $x - 3 \geq 0$, then $ | x - 3 | = x - 3 $ and therefore $ | x - 3 | > 3 = x - 3 > 3$. Solving for $x$, we have $x > 6$ and then multiplying both sides by $x$ we have $x^2 > 6x$. In the case that $x - 3 < 0$, then $ | x - 3 | = 3 - x $ and therefore $3 - x > 3$. Solving for $x$ we have $x < 0$. Multiplying both sides of $ x < 0 $ by $ 6 - x $ we have $ 6x - x^2 < 0$ and therefore $x^2 > 6x$. 
\end{proof}

\section*{3.5.11}
\begin{theorem} For every real number x, $|2x - 6| > x$ iff $|x - 4| > 2$.
\end{theorem}

\begin{proof}
$(\rightarrow)$ Let $x$ be an arbitrary real number and suppose $| 2x - 6 | > x $. Our goal $| x - 4 | > 2$ means that either $x - 4 > 2$ or $4 - x > 2$. Since $ | 2x - 6 | > 2 $ then either $2x - 6 > x$ or $6 - 2x > x$. If $2x - 6 > x$ then it follows that $x - 4 > 2$. Now if $6 - 2x > x$ then if follows that $4 - x >2$. 

$(\leftarrow)$ Now suppose $| x - 4 | > 2$. Our goal $| 2x - 6 | > x$ means that either $2x - 6 > x$ or $6 - 2x > x$. Since $| x - 4 | > 2$ then either $x - 4 > 2$ or $4 - x > 2$. If $x - 4 > 2$ then it follows that $2x - 6 > x$. In the case that $4 - x > 2$ then it follows that $6 - 2x > x$.
\end{proof}

\section*{3.5.12}
\begin{theorem} For all real numbers $a$ and $b$, $| a | \leq b$ if and only if $-b \leq a \leq b$.
\end{theorem}

\begin{proof}
$(\rightarrow)$ Suppose $a$ and $b$ are arbitrary real numbers and that $| a | \leq b$. There are two cases to consider: $a \geq 0$ and $a < 0$. If $a \geq 0$ then $| a | = a \leq b$. It follows that $-b \leq -a$ and since $a \geq 0$ then $-a \leq a$. Therefore, $-b \leq -a \leq a \leq b$ and $-b \leq a \leq b$. Now in the case that $a < 0$ then $| a | = -a \leq b$. It follows that $-b \leq a$ and since $a < 0$ then $-a > a$ or $a < -a$. Therefore $-b \leq a < -a \leq b$ and $-b \leq a \leq b$.

$(\leftarrow)$ Now suppose $-b \leq a \leq b$ and therefore $a \leq b$. Now we must prove that $-a \leq b$ to complete the proof. If we subtract $a$ from both sides of $-b \leq a$ and add $b$ to both sides we have $-a \leq b$. 
\end{proof}

\section*{3.5.13}
\begin{theorem} For every integer $x$, $x^2 + x$ is even.
\end{theorem}

\begin{proof}
Let $x$ be an arbitrary integer. There are two cases to consider: $x$ is even or $x$ is odd. If $x$ is even then there exists an integer $k$ such that $x = 2k$. Plugging in $2k$ for $x$ in $x^2 + x$ we have $x^2 + x = (2k)^2 + 2k = 4k^2 + 2k = 2(2k^2 + k)$. Since $2k^2 + k$ is an integer then $x^2 + x$ is even. In the case that $x$ is odd there is a $j$ such that $x = 2j + 1$. Plugging in $2j +1$ for $x$ in $x^2 + x$ we have $x^2 + x = (2j + 1)^2 + (2j + 1) = (4j^2 + 4j + 1) + (2j + 1) = 4j^2 + 6j + 2 = 2(2j^2 + 3j + 1)$. Since $2j^2 + 3j + 1$ is an integer, $x^2 + x$ is even.
\end{proof}

\section*{3.5.14}
\begin{theorem} For every integer $x$, the remainder when $x^4$ is divided by $8$ is either $0$ or $1$.
\end{theorem}

\begin{proof}
Suppose $x$ is an integer and there exists an integer $k$ such that $8k = x^4$. Since $x$ is an integer, $x$ is either even or odd. If $x$ is even then there exists an integer $m$ such that $x=2m$. Then $8k = (2m)^4 = 16m^4$ and $k = 2m^4$ r $0$. In the case that $x$ is odd, then there exists an integer $m$ such that $x = 2m + 1$. Then $8k = (2m+1)^4 = 16x^4 + 32x^3 + 24x^2 + 8x + 1$ and $k = 2x^4 + 4x^3 + 3x^2 + x$ r $1$. Therefore, when $x^4$ is divided by 8 the remainder is either $0$ or $1$.
\end{proof}

\section*{3.5.15}
Suppose $\F$ and $\G$ are nonempty families of sets.

\begin{theorem} $\cup ( \F \cup \G ) = (\cup \F) \cup (\cup \G)$
\end{theorem}

\begin{proof}
$(\rightarrow)$ Suppose $x \in \cup(\F \cup \G)$, which means there is a set in $\F \cup \G$ that contains $x$. Thus the set that contains $x$ is in $\F$ or $\G$. If the set that contains $x$ is in $\F$ then $x \in \cup \F$ and $x \in (\cup \F) \cup (\cup \G)$. In the case that the set that contains $x$ is in $\G$, then $x \in \cup \G$ and $x \in (\cup \F) \cup (\cup \G)$ 


$(\leftarrow)$ Now suppose $x \in (\cup F) \cup (\cup G)$, which means there is a set in $\F$ that contains $x$ or a set in $\G$ that contains $x$. If there is a set in $\F$ that contains $x$, and this same set is in $\F \cup \G$. Thus there is a set in $\F \cup \G$ that contains $x$. In the case that there is a set in $\G$ that contains $x$, then this set is in $\F \cup \G$. Thus there is a set in $\F \cup \G$ that contains $x$. Therefore $x \in \cup (\F \cup \G)$. \\

\end{proof}

Alternate proof?
\begin{proof}
\begin{align*}
x \in \cup (\F \cup \G) ~ &\text{iff} \\
\exists M \in \F \cup \G (x \in M) ~ &\text{iff} \\
\exists M \in \F (x \in M) \lor \exists M \in \G (x \in M) ~ &\text{iff} \\
x \in \cup \F \lor x \in \cup \G ~ &\text{iff} \\
x \in (\cup \F) \cup (\cup \G)
\end{align*}
\end{proof}

\section*{3.5.16}
Suppose $\F$ is a nonempty family of sets and $B$ is a set.

\subsection*{A}
\begin{theorem} $B \cup (\cup \F) \subseteq \cup (\F \cup \{B\})$
\end{theorem}

\begin{proof}
$(\rightarrow)$ Suppose $x$ is arbitrary and $x \in B \cup (\cup \F)$. Then $x \in B$ or $x \in \cup \F$. If $x \in B$ then because $B \in \F \cup \{B\}$, it follows that $x \in \cup (\F \cup \{B\})$. In the case that $x \in \cup \F$, there is a set $M \in \F$ such that $x \in M$. Since $M \in \F$ then $M \in \F \cup \{B\}$ and therefore $x \in \cup (\F \cup \{B\})$.

$(\leftarrow)$ Now suppose $x \in \cup (\F \cup \{B\})$. Then there is a set $M$ such that $x \in M$ and $M \in (\F \cup \{B\})$, which means $M \in \F$ or $M \in \{B\}$. If $M \in \F$ then it follows that $x \in \cup \F$ and thus $x \in B \cup (\cup \F)$. In the case that $M \in \{B\}$ then it follows that $x \in B$ and thus $x \in B \cup (\cup \F)$.  
\end{proof}

\subsection*{B}
\begin{theorem} $B \cup (\cap \F) = \bigcap_{A \in \F} (B \cup A)$
\end{theorem}

\begin{proof}
$(\rightarrow)$ Let $x$ be arbitrary and suppose $x \in B \cup (\cap \F)$. Then $x \in B$ or $x \in \cap \F$. If $x \in B$, then $x \in B \cup A$ for any set $A$ and thus $x \in \bigcap_{A \in \F} (B \cup A)$. In the case that $x \in \cap \F$, then $x$ is in every set $A \in \F$ and so $x \in \bigcap_{A \in \F} A$. Therefore $x \in \bigcap_{A \in \F} (B \cup A)$. Since $x$ was arbitrary then $B \cup (\cap \F) \subseteq \bigcap_{A \in \F} (B \cup A)$.

$(\leftarrow)$ Now suppose $x \in \bigcap_{A \in \F} (B \cup A)$. Thus $x \in B$ or $x \in A$ for all $A \in \F$. If $x \in B$ then $x \in B \cup (\cap \F)$. If $x \in A$ for all $A \in \F$ then $x \in \cap \F$ and therefore $x \in B \cup (\cap \F)$. Since $x$ was arbitrary then $\bigcap_{A \in \F} (B \cup A) \subseteq B \cup (\cap \F)$.
\end{proof}

\subsection*{C}
\begin{theorem} $B \cap ( \cap \F ) = \bigcap_{A \in \F} ( B \cap A )$
\end{theorem}

\begin{proof}
$(\rightarrow)$ Let $x$ be arbitrary and suppose $x \in B \cap ( \cap \F )$, which means $x \in B$ and for all $A \in \F$, $x \in A$. Thus $x \in A \cap B$ and since $x \in A$ for all $A \in \F$, then $x \in \bigcap_{A \in \F} (A \cap B)$. Since $x$ was arbitrary, we conclude $B \cap ( \cap \F ) \subseteq \bigcap_{ A \in \F } ( B \cap A )$.

$(\leftarrow)$ Now suppose $x \in \bigcap_{A \in \F} (A \cap B)$, which means for all $A \in \F$, $x \in A \cap B$. Therefore $x \in B$ and for all $A \in \F$, $x \in A$ and thus $x \in \cap \F$. Since $x$ was arbitrary we conclude $\bigcap_{A \in \F} ( B \cap A ) \subseteq B \cap ( \cap \F )$.
\end{proof}

\section*{ 3.5.17 }

\begin{theorem}
Suppose $\F$, $\G$, and $\mathcal{H}$ are nonempty families of sets and for every $A \in \F$ and every $B \in \G$, $A \cup B \in \mathcal{H}$, then $\cap \mathcal{H}$ is a subset of $( \cap \F ) \cup ( \cap \G )$. 
\end{theorem}

\begin{proof}
Suppose $A$ and $B$ are arbitrary sets, $A \in \F$, $B \in \G$, and $A \cup B \in \mathcal{H}$. Let $x$ be arbitrary and suppose $x \in \cap \mathcal{H}$, which means $x$ is in every set in $\mathcal{H}$. Since $A \cup B \in \mathcal{H}$, it follows that $x \in A$ or $x \in B$. If $x \in A$, then since $A$ is an arbitrary set in $\F$, then $x \in \cap \F$. If $x \in B$, then since $B$ is an arbitrary set in $\G$, then $x \in \cap \G$. Therefore, $ x \in ( \cap \F ) \cup ( \cap \G )$ and since $x$ was arbitrary we conclude that $\cap \mathcal{H} \subseteq ( \cap \F ) \cup ( \cap \G )$.
\end{proof}

\section*{3.5.18}
\begin{theorem} Suppose $A$ and $B$ are sets. Then $\forall x (x \in A \triangle B \iff (x \in A \iff x \notin B))$
\end{theorem}

\begin{proof}
Let $x$ be arbitrary and suppose $x \in A \triangle B$. Then

\begin{align*}
x \in A \triangle B ~ &\text{iff} ~ x \in (A \cup B) \setminus (A \cap B)\\
&\text{iff} ~ (x \in A \cup B) \land x \notin (A \cap B)  \\
&\text{iff} ~ (x \in A \lor x \in B) \land (x \notin A \lor x \notin B) \\
&\text{iff} ~ (x \notin B \implies x \in A) \land (x \in A \implies x \notin B) \\
&\text{iff} ~ x \in A \iff x \notin B
\end{align*}
\end{proof}


\section*{3.5.19}
\begin{theorem} Suppose $A$, $B$, and $C$ are sets. Then $A \triangle B$ and $C$ are disjoint if and only if $A \cap C = B \cap C$.
\end{theorem}

\begin{proof}
$(\rightarrow)$ We will prove by contradiction. Suppose $(A \triangle B) \cap C = \varnothing$. Recall that $x \in A \triangle B$ means that $x \in A \setminus B$ or $x \in B \setminus A$. Now suppose $A \cap C \neq B \cap C$, which means $A \cap C \not\subseteq B \cap C$ or $B \cap C \not\subseteq A \cap C$. If $A \cap C \not\subseteq B \cap C$ then there exists an $x$ such that $x \in A \cap C$ and $x \notin B \cap C$. Thus $x \in C$ and $x \in A \setminus B$, which also means $x \in A \triangle B$. However this contradicts our assumption that $(A \triangle B) \cap C = \varnothing$. In the case that $B \cap C \not\subseteq A \cap C$, there exists an $x$ such that $x \in B \cap C$ and $x \notin A \cap C$. Thus $x \in C$ and $x \in B \setminus A$, which also means $x \in A \triangle B$. However this contradicts our assumption that $(A \triangle B) \cap C = \varnothing$.

$(\leftarrow)$ We will prove by contradiction. Suppose $A \cap C = B \cap C$. Now suppose that $(A \triangle B) \cap C \neq \varnothing$, which means there exists an $x$ such that $x \in (A \setminus B) \cap C$ or $x \in (B \setminus A) \cap C$. If $ x \in (A \setminus B) \cap C$, then $x \in A \setminus B$, which means $x \in A$ and $x \notin B$. Since $x \in A$ and $x \in C$, then $x \in A \cap C$. If follows that $x \in B \cap C$ because $A \cap C = B \cap C$, however this contradicts our assumption that $x \notin B$. In the case that $x \in (B \setminus A) \cap C$, $x \in B \setminus A$. Thus $x \in B$ and $x \notin A$. Since $x \in B$ and $x \in C$, then $x \in B \cap C$. It follows that $x \in A \cap C$ because $A \cap C = B \cap C$. However, this contradicts our assumption that $x \notin A$. 
\end{proof}

\section*{3.5.20}
\begin{theorem} Suppose $A$, $B$, and $C$ are sets. Then $A \triangle B \subseteq C$ if and only if $A \cup C = B \cup C$.
\end{theorem}

\begin{proof}
$(\rightarrow)$ Suppose $A \triangle B \subseteq C$. Let $x$ be arbitrary and suppose $x \in A \cup C$. Thus $x \in A$ or $x \in C$.  If $x \in C$ then $x \in B \cup C$. In the case that $x \in A$ and $x \notin C$, then it follows that $x \notin A \triangle B$ and thus $x \in A \cap B$. Since $x \in A \cap B$, $x \in B$ and thus $x \in B \cup C$. Now to prove the other direction suppose $x \in B \cup C$. Thus $x \in B$ or $x \in C$. If $x \in C$ then $x \in A \cup C$. In the case that $x \in B$ and $x \notin C$, then it follows that $x \notin A \triangle B$ and thus $x \in A \cap B$. Since $x \in A \cap B$, $x \in A$ and thus $x \in A \cup C$.

$(\leftarrow)$ Now suppose $A \cup C = B \cup C$ and $x \in A \triangle B$. By the definition of symmetrical difference, $x \in A \setminus B$ or $x \in B \setminus A$. If $x \in A \setminus B$ then $x \in A$ and $x \notin B$. It follows that $x \in A \cup C$ and therefore $x \in B \cup C$. Since $x \in B \cup C$ and $x \notin B$, then $x \in C$. In the case that $x \in B \setminus A$, $x \in B$ and $x \notin A$. It follows that $x \in B \cup C$ and therefore $x \in A \cup C$. Since $x \in A \cup C$ and $x \notin A$, then $x \in C$. Thus if $x \in A \triangle B$ then $x \in C$.
\end{proof}

\section*{3.5.21}
\begin{theorem} Suppose $A$, $B$, and $C$ are sets. Then $C \subseteq A \triangle B$ if and only if $C \subseteq A \cup B$ and $A \cap B \cap C = \varnothing$.
\end{theorem}

\begin{proof}
$(\rightarrow)$ Suppose $C \subseteq A \triangle B$. To show that $C \subseteq A \cup B$, let $x$ be arbitrary and suppose $x \in C$. Since $x \in C$ then $x \in A \triangle B$. By the definition of symmetric difference, $x \in A \cup B$ and $x \notin A \cap B$. Thus if $x \in C$ then $x \in A \cup B$. To show that $A \cap B \cap C = \varnothing$ we will used proof by contradiction. Suppose there is an element $y$ such that $y \in A$, $y \in B$, and $y \in C$. Since $y \in C$ then $y \in A \triangle B$. As noted earlier, if $y \in A \triangle B$ then $y \notin A \cap B$; however, this contradicts our assumption that $y \in A$ and $y \in B$. Thus, it must be that $A \cap B \cap C = \varnothing$.

$(\leftarrow)$ Now suppose $C \subseteq A \cup B$ and $A \cap B \cap C = \varnothing$. Let $x$ be arbitrary and suppose $x \in C$. It follows that $x \in A \cup B$. Since $x \in C$ and $A \cap B \cap C = \varnothing$, then $x \notin A \cap B$. Now since $x \in A \cup B$ and $x \notin A \cap B$, then $x \in A \triangle B$. 
\end{proof}


\section*{3.5.22}
Suppose $A$, $B$, and $C$ are sets.

\subsection*{A}
\begin{theorem} $A \setminus C \subseteq (A \setminus B) \cup (B \setminus C)$
\end{theorem}

\begin{proof}
Suppose $x$ is arbitrary and $x \in A \setminus C$, which means $x \in A$ and $x \notin C$. Now either $x \in B$ or $x \notin B$. If $x \in B$, then it follows that $x \in B \setminus C$ and thus $x \in (A \setminus B) \cup (B \setminus C)$. Therefore if $x \in A \setminus C$ then $x \in (A \setminus B) \cup (B \setminus C)$ and since $x$ was arbitrary we can conclude $A \setminus C \subseteq (A \setminus B) \cup (B \setminus C)$
\end{proof}

\subsection*{B}

\begin{theorem} $A \triangle C \subseteq (A \triangle B) \cup (B \triangle C)$
\end{theorem}

\begin{proof}
Suppose $x$ is arbitrary and $x \in A \triangle C$, which means $x \in A \setminus C$ or $x \in C \setminus A$. Also, either $x \in B$ or $x \notin B$. Thus, we have four cases to consider:

Case 1: $x \in A \setminus C$ and $x \in B$. Since $x \in A \setminus C$ then $x \in A$ and $x \notin C$. Therefore $x \in B$ and $x \notin C$ and by the definition of symmetric difference, $x \in B \triangle C$. Therefore if $x \in A \triangle C$ then $x \in (A \triangle B) \cup (B \triangle C)$ and since $x$ was arbitrary we can conclude $A \triangle C \subseteq (A \triangle C) \cup (B \triangle C)$.

Case 2: $x \in A \setminus C$ and $x \notin B$. Since $x \in A \setminus C$ then $x \in A$ and $x \notin C$. Therefore, $x \in A$ and $x \notin B$ and therefore $x \in A \triangle B$. Therefore, if $x \in A \triangle C$ then $x \in (A \triangle B) \cup (B \triangle C)$ and since $x$ was arbitrary we can conclude $A \triangle C \subseteq (A \triangle B) \cup (B \triangle C)$.

Case 3: $x \in C \setminus A$ and $x \in B$. Since $x \in C \setminus A$ then $x \in C$ and $x \in A$. Therefore $x \in B$ and $x \notin A$, and therefore $x \in A \triangle B$. Therefore, if $x \in A \triangle C$ then $x \in (A \triangle B) \cup (B \triangle C)$ and since $x$ was arbitrary we can conclude that $A \triangle C \subseteq (A \triangle B) \cup (B \triangle C)$.

Case 4: $x \in C \setminus A$ and $x \notin B$. Since $x \in C \setminus A$ then $x \in C$ and $x \notin A$. Therefore $x \in C \setminus B$ and thus $x \in B \triangle C$. Therefore, if $x \in A \triangle C$ then $x \in (A \triangle B) \cup (B \triangle C)$ and since $x$ was arbitrary we can conclude that $A \triangle C \subseteq (A \triangle B) \cup (B \triangle C)$. 
\end{proof}

\section*{3.5.23}

Suppose $A$, $B$, and $C$ are sets.

\subsection*{A}
\begin{theorem} $( A \cup B ) \triangle C \subseteq (A \triangle C) \cup (B \triangle C)$
\end{theorem}

\begin{proof}
Suppose $x$ is arbitrary and $x \in (A \cup B) \triangle C$. Thus $x \in A \setminus C$ or $x \in C \setminus (A \cup B)$, and we have 4 cases to consider:

Case 1: $x \in A$, $x \notin B$, and $x \notin C$. Since $x \in A$ and $x \notin C$ then $x \in A \setminus C$ and $x \in A \triangle C$. Therefore if $x \in (A \cup B) \triangle C$ then $x \in (A \triangle C) \cup (B \triangle C)$ and since $x$ was arbitrary we can conclude $(A \cup B) \triangle C \subseteq (A \triangle C) \cup (B \triangle C)$.

Case 2: $x \in B$, $x \notin A$, and $x \notin C$. Since $x \in B$ and $x \notin C$ then $x \in B \setminus C$ and $x \in B \triangle C$. Therefore if $x \in (A \cup B) \triangle C$ then $x \in (A \triangle C) \cup (B \triangle C)$ and since $x$ was arbitrary we can conclude $(A \cup B) \triangle C \subseteq (A \triangle C) \cup (B \triangle C)$.

Case 3: $x \in A$, $x \in B$, and $x \notin C$. Since $x \in B$ and $x \notin C$ then $x \in B \setminus C$ and $x \in B \triangle C$. Since $x \in A$ and $x \notin C$ then $x \in A \setminus C$ and $x \in A \triangle C$. Thus $x \in (A \triangle C) \cup (B \triangle C)$. Therefore if $x \in (A \cup B) \triangle C$ then $ x \in (A \triangle C) \cup (B \triangle C)$ and since $x$ was arbitrary we can conclude $(A \cup B) \triangle C \subseteq (A \triangle C) \cup (B \triangle C)$.

Case 4: $x \in C$, $x \notin A$, $x \notin B$. Since $x \in C$ and $x \notin A$ then $x \in C \setminus A$ and $x \in A \triangle C$. Since $x \in C$ and $x \notin B$ then $x \in C \setminus B$ and $x \in B \triangle C$. Thus $x \in (A \triangle C) \cup (B \triangle C)$. Therefore if $x \in (A \cup B) \triangle C$ then $x \in (A \triangle C) \cup (B \triangle C$ and since $x$ was arbitrary we can conclude $(A \cup B) \triangle C \subseteq (A \triangle C) \cup (B \triangle C)$.
\end{proof}


\subsection*{B}
Find an example of sets $A$, $B$, and $C$ such that $( A \cup B ) \triangle C \neq (A \triangle C) \cup (B \triangle C)$ \\

Let $A = \{1, 2\}$, $B = \{3, 4\}$ and $C = \{1, 5\}$. \\

Then $A \triangle C = (A \setminus C) \cup (C \setminus A) = \{2\} \cup \{5\} = \{2, 5\}$. \\

Now $B \triangle C = (B \setminus C) \cup (C \setminus B) = \{1, 3, 4, 5\} \cup \{1, 3, 4, 5\} = \{1, 3, 4, 5\}$. \\

Therefore $(A \triangle C) \cup (B \triangle C) = \{2, 5\} \cup \{1, 3, 4, 5\} = \{1, 2, 3, 4, 5\}$. \\

Now $(A \cup B) = \{1, 2\} \cup \{3, 4\} = \{1, 2, 3, 4\}$. \\

Therefore $(A \cup B) \triangle C = ((A \cup B) \setminus C) \cup (C \setminus (A \cup B)) = \{2, 3, 4\} \cup \{5\} = \{2, 3, 4, 5\}$. \\

Since $\{1, 2, 3, 4, 5\} \neq \{2, 3, 4, 5\}$ then $( A \cup B ) \triangle C \neq (A \triangle C) \cup (B \triangle C)$.


\section*{3.5.24}
Suppose $A$, $B$, and $C$ are sets.

\subsection*{A} 

\begin{theorem} $( A \triangle C) \cap (B \triangle C) \subseteq (A \cap B) \triangle C$.
\end{theorem}

\begin{proof}
Let $x$ be arbitrary and suppose $x \in (A \triangle C) \cap (B \triangle C)$. Then we have two cases to consider:

Case 1: $x \in A$, $x \in B$, and $x \notin C$. Since $x \in A$ and $x \in B$, then $x \in A \cap B$ and $x \notin C$. Therefore $ x \in (A \cap B) \setminus C$ and if $x \in (A \triangle C) \cap (B \triangle C)$ then $x \in (A \cap B) \triangle C$. Since $x$ was arbitrary we can conclude $(A \triangle C) \cap (B \triangle C) \subseteq (A \cap B) \triangle C$. 

Case 2: $x \notin A$, $x \notin B$, $x \in C$. Since $x \notin A$ and $x \notin B$, then $x \notin A \cap B$ and since $x \in C$ then $x \in C \setminus (A \cap B)$. Therefore if $x \in (A \triangle C) \cap (B \triangle C)$ then $x \in (A \cap B) \triangle C$ and since $x$ was arbitrary we can conclude $(A \triangle C) \cap (B \triangle C) \subseteq (A \cap B) \triangle C$.
\end{proof}

\subsection*{B}
\begin{theorem} $(A \cap B) \triangle C \subseteq (A \triangle C) \cap (B \triangle C)$
\end{theorem}

\begin{proof}
Let $x$ be arbitrary and suppose $x \in (A \cap B) \triangle C$. Thus $x \in A \cap B \setminus C$ or $x \in C \setminus A \cap B$.

Case 1: $x \in (A \cap B) \setminus C$. Thus $x \in A \cap B$ and $x \notin C$. Since $x \in A \cap B$ then $x \in A$ and $x \in B$. Since $x \in A$ and $x \notin C$ then $x \in A \setminus C$ and since $x \in B$ and $x \notin C$ then $x \in B \setminus C$. Therefore $x \in A \triangle C$ and $x \in B \triangle C$ and if $x \in (A \cap B) \setminus C$ then $x \in (A \triangle C) \cap (B \triangle C)$. Since $x$ was arbitrary we can conclude $(A \cap B) \triangle C \subseteq (A \triangle C) \cap (B \triangle C)$.

Case 2: $x \in C \setminus (A \cap B)$. Thus $x \in C$ and $x \notin A \cap B$. Since $x \notin A \cap B$ then $x \notin A$ and $x \notin B$. Since $x \in A$ and $x \notin C$ then $x \in A \setminus C$. Also since $x \in B$ and $x \notin C$ then $x \in B \setminus B$. Therefore $x \in A \triangle C$ and $x \in B \triangle C$ and if $x \in C \setminus (A \cap B)$ then $x \in (A \triangle C) \cap (B \triangle C)$. Since $x$ was arbitrary then we can conclude $(A \cap B) \triangle C \subseteq (A \triangle C) \cap (B \triangle C)$.
\end{proof}


\section*{3.5.25}
Suppose $A$, $B$, and $C$ are sets. Consider the sets $(A \setminus B) \triangle C$ and $(A \triangle C) \setminus (B \triangle C)$. Can you prove that either is a subset of the other? \\

To show that $(A \setminus B) \triangle C$ is not a subset of $(A \triangle C) \setminus (B \triangle C)$ consider the counterexample where $A = \{1,2\}$, $B = \{1,3\}$, and $C = \{3,4\}$. Then $A \setminus B = \{2\}$ and $(A \setminus B) \triangle C = \{3,4\}$. Also, $A \triangle C = \{1,2,3,4\}$, $B \triangle C = \{1,4\}$, and $(A \triangle C) \setminus (B \triangle C) = \{2,3\}$. Therefore $(A \setminus B) \triangle C = \{3, 4\} \nsubseteq \{2, 3\} = (A \triangle C) \setminus (B \triangle C)$. \\

We will show that $(A \triangle C) \setminus (B \triangle C) \subseteq (A \setminus B) \triangle C$.
\begin{proof}
Suppose $x$ is arbitrary and $x \in (A \triangle C) \setminus (B \triangle C)$, which means $x \in A \triangle C$ and $x \notin B \triangle C$. Consider the two cases, either $x \in A$, $x \notin B$, and $x \notin C$ or $x \notin A$, $x \in B$, and $x \in C$. If $x \in A$, $x \notin B$, and $x \notin C$, then $x \in A$ and $x \notin B$, thus $x \in A \setminus B$. Since $x \in A \setminus B$ and $x \notin C$ then $x \in (A \setminus B) \setminus C$ and therefore $x \in (A \setminus B) \triangle C$. If $x \notin A$, $x \in B$, and $x \in C$, then since $x \notin A$ and $x \in B$ then $x \notin A \setminus B$. Then since $x \in C$, we can conclude $x \in C \setminus (A \setminus B)$ and thus $x \in (A \setminus B) \triangle C$.

\end{proof}


\end{document}