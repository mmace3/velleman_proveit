\documentclass{article}

%\documentclass{amsart}
%\documentclass{scrartcl}
%\usepackage{changepage}
%\usepackage{scrextend}

\usepackage{amssymb,amsmath,amsthm}
% amssymb has empty set symbo
\usepackage{scrextend} % for \begin{addmargin}[0.55cm]{0cm} text \end{margin}

\usepackage{mathrsfs} % for \mathscr{P}
\usepackage{float}


\newcommand{\n}{ \noindent }
\newcommand{\F}{\mathcal{F}}
\newcommand{\G}{\mathcal{G}}
%\newcommand{\pwset}{\mathcal{P}}
\newcommand{\pwset}{\mathscr{P}}


\newtheorem*{theorem}{Theorem}  % This enables \begin{theorem}

%\DeclareFontFamily{U}{MnSymbolC}{}
%\DeclareSymbolFont{MnSyC}{U}{MnSymbolC}{m}{n}
%\DeclareFontShape{U}{MnSymbolC}{m}{n}{
%  <-6>    MnSymbolC5
%  <6-7>   MnSymbolC6
%  <7-8>   MnSymbolC7
%  <8-9>   MnSymbolC8
%  <9-10>  MnSymbolC9
%  <10-12> MnSymbolC10
%  <12->   MnSymbolC12%
%}{}
%\DeclareMathSymbol{\powerset}{\mathord}{MnSyC}{180}

%\usepackage[left=2cm,right=2cm,top=2cm,bottom=2cm]{geometry}

\begin{document}
\section*{3.5.1}
Suppose $A$, $B$, and $C$ are sets.

\begin{theorem} $A \cap ( B \cup C ) \subseteq ( A \cap B ) \cup C $
\end{theorem}
\begin{proof}
Let $x$ be arbitrary and suppose $x \in A \cap ( B \cup C )$. Thus $x \in A$ and $x \in B$ or $x \in C$. If $x \in C$ then $x \in ( A \cap B ) \cup C$. In the case where $x \in B$ it follows that $x \in A \cap B$ and therefore $x \in ( A \cap B ) \cup C$. Since $x$ was arbitrary we can conclude that $A \cap ( B \cup C ) \subseteq ( A \cap B ) \cup C $.
\end{proof}

\section*{3.5.2}
Suppose $A$, $B$, and $C$ are sets.

\begin{theorem} $(A \cup B) \setminus C \subseteq A \cup (B \setminus C)$
\end{theorem}
\begin{proof}
Let $x$ be arbitrary and suppose $x \in ( A \cup B ) \setminus C$. Thus $x \notin C$ and $x \in A$ or $x \in B$. If $x \in A$ then $x \in A \cup ( B \setminus C )$. If $x \in B$ then if follows that $x \in B \setminus C$ and therefore $x \in A \cup ( B \setminus C )$. Since $x$ was arbitrary we can conclude $A \cap ( B \cup C ) \subseteq ( A \cap B ) \cup C$.
\end{proof}

\section*{3.5.3}
Suppose $A$ and $B$ are sets.

\begin{theorem} $A \setminus ( A \setminus B ) = A \cap B$
\end{theorem}

\begin{proof}
Let $x$ be arbitrary and suppose $x \in A \setminus ( A \setminus B )$. Then
\begin{align*}
x \in A \setminus ( A \setminus B ) ~ &\text{iff} ~ x \in A \land x \notin A \setminus B \\
&\text{iff} ~  x \in A \land \lnot ( x \in A \land x \notin B ) \\
&\text{iff} ~ x \in A \land ( x \notin A \lor x \in B ) \\
&\text{iff} ~ ( x \in A \land x \notin A ) \lor ( x \in A \land x \in B ) \\
&\text{iff} ~ x \in A \land x \in B \\
&\text{iff} ~ x \in ( A \cap B )
\end{align*}
\end{proof}

\section*{3.5.4}
\begin{theorem} If $A \cap C \subseteq B \cap C$ and $A \cup C \subseteq B \cup C$ then $A \subseteq B$.
\end{theorem}

\begin{proof}
Suppose $A \cap C \subseteq B \cap C$ and $A \cup C \subseteq B \cup C$. Let $x$ be arbitrary and suppose $x \in A$. Thus $x \in A \cup C$ and it follows that $x \in B \cup C$. Now if $x \in B \cup C$ then either  $x \in B$ or $x \in C$. If $x \in B$ then since $x$ was arbitrary we can conclude $A \subseteq B$. In the case that $x \in C$, then $x \in A \cap C$ and it follows that $x \in B \cap C$. Therefore $x \in C$ and $x \in B$. Thus, if $x \in A$ then $x \in B$ and since $x$ was arbitrary we can conclude $A \subseteq B$. 
\end{proof}

\section*{3.5.5}
Suppose $A$ and $B$ are sets.

\begin{theorem} If $A \triangle B \subseteq A$ then $B \subseteq A$.
\end{theorem}

\begin{proof}
Suppose $A \triangle B \subseteq A$. We will prove by contradiction. Let $x$ be arbitrary and suppose $x \in B$ and $x \notin A$. Since $x \in B$ and $x \notin A$ then $x \in A \triangle B$. Since $A \triangle B \subseteq A$, then $x \in A$. But this contradicts $x \notin A$. Therefore, if $x \in B$ then $x \in A$ and since $x$ was arbitrary we can conclude that $B \subseteq A$.
\end{proof}

\end{document}