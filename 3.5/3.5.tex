\documentclass{article}

%\documentclass{amsart}
%\documentclass{scrartcl}
%\usepackage{changepage}
%\usepackage{scrextend}

\usepackage{amssymb,amsmath,amsthm}
% amssymb has empty set symbo
\usepackage{scrextend} % for \begin{addmargin}[0.55cm]{0cm} text \end{margin}

\usepackage{mathrsfs} % for \mathscr{P}
\usepackage{float}


\newcommand{\n}{ \noindent }
\newcommand{\F}{\mathcal{F}}
\newcommand{\G}{\mathcal{G}}
%\newcommand{\pwset}{\mathcal{P}}
\newcommand{\pwset}{\mathscr{P}}


\newtheorem*{theorem}{Theorem}  % This enables \begin{theorem}

%\DeclareFontFamily{U}{MnSymbolC}{}
%\DeclareSymbolFont{MnSyC}{U}{MnSymbolC}{m}{n}
%\DeclareFontShape{U}{MnSymbolC}{m}{n}{
%  <-6>    MnSymbolC5
%  <6-7>   MnSymbolC6
%  <7-8>   MnSymbolC7
%  <8-9>   MnSymbolC8
%  <9-10>  MnSymbolC9
%  <10-12> MnSymbolC10
%  <12->   MnSymbolC12%
%}{}
%\DeclareMathSymbol{\powerset}{\mathord}{MnSyC}{180}

%\usepackage[left=2cm,right=2cm,top=2cm,bottom=2cm]{geometry}

\begin{document}
\section*{3.5.1}
Suppose $A$, $B$, and $C$ are sets.

\begin{theorem} $A \cap ( B \cup C ) \subseteq ( A \cap B ) \cup C $
\end{theorem}
\begin{proof}
Let $x$ be arbitrary and suppose $x \in A \cap ( B \cup C )$. Thus $x \in A$ and $x \in B$ or $x \in C$. If $x \in C$ then $x \in ( A \cap B ) \cup C$. In the case where $x \in B$ it follows that $x \in A \cap B$ and therefore $x \in ( A \cap B ) \cup C$. Since $x$ was arbitrary we can conclude that $A \cap ( B \cup C ) \subseteq ( A \cap B ) \cup C $.
\end{proof}

\section*{3.5.2}
Suppose $A$, $B$, and $C$ are sets.

\begin{theorem} $(A \cup B) \setminus C \subseteq A \cup (B \setminus C)$
\end{theorem}
\begin{proof}
Let $x$ be arbitrary and suppose $x \in ( A \cup B ) \setminus C$. Thus $x \notin C$ and $x \in A$ or $x \in B$. If $x \in A$ then $x \in A \cup ( B \setminus C )$. If $x \in B$ then if follows that $x \in B \setminus C$ and therefore $x \in A \cup ( B \setminus C )$. Since $x$ was arbitrary we can conclude $A \cap ( B \cup C ) \subseteq ( A \cap B ) \cup C$.
\end{proof}

\section*{3.5.3}
Suppose $A$ and $B$ are sets.

\begin{theorem} $A \setminus ( A \setminus B ) = A \cap B$
\end{theorem}

\begin{proof}
Let $x$ be arbitrary and suppose $x \in A \setminus ( A \setminus B )$. Then
\begin{align*}
x \in A \setminus ( A \setminus B ) ~ &\text{iff} ~ x \in A \land x \notin A \setminus B \\
&\text{iff} ~  x \in A \land \lnot ( x \in A \land x \notin B ) \\
&\text{iff} ~ x \in A \land ( x \notin A \lor x \in B ) \\
&\text{iff} ~ ( x \in A \land x \notin A ) \lor ( x \in A \land x \in B ) \\
&\text{iff} ~ x \in A \land x \in B \\
&\text{iff} ~ x \in ( A \cap B )
\end{align*}
\end{proof}

\section*{3.5.4}
\begin{theorem} If $A \cap C \subseteq B \cap C$ and $A \cup C \subseteq B \cup C$ then $A \subseteq B$.
\end{theorem}

\begin{proof}
Suppose $A \cap C \subseteq B \cap C$ and $A \cup C \subseteq B \cup C$. Let $x$ be arbitrary and suppose $x \in A$. Thus $x \in A \cup C$ and it follows that $x \in B \cup C$. Now if $x \in B \cup C$ then either  $x \in B$ or $x \in C$. If $x \in B$ then since $x$ was arbitrary we can conclude $A \subseteq B$. In the case that $x \in C$, then $x \in A \cap C$ and it follows that $x \in B \cap C$. Therefore $x \in C$ and $x \in B$. Thus, if $x \in A$ then $x \in B$ and since $x$ was arbitrary we can conclude $A \subseteq B$. 
\end{proof}

\section*{3.5.5}
Suppose $A$ and $B$ are sets.

\begin{theorem} If $A \triangle B \subseteq A$ then $B \subseteq A$.
\end{theorem}

\begin{proof}
Suppose $A \triangle B \subseteq A$. We will prove by contradiction. Let $x$ be arbitrary and suppose $x \in B$ and $x \notin A$. Since $x \in B$ and $x \notin A$ then $x \in A \triangle B$. Since $A \triangle B \subseteq A$, then $x \in A$. But this contradicts $x \notin A$. Therefore, if $x \in B$ then $x \in A$ and since $x$ was arbitrary we can conclude that $B \subseteq A$.
\end{proof}


\section*{3.5.6}
Suppose $A$, $B$, and $C$ are sets.
\begin{theorem} $A \cup C \subseteq B \cup C \iff A \setminus C \subseteq B \setminus C$.
\end{theorem}

\begin{proof}
$(\rightarrow)$ Suppose $A$, $B$, and $C$ are sets. Suppose $(A \cup C) \subseteq (B \cup C)$. Let $x$ be arbitrary and suppose $c \in A \setminus C$, which means $x \in A$ and $x \notin C$. Since $x \in A$, then $x \in A \cup C$ and therefore $x \in B \cup C$. This means $x \in B$ or $x \in C$ and since $x \notin C$, it must be that $x \in B$. Now since $x \in B$ and $x \notin C$ then $x \in B \setminus C$. Therefore, if $x \in A \setminus C$ then $x \in B \setminus C$ and since $x$ was arbitrary we can conclude if $A \cup C \subseteq B \cup C$ then $A \setminus C \subseteq B \setminus C$.

$(\leftarrow)$ Now suppose $A \setminus C \subseteq B \setminus C$. Let $x$ be arbitrary and suppose $x \in A \cup C$, which means $x \in A$ or $x \in C$. If $x \in C$ then $x \in B \cup C$ and since $x$ was arbitrary then $A \cup C \subseteq B \cup C$. In the case that $x \in A$, since $A \setminus C \subseteq B \setminus C$ then $x \in B$. Therefore, $x \in B \cup C$ and since $x$ was arbitrary then $A \cup C \subseteq B \cup C$.  
\end{proof}

\section*{3.5.7}
\begin{theorem} For any sets $A$ and $B$, $\pwset (A) \cup \pwset (B) \subseteq \pwset (A \cup B)$
\end{theorem}

\begin{proof}
Let $A$ and $B$ be arbitrary sets. Let $M$ be arbitrary and suppose $M \in \pwset (A) \cup \pwset (B)$. Thus $M \in \pwset (A)$ or $M \in \pwset (B)$, which means $M \subseteq A$ or $M \subseteq B$. In the case where $M \subseteq A$, let $x$ be an arbitrary member of $M$ and it follows that $x \in A$. Since $x \in A$ then $x \in A \cup B$ and because $x$ was arbitrary we can conclude $M \subseteq A \cup B$ and therefore $M \in \pwset (A \cup B)$. In the case where $M \subseteq B$, let $x$ be an arbitrary member of $M$ and it follows that $x \in B$. Since $x \in B$ then $x \in A \cup B$ and because $x$ was arbitrary we can conclude $M \subseteq A \cup B$ and therefore $M \in \pwset (A \cup B)$.
\end{proof}

\section*{3.5.8}
\begin{theorem} For any sets $A$ and $B$, if $\pwset (A) \cup \pwset (B) = \pwset (A \cup B)$ then either $A \subseteq B$ or $B \subseteq A$.
\end{theorem}

\begin{proof}

We will prove the contrapositive. Since we proved that $\pwset (A) \cup \pwset (B) \subseteq \pwset (A \cup B)$ in exercise 3.5.7, we must show that $\pwset (A \cup B) \not\subseteq \pwset (A) \cup \pwset (B)$ to prove our goal that $\pwset (A) \cup \pwset (B) \neq \pwset (A \cup B)$. Let $A$ and $B$ be arbitrary sets and suppose $A \not\subseteq B$ and $B \not\subseteq A$. This means there is an element $x \in A \setminus B$ and an element $y \in B \setminus A$. Since $x \in A$ and $y \in B$ then both $x$ and $y$ are in $A \cup B$ and therefore the set $\{ x, y \}$ is in $\pwset (A \cup B)$ but not in $\pwset (A)$ or $\pwset (B)$. Thus $\pwset (A \cup B) \not\subseteq \pwset (A) \cup \pwset (B)$.
\end{proof}

\section*{3.5.9}
\begin{theorem} Suppose $x$ and $y$ are real numbers and $x \neq 0$. Then $y + 1/x = 1 + y/x$ iff either $x = 1$ or $y = 1$.
\end{theorem}

\begin{proof}
$(\rightarrow)$ Suppose that $y + 1/x = 1 + y/x$. Now if $y = 1$ then we have proven our goal. So now assume $y \neq 1$ and $y + 1/x = 1 + y/x$, then it follows that $x = 1$.

$(\leftarrow)$ Now suppose $x = 1$ or $y = 1$. In the case that $x = 1$ we have

\begin{equation*}
y + \frac{1}{x} = y + \frac{1}{1} = y + 1 = 1 + \frac{y}{1} = 1 + \frac{y}{x}
\end{equation*}
In the case that $y = 1$ we have
\begin{equation*}
y + \frac{1}{x} = 1 + \frac{1}{x} = 1 + \frac{y}{x}
\end{equation*}
\end{proof}

\section*{3.5.10}
\begin{theorem} For every real number $x$, if $ |x - 3| > 3$ then $x^2 > 6x$.
\end{theorem}

\begin{proof}
Suppose that $x$ is an arbitrary real number and that $ | x - 3 | > 3 $. Then either $x - 3 \geq 0$ or $x - 3 < 0$. In the case that $x - 3 \geq 0$, then $ | x - 3 | = x - 3 $ and therefore $ | x - 3 | > 3 = x - 3 > 3$. Solving for $x$, we have $x > 6$ and then multiplying both sides by $x$ we have $x^2 > 6x$. In the case that $x - 3 < 0$, then $ | x - 3 | = 3 - x $ and therefore $3 - x > 3$. Solving for $x$ we have $x < 0$. Multiplying both sides of $ x < 0 $ by $ 6 - x $ we have $ 6x - x^2 < 0$ and therefore $x^2 > 6x$. 
\end{proof}

\section*{3.5.11}
\begin{theorem} For every real number x, $|2x - 6| > x$ iff $|x - 4| > 2$.
\end{theorem}

\begin{proof}
$(\rightarrow)$ Let $x$ be an arbitrary real number and suppose $| 2x - 6 | > x $. Our goal $| x - 4 | > 2$ means that either $x - 4 > 2$ or $4 - x > 2$. Since $ | 2x - 6 | > 2 $ then either $2x - 6 > x$ or $6 - 2x > x$. If $2x - 6 > x$ then it follows that $x - 4 > 2$. Now if $6 - 2x > x$ then if follows that $4 - x >2$. 

$(\leftarrow)$ Now suppose $| x - 4 | > 2$. Our goal $| 2x - 6 | > x$ means that either $2x - 6 > x$ or $6 - 2x > x$. Since $| x - 4 | > 2$ then either $x - 4 > 2$ or $4 - x > 2$. If $x - 4 > 2$ then it follows that $2x - 6 > x$. In the case that $4 - x > 2$ then it follows that $6 - 2x > x$.
\end{proof}

\section*{3.5.12}
\begin{theorem} For all real numbers $a$ and $b$, $| a | \leq b$ if and only if $-b \leq a \leq b$.
\end{theorem}

\begin{proof}
$(\rightarrow)$ Suppose $a$ and $b$ are arbitrary real numbers and that $| a | \leq b$. There are two cases to consider: $a \geq 0$ and $a < 0$. If $a \geq 0$ then $| a | = a \leq b$. It follows that $-b \leq -a$ and since $a \geq 0$ then $-a \leq a$. Therefore, $-b \leq -a \leq a \leq b$ and $-b \leq a \leq b$. Now in the case that $a < 0$ then $| a | = -a \leq b$. It follows that $-b \leq a$ and since $a < 0$ then $-a > a$ or $a < -a$. Therefore $-b \leq a < -a \leq b$ and $-b \leq a \leq b$.

$(\leftarrow)$ Now suppose $-b \leq a \leq b$ and therefore $a \leq b$. Now we must prove that $-a \leq b$ to complete the proof. If we subtract $a$ from both sides of $-b \leq a$ and add $b$ to both sides we have $-a \leq b$. 

\end{proof}


\end{document}