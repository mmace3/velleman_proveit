\documentclass{article}

%\documentclass{amsart}
%\documentclass{scrartcl}
%\usepackage{changepage}
%\usepackage{scrextend}

\usepackage{amssymb,amsmath,amsthm}
% amssymb has empty set symbo
\usepackage{scrextend} % for \begin{addmargin}[0.55cm]{0cm} text \end{margin}

\usepackage{mathrsfs} % for \mathscr{P}
\usepackage{float}


\newcommand{\n}{ \noindent }
\newcommand{\F}{\mathcal{F}}
\newcommand{\G}{\mathcal{G}}
%\newcommand{\pwset}{\mathcal{P}}
\newcommand{\pwset}{\mathscr{P}}


\newtheorem*{theorem}{Theorem}  % This enables \begin{theorem}

%\DeclareFontFamily{U}{MnSymbolC}{}
%\DeclareSymbolFont{MnSyC}{U}{MnSymbolC}{m}{n}
%\DeclareFontShape{U}{MnSymbolC}{m}{n}{
%  <-6>    MnSymbolC5
%  <6-7>   MnSymbolC6
%  <7-8>   MnSymbolC7
%  <8-9>   MnSymbolC8
%  <9-10>  MnSymbolC9
%  <10-12> MnSymbolC10
%  <12->   MnSymbolC12%
%}{}
%\DeclareMathSymbol{\powerset}{\mathord}{MnSyC}{180}

%\usepackage[left=2cm,right=2cm,top=2cm,bottom=2cm]{geometry}

\begin{document}
\section*{3.5.1}
Suppose $A$, $B$, and $C$ are sets.

\begin{theorem} $A \cap ( B \cup C ) \subseteq ( A \cap B ) \cup C $
\end{theorem}
\begin{proof}
Let $x$ be arbitrary and suppose $x \in A \cap ( B \cup C )$. Thus $x \in A$ and $x \in B$ or $x \in C$. If $x \in C$ then $x \in ( A \cap B ) \cup C$. In the case where $x \in B$ it follows that $x \in A \cap B$ and therefore $x \in ( A \cap B ) \cup C$. Since $x$ was arbitrary we can conclude that $A \cap ( B \cup C ) \subseteq ( A \cap B ) \cup C $.
\end{proof}


\end{document}