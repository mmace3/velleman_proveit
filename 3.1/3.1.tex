\documentclass{article}

%\documentclass{scrartcl}
%\usepackage{changepage}
%\usepackage{scrextend}
\usepackage{amssymb,amsmath,amsthm}
\usepackage{graphicx}

\usepackage{scrextend} % for \begin{addmargin}[0.55cm]{0cm} text \end{margin}

\usepackage{mathrsfs} % for \mathscr{P}

\usepackage{amssymb} % empty set symbol

\newcommand{\n}{ \noindent }
\newcommand{\F}{\mathcal{F}}
\newcommand{\G}{\mathcal{G}}
%\newcommand{\pwset}{\mathcal{P}}
\newcommand{\pwset}{\mathscr{P}}
\newtheorem*{theorem}{Theorem}  % This enables \begin{theorem}
%\DeclareFontFamily{U}{MnSymbolC}{}
%\DeclareSymbolFont{MnSyC}{U}{MnSymbolC}{m}{n}
%\DeclareFontShape{U}{MnSymbolC}{m}{n}{
%  <-6>    MnSymbolC5
%  <6-7>   MnSymbolC6
%  <7-8>   MnSymbolC7
%  <8-9>   MnSymbolC8
%  <9-10>  MnSymbolC9
%  <10-12> MnSymbolC10
%  <12->   MnSymbolC12%
%}{}
%\DeclareMathSymbol{\powerset}{\mathord}{MnSyC}{180}

\begin{document}
\section*{Exercise 3.1.5}
Suppose $a$ and $b$ are real numbers. Prove that if $a<b<0$ then $a^2 > b^2$.\\

\n So we want to prove that 
$(a<b<0) \rightarrow (a^2 > b^2)$ \\

\n First we assume the antecedent and make the consequent our goal to prove.

\begin{table}[h]
\begin{tabular}{ll}
\hline
Givens & Goals   \\ \hline
$a<b<0$ & $a^2 > b^2$   \\ \hline
\end{tabular}
\end{table}

\n Suppose $a<b<0$ \\
\indent [proof of $a^2 > b^2$ ] \\
\n So if $a<b<0$ then $a^2 > b^2$ \\

\n If we multiply the inequality $a<b$ on both sides by the negative number $b$ we have $ab>b^2$ and multiplying $a<b$ on both sides by the negative number $a$ we have $a^2>ab$. Therefore $a^2 > ab > b^2$ and we have proven our goal and now we can write our proof.

\begin{theorem} Suppose $a$ and $b$  are real numbers. Prove that if $a<b<0$ then $a^2 > b^2$. 
\end{theorem}
\begin{proof}
Suppose $a<b<0$. Multiplying the inequality $a<b$ by the negative number $a$ we can conclude $a^2 > ab$, and, similarly, multiplying $a<b$ by the negative number $b$ we get $ab>b^2$. Therefore, $a^2>ab>b^2$ and $a^2>b^2$. Thus, if $a<b<0$ then $a^2>b^2$.
\end{proof}


\section*{Exercise 3.1.6}
Suppose $a$ and $b$ are real numbers. Prove that if $0<a<b$ then $1/b < 1/a$. \\

\n So we want to prove that 
$(0<a<b) \rightarrow (1/b < 1/a)$. \\

\n First we assume the antecedent and make the consequent our goal to prove.

\begin{table}[h]
\begin{tabular}{ll}
\hline
Givens & Goals   \\ \hline
$0<a<b$ & $1/b < 1/a$   \\ \hline
\end{tabular}
\end{table}

\n Suppose $0<a<b$ \\
\indent [proof of $1/b < 1/a$] \\
\n So if $0<a<b$ then $1/b < 1/a$. \\

\n If we multiply both sides of the inequality $a<b$ by $1/ab$ we see that $1/a < 1/b$, which is our goal.

\begin{theorem} Suppose $a$ and $b$ are real numbers. If $0<a<b$ then $1/b < 1/a$.
\end{theorem}
\begin{proof}
Suppose $0<a<b$. Multiplying both sides of the inequality $a<b$ by $1/ab$ we can conclude that $1/b < 1/a$. Therefore, if $0<a<b$ then $1/b < 1/a$.
\end{proof}

\section*{Exercise 3.1.7}
Suppose that $a$ is a real number. Prove that if $a^3 > a$ then $a^5 > a$. (Hint: One approach is to start by completing the following equation: $a^5 - a = (a^3 - 1) \cdot \text{\underline{?}}$.)

\n So we want to prove that
$(a^3) \rightarrow (a^5)$. \\

\n First we assume the antecedent and make the consequent our goal to prove.

\begin{table}[h]
\begin{tabular}{ll}
\hline
Givens & Goals   \\ \hline
$a^3 > a$ & $a^5$   \\ \hline
\end{tabular}
\end{table}

\n Suppose $a^3 > a$ \\
\indent [proof of $a^5 > a$] \\
\n So if $a^3 > a$ then $a^5 > a$. \\

\n If we multiply both side of $a^3 - a > 0$ by $a^2+1$ we can conclude that $a^5-a>0$ or $a^5>a$, which was our goal.

\begin{theorem} Suppose a is a real number. If $a^3>a$ then $a^5>a$.
\end{theorem}

\begin{proof}
Suppose $a^3>a$, then $a^3 - a > 0$. Multiplying both sides of the inequality $a^3-a>0$ by $a^2+1$ we can conclude $a^5-a>0$. Therefore, if $a^3>a$ then $a^5>a$.
\end{proof}

\section*{Exercise 3.1.8}
Suppose $A \setminus B \subseteq C \cap D$ and $x \in A$. Prove that if $x \notin D$ then $x \in B$.

\n So we want to prove that 
$(x \notin D) \rightarrow (x \in B)$. \\

\n The contrapositive of the goal is $\neg(x \in B) \rightarrow \neg(x \notin D)$, or in other words $x \notin B \rightarrow x \in D$. First we assume the antecedent and make the consequent our goal.

\begin{table}[h]
\begin{tabular}{ll}
\hline
Givens & Goals   \\ \hline
$A \setminus B \subseteq C \cap D$ & $x \in D$ \\
$x \in A$ & \\
$x \notin B$ &   \\ \hline
\end{tabular}
\end{table}

\n Looking at our givens we can rewrite $A \setminus B \subseteq C \cap D$ as $(x \in A \wedge x \notin B) \rightarrow (x \in A \wedge x \in D)$. Looking at our other givens $x \in A$ and $x \notin B$ we can conclude that $x \in C \cap D$ and therefore $x \in D$, which was our goal to prove.

\begin{theorem} Suppose $A \setminus B \subseteq C \cap D$ and $x \in A$. If $x \notin D$ then $x \in B$.
\end{theorem}
\begin{proof}
We will prove the contrapositive. Suppose $x \notin B$. Since $x \in A$ and $x \notin B$ we can conclude that $x \in C \cap D$ and it follows that $x \in D$. Therefore, if $x\notin D$ then $x \in B$.
\end{proof}

\section*{Exercise 3.1.9}
Suppose $a$ and $b$ are real numbers. Prove that if $a<b$ then $\tfrac{a+b}{2} < b$. \\

\n So we want to prove that
$(a<b) \rightarrow (\tfrac{a+b}{2} < b$). \\

\n First we assume the antecedent and make the consequent our goal to prove.

\begin{table}[h]
\begin{tabular}{ll}
\hline
Givens & Goals   \\ \hline
$a<b$ & $\tfrac{a+b}{2} < b$   \\ \hline
\end{tabular}
\end{table}

\n If we add $b$ to both sides of $a<b$ we see that $a + b < b + b$ or $a + b < 2b$. Then if we divide both sides of $a + b < 2b$ by $2$ we can conclude that $\tfrac{a+b}{2} < b$, which was our goal to prove.

\begin{theorem} Suppose $a$ and $b$ are real numbers. If $a<b$ then $\tfrac{a + b}{2}$.
\end{theorem}
\begin{proof}
Suppose $a<b$. Adding $b$ to both sides of the inequality $a<b$ and then dividing both sides by $2$, we can conclude that $\tfrac{a+b}{2} < b$. Therefore, if $a<b$ then $\tfrac{a+b}{2} < b$.
\end{proof}

\section*{Exercise 3.1.10}
Suppose $x$ is a real number and $x \neq 0$. Prove that if $\tfrac{\sqrt[\leftroot{-1}\uproot{1}3]{x} + 5}{x^2 + 6} = \tfrac{1}{x}$ then $x \neq 8$. \\

\n So we want to prove that
$\left(\tfrac{\sqrt[\leftroot{-1}\uproot{1}3]{x} + 5}{x^2 + 6} = \tfrac{1}{x} \right) \rightarrow (x \neq 8)$. \\

\n The contrapositive of the goal is $\neg(x \neq 8) \rightarrow \neg(\tfrac{\sqrt[\leftroot{-1}\uproot{1}3]{x} + 5}{x^2 + 6} = \tfrac{1}{x})$ or in other words $(x = 8) \rightarrow (\tfrac{\sqrt[\leftroot{-1}\uproot{1}3]{x} + 5}{x^2 + 6} \neq \tfrac{1}{x})$. First we assume the antecedent and make the consequent our goal.

\begin{table}[h]
\begin{tabular}{ll}
\hline
Givens & Goals  \\ \hline
$x \neq 0$ & $\tfrac{\sqrt[\leftroot{-1}\uproot{1}3]{x} + 5}{x^2 + 6} \neq \tfrac{1}{x}$ \\
$x = 8$ & \\ \hline
\end{tabular}
\end{table}

\n If we evaluate the expression $\tfrac{\sqrt[\leftroot{-1}\uproot{1}3]{x} + 5}{x^2 + 6}$ for $x = 8$ we see that $\tfrac{\sqrt[\leftroot{-1}\uproot{1}3]{8} + 5}{8^2 + 6} = \tfrac{1}{7}$ and $\tfrac{1}{7} \neq \tfrac{1}{8}$, therefore $\tfrac{\sqrt[\leftroot{-1}\uproot{1}3]{x} + 5}{x^2 + 6} \neq \tfrac{1}{x}$, which was our goal to prove.

\begin{theorem} Suppose $x$ is a real number and $x \neq 0$. If $\tfrac{\sqrt[\leftroot{-1}\uproot{1}3]{x} + 5}{x^2 + 6} = \tfrac{1}{x}$ then $x \neq 8$
\end{theorem}
\begin{proof}
We will prove the contrapositive. Suppose $x = 8$. Then evaluating the equation $\tfrac{\sqrt[\leftroot{-1}\uproot{1}3]{x} + 5}{x^2 + 6} = \tfrac{1}{x}$ for $x = 8$ we see that $\tfrac{1}{7} \neq \tfrac{1}{8}$ and therefore if $\tfrac{\sqrt[\leftroot{-1}\uproot{1}3]{x} + 5}{x^2 + 6} = \tfrac{1}{x}$ then $x \neq 8$.
\end{proof}


\section*{Exercise 3.1.11}
Suppose $a$, $b$, $c$, and $d$ are real numbers, $0<a<b$, and $d>0$. Prove that if $ac \geq bd$ then $c>d$. \\

\n So we want to prove that
$(ac \geq bd) \rightarrow (c>d)$ \\

\n The contrapositive of the goal is $\neg(c>d) \rightarrow \neg(ac \geq bd)$ or in other words $(c \leq d) \rightarrow (ac < bd)$. First we assume the antecedent and make the consequent our goal.

\begin{table}[h]
\begin{tabular}{ll}
\hline
Givens & Goals   \\ \hline
$c \leq d$ & $ac < bd$   \\ \hline
\end{tabular}
\end{table}

\n If we multiply both side of the inequality $c \leq d$ by $a$ we see that $ac \leq ad$ and multiplying both sides of the inequality $a < b$ by $d$ we see that $ad < bd$. Therefore, $ac \leq ad < bd$ and $ac < ad$, which was our goal to prove.

\begin{theorem} Suppose $a$, $b$, $c$, and $d$ are real numbers, $0<a<b$ and $d>0$. If $ac \geq bd$ then $c>d$.
\end{theorem}
\begin{proof}
We will prove the contrapositive. Suppose $c \leq d$. Multiplying the inequality $c \leq d$ on both sides by $a$ we have $ac \leq ad$ and multiplying the inequality $a<b$ on both sides by $d$ we have $ad < bd$. It follows that $ac \leq ad < bd$ and $ac < bd$. Therefore, if $ac \geq bd$ then $c>d$.
\end{proof}

\section*{Exercise 3.1.12}
Suppose $x$ and $y$ are real numbers, and $3x + 2y \leq 5$. Prove that if $x>1$ then $y<1$.

\n So we want to prove that
$(x>1) \rightarrow (y<1)$ \\

\n First we assume the antecedent and make the consequent our goal.

\begin{table}[h]
\begin{tabular}{ll}
\hline
Givens & Goals   \\ \hline
$x > 1$ & $y<1$   \\ \hline
\end{tabular}
\end{table}

\n Rearranging the inequality $3x + 2y \leq 5$ we see that $\tfrac{5-2y}{3} > x$. Our given is $x>1$ and so we can conclude that $\tfrac{5-2y}{3} > x > 1$ and $\tfrac{5-2y}{3} > 1$. Solving the latter inequality for $y$ we have $y < 1$, which was our goal to prove.

\begin{theorem} Suppose $x$ and $y$ are real numbers and $3x + 2y \leq 5$. If $x>1$ then $y<1$.
\end{theorem}
\begin{proof}
Suppose $3x+2y \leq 5$, then it follows that $\tfrac{5-2y}{3} > x$. Suppose $x>1$, then $\tfrac{5-2y}{3} > x > 1$ and $\tfrac{5-2y}{3} > 1$. Then it follows that $y<1$. Therefore, if $x>1$ then $y<1$.
\end{proof}

\section*{Exercise 3.1.13}
Suppose that $x$ and $y$ are real numbers. Prove that if $x^2 + y = -3$ and $2x - y = 2$ then $x = -1$.

\n So we want to prove that 
$(x^2 + y = -3 \land 2x - y = 2) \rightarrow (x = -1)$ \\

\n First we assume the antecedent and make the consequent our goal.

\begin{table}[h]
\begin{tabular}{ll}
\hline
Givens & Goals   \\ \hline
$x^2 + y = -3$ & $x = -1$   \\
$ 2x - y = 2$ &\\ \hline
\end{tabular}
\end{table}

\n Solving $x^2 + y = -3$ for $y$ we have $y = -3-x^2$. Substituting $y = -3-x^2$ into $x^2 + y = -3$ and solving for $x$ we can conclude that $x = -1$, which was our goal prove.

\begin{theorem} Suppose $x$ and $y$ are real numbers. If $x^2 + y = -3$, and $2x - y = 2$ then $x = -1$.
\end{theorem}
\begin{proof}
Suppose $x^2 + y = -3$ and $2x - y = 2$. If $x^2 + y = -3$ then it follows that $y = -3-x^2$. Substituting $y = -3-x^2$ into the equation $x^2 + y = -3$ we can conclude that $x = -1$. Therefore, if $x^2 + y = -3$ and $2x - y = 2$ then $x = -1$.
\end{proof}

\section*{Exercise 3.1.14}
Prove the first theorem in Example 3.1.1. (Hint: You might find it useful to apply the theorem from Example 3.1.2.) \\

\n The first theorem in Example 3.1.1. is: If $x>3$ and $y<2$, then $x^2 - 2y < 5$. The theorem from Example 3.1.2 states: Suppose $a$ and $b$ are real numbers. If $0<a<b$ then $a^2 < b^2$.\\

\n So we want to prove that
$(x>3 \land y<2) \rightarrow (x^2 - 2y < 5)$. \\

\n First we assume the antecedent and make the consequent our goal. \\

\begin{table}[h]
\begin{tabular}{ll}
\hline
Givens & Goals   \\ \hline
$x > 3$ & $x^2 - 2y > 5$ \\
$y < 2$ &\\ \hline
\end{tabular}
\end{table}


\n Since $0<3<x$ we can apply theorem 3.1.1 and conclude that $x^2 > 9$. Multiplying the inequality $y <2$ by $2$ on both sides we have $2y<4$. Then adding the two inequalities $x^2 > 9$ and $4 > 2y$ we can conclude that $4 + x^2 > 9 + 2y$ and if follows that $x^2 - 2y > 5$, which was our goal to prove.

\begin{theorem} Suppose $x>3$ and $y<2$, then $x^2 - 2y < 5$.
\end{theorem}
\begin{proof}
Suppose $x>3$ and $y<2$. Since $0<3<x$ we can apply theorem 3.1.1 and conclude that $x^2 > 9$. Multiplying the inequality $y <2$ by $2$ on both sides we have $2y<4$. Then adding the two inequalities $x^2 > 9$ and $4 > 2y$ we can conclude that $4 + x^2 > 9 + 2y$. Therefore, if $x>3$ and $y<2$, then $x^2 - 2y < 5$.
\end{proof}

\section*{Exercise 3.1.15}
\subsection*{a}
The theorem has a goal of the form $a \rightarrow b$ where $a$ is $\tfrac{2x-5}{x-4}$ and $b$ is $x = 7$. To prove the theorem we could assume $a$ and prove $b$ is true or prove the contrapositive $\lnot b \rightarrow \lnot a$ and assume $\lnot b$ and prove $\lnot a$. However the proof given here shows that $b \rightarrow a$, which does not suffice to prove the theorem.

\subsection*{b}
\begin{theorem} Suppose $x$ is a real number and $x \neq 4$. If $\tfrac{2x-5}{x-4} = 3$, then $x=7$.
\end{theorem}
\begin{proof}
Suppose $\tfrac{2x-5}{x-4} = 3$, then if follows that $x=7$. Therefore if $\tfrac{2x-5}{x-4} = 3$, then $x = 7$.
\end{proof}

\section*{Exercise 3.1.16}
\subsection*{a}
The mistake is assuming that since $x \neq 3$ then $x^2 \neq 9$, which is not true because if $x = -3$ then $x^2 = 9$.
\subsection*{b}
If $x = -3$ then $-3^2y=9y$ then $9y=9y$ and $y = 1$.


\end{document}