\documentclass{article}

%\documentclass{amsart}
%\documentclass{scrartcl}
%\usepackage{changepage}
%\usepackage{scrextend}

\usepackage{amssymb,amsmath,amsthm}
% amssymb has empty set symbo
\usepackage{scrextend} % for \begin{addmargin}[0.55cm]{0cm} text \end{margin}

\usepackage{mathrsfs} % for \mathscr{P}
\usepackage{float}


\newcommand{\n}{ \noindent }
\newcommand{\F}{\mathcal{F}}
\newcommand{\G}{\mathcal{G}}


%\newcommand{\pwset}{\mathcal{P}}
\newcommand{\pwset}{\mathscr{P}}


\newtheorem*{theorem}{Theorem}  % This enables \begin{theorem}

%\DeclareFontFamily{U}{MnSymbolC}{}
%\DeclareSymbolFont{MnSyC}{U}{MnSymbolC}{m}{n}
%\DeclareFontShape{U}{MnSymbolC}{m}{n}{
%  <-6>    MnSymbolC5
%  <6-7>   MnSymbolC6
%  <7-8>   MnSymbolC7
%  <8-9>   MnSymbolC8
%  <9-10>  MnSymbolC9
%  <10-12> MnSymbolC10
%  <12->   MnSymbolC12%
%}{}
%\DeclareMathSymbol{\powerset}{\mathord}{MnSyC}{180}

%\usepackage[left=2cm,right=2cm,top=2cm,bottom=2cm]{geometry}

\begin{document}

\section*{3.6.2}
\begin{theorem} There is a unique $x \in \mathbb{R}$ such that for all $y \in \mathbb{R}$, $xy + x - 4 = 4y$.
\end{theorem}
\begin{proof}
First we prove existence. Let $x = 4$ and suppose $y$ is an arbitrary real number. Then we have $4y + 4 - 4 = 4y + 0 = 4y$, as desired. To prove uniqueness suppose $a$ and $b$ are arbitrary real numbers and that $ay + a - 4 = 4y$ and that $by + b - 4 = 4b$. For $ay + a - 4 = 4y$, let $y = b$ and we have $ab + a - 4 = 4b$. For $by + b - 4 = 4y$, let $y = a$ and we have $ba + b - 4 = 4a$. Now subtracting both sides of $ab + a - 4 = 4b$ from $ba + b - 4 = 4a$ we have

\begin{align*}
ba + b - 4 - (ab + a - 4) &= 4a - 4b \\
ba + b - 4 - ab - a + 4 &= 4a - 4b \\
b - a &= 4a - 4b \\
b + 4b &= 4a + a \\
5b &= 5a \\
b &= a
\end{align*}

Therefore, if $ay + a - 4 = 4y$ and $by + b - 4 = 4y$, then $a = b$.

\end{proof}

\section*{3.6.3}
\begin{theorem} $\forall x \in \mathbb{R} [ (x \neq 0 \land x \neq 1) \implies \exists! y \in \mathbb{R} (y/x = y - x) ]$
\end{theorem}
\begin{proof}
Suppose $x$ is an arbitrary real number, $x \neq 0$, and $x \neq 1$. Let $y = x^2/(x-1)$, which is defined because $x \neq 1$. Then

\begin{align*}
\frac{y}{x} = \frac{\frac{x^2}{x-1}}{x} = \frac{x^2}{x-1} \cdot \frac{1}{x} = \frac{x^2}{x(x-1)} &= \frac{x}{x-1} \\
&= \frac{x^2 - x^2 + x}{x-1} \\
&= \frac{x^2 - x(x-1)}{x-1} \\
& = \frac{x^2}{x-1} - \frac{x(x-1)}{x-1} = \frac{x^2}{x-1} - x = y - x
\end{align*}
\end{proof}

\section*{3.6.4}
\begin{theorem} $\forall x \in \mathbb{R} ( x \neq 0 \implies \exists! y \in \mathbb{R} \forall z \in \mathbb{R} (zy = z/x))$.
\end{theorem}

\begin{proof}
Let $x$ be an arbitrary real number. To prove existence, suppose $x \neq 0$ and $y = 1/x$. Then $zy = z(1/x) = z/x$. To prove uniqueness let $a$ and $b$ be arbitrary real numbers and suppose $za = z/x$ and $zb = z/x$. For $za = z/x$ let $z = b$ and we have $ba = bx$, which can be rearranged as $xba = b$. For $zb = z/x$ let $z = a$ and we have $ab = a/x$, which can be rearranged as $xab = a$. Subtracting both sides of $xab = b$ from $xab = a$ we have $xab - xab = a - b$ and so $a - b = 0$ or $a = b$.
\end{proof}

\section*{3.6.5}
If $\F$ is a family of sets, then $\cup \F = \{x | \exists A(A \in \F \land x \in A) \}$. Define a new set $\cup!\F$ by the formula $\cup ! \F = \{x | \exists! A (A \in \F \land x \in A )\}$.

\subsection*{(a)}
\begin{theorem} $\forall \F (\cup ! \F \subseteq \cup F)$
\end{theorem}
\begin{proof}
Suppose $\F$ is and arbitrary family of sets. Let $x$ be arbitrary and suppose $x \in \cup! \F$. This means $\exists! A \in \F (x \in A)$. Since $A \in \F$ and $x \in A$ then we can concluded that $x \in \cup \F$. Since $x$ was arbitrary then $\cup ! \F \subseteq \cup \F$, and since $\F$ was arbitrary we can conclude for all $\F$, $\cup ! \F \subseteq \cup \F$.
\end{proof}

\subsection*{(b)}
\begin{theorem} $\forall \F (\cup! \F = \cup \F ~\text{iff}~ \F ~ \text{is pairwise disjoint})$. 
\end{theorem}

\begin{flushleft} Note that pairwise disjoint means that $\forall A \in \F \forall B \in \F (A \neq B \implies A \cap B = \varnothing)$.
\end{flushleft}

\begin{proof} Let $\F$ be an arbitrary family of sets. \\
$(\rightarrow)$ Suppose $\cup! \F = \cup \F$. We will prove the contrapositive. Let $A$ and $B$ be arbitrary, suppose $A \in \F$, $B \in \F$, and $A$ and $B$ are not disjoint. Then there is an element $x$ such that $x \in A \cap B$. Since $A \in \F$ and $B \in \F$ then $x \in \cup \F$ and it follows by assumption that $x \in \cup! \F$. Since $x \in \cup! \F$ then there is a unique set $X \in \F$ such that $x \in X$ and so $x \in A = X$ and $x \in B = X$. Therefore $A = B$.

$(\leftarrow)$ Now suppose that $\F$ is pairwise disjoint. We need to show that $\cup! \F = \cup \F$, which means $\cup! \F \subseteq \cup F$ and $\cup \F \subseteq \cup ! \F$. 

To see that $\cup! \F \subseteq \cup \F$, let $y$ be arbitrary and suppose $y \in \cup! \F$. Then there is a unique set $Y \in \F$ such that $y \in Y$ and therefore $y \in \cup \F$. Since $y$ was arbitrary, this shows that $\cup! \F \subseteq \cup \F$.

To see that $\cup \F \subseteq \cup! \F$, now suppose $y \in \cup \F$. Then there is a set $Y \in \F$ such that $y \in Y$. To see that $Y$ is unique, suppose there is another set $Z \in \F$ such that $y \in Z$. By assumption, $\F$ is pairwise disjoint and since $y \in Y$ and $y \in Z$, it follows that $Y = Z$. Thus there is a unique set $Y \in \F$ such that $y in Y$ and therefore $y \in \cup! \F$. Since $y$ was arbitrary this shows that $\cup \F \subseteq \cup! \F$.

We have shown that $\cup! \F \subseteq \cup F$ and $\cup \F \subseteq \cup ! \F$ and therefore $\cup! \F = \cup \F$.
\end{proof}


\end{document}