\documentclass{article}

%\documentclass{amsart}
%\documentclass{scrartcl}
%\usepackage{changepage}
%\usepackage{scrextend}

\usepackage{amssymb,amsmath,amsthm}
% amssymb has empty set symbo
\usepackage{scrextend} % for \begin{addmargin}[0.55cm]{0cm} text \end{margin}

\usepackage{mathrsfs} % for \mathscr{P}
\usepackage{float}


\newcommand{\n}{ \noindent }
\newcommand{\F}{\mathcal{F}}
\newcommand{\G}{\mathcal{G}}


%\newcommand{\pwset}{\mathcal{P}}
\newcommand{\pwset}{\mathscr{P}}


\newtheorem*{theorem}{Theorem}  % This enables \begin{theorem}

%\DeclareFontFamily{U}{MnSymbolC}{}
%\DeclareSymbolFont{MnSyC}{U}{MnSymbolC}{m}{n}
%\DeclareFontShape{U}{MnSymbolC}{m}{n}{
%  <-6>    MnSymbolC5
%  <6-7>   MnSymbolC6
%  <7-8>   MnSymbolC7
%  <8-9>   MnSymbolC8
%  <9-10>  MnSymbolC9
%  <10-12> MnSymbolC10
%  <12->   MnSymbolC12%
%}{}
%\DeclareMathSymbol{\powerset}{\mathord}{MnSyC}{180}

%\usepackage[left=2cm,right=2cm,top=2cm,bottom=2cm]{geometry}

\begin{document}

\section*{3.6.2}
\begin{theorem} There is a unique $x \in \mathbb{R}$ such that for all $y \in \mathbb{R}$, $xy + x - 4 = 4y$.
\end{theorem}
\begin{proof}
First we prove existence. Let $x = 4$ and suppose $y$ is an arbitrary real number. Then we have $4y + 4 - 4 = 4y + 0 = 4y$, as desired. To prove uniqueness suppose $a$ and $b$ are arbitrary real numbers and that $ay + a - 4 = 4y$ and that $by + b - 4 = 4b$. For $ay + a - 4 = 4y$, let $y = b$ and we have $ab + a - 4 = 4b$. For $by + b - 4 = 4y$, let $y = a$ and we have $ba + b - 4 = 4a$. Now subtracting both sides of $ab + a - 4 = 4b$ from $ba + b - 4 = 4a$ we have

\begin{align*}
ba + b - 4 - (ab + a - 4) &= 4a - 4b \\
ba + b - 4 - ab - a + 4 &= 4a - 4b \\
b - a &= 4a - 4b \\
b + 4b &= 4a + a \\
5b &= 5a \\
b &= a
\end{align*}

Therefore, if $ay + a - 4 = 4y$ and $by + b - 4 = 4y$, then $a = b$.

\end{proof}

\section*{3.6.3}
\begin{theorem} $\forall x \in \mathbb{R} [ (x \neq 0 \land x \neq 1) \implies \exists! y \in \mathbb{R} (y/x = y - x) ]$
\end{theorem}
\begin{proof}
Suppose $x$ is an arbitrary real number, $x \neq 0$, and $x \neq 1$. Let $y = x^2/(x-1)$, which is defined because $x \neq 1$. Then

\begin{align*}
\frac{y}{x} = \frac{\frac{x^2}{x-1}}{x} = \frac{x^2}{x-1} \cdot \frac{1}{x} = \frac{x^2}{x(x-1)} &= \frac{x}{x-1} \\
&= \frac{x^2 - x^2 + x}{x-1} \\
&= \frac{x^2 - x(x-1)}{x-1} \\
& = \frac{x^2}{x-1} - \frac{x(x-1)}{x-1} = \frac{x^2}{x-1} - x = y - x
\end{align*}

\end{proof}

\end{document}